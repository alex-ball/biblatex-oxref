% \iffalse
%<*internal|bbx-0|bbx-n|bbx-y|cbx-n|cbx-y|dbx|lbx-gb|doc-n|doc-y>
\def\Version{2015/05/25 v0.1}
%</internal|bbx-0|bbx-n|bbx-y|cbx-n|cbx-y|dbx|lbx-gb|doc-n|doc-y>
%<*internal>
\iffalse
%</internal>
%<*bbx-0|bbx-n|bbx-y|cbx-n|cbx-y|dbx|lbx-gb>
\NeedsTeXFormat{LaTeX2e}[1999/12/01]
%</bbx-0|bbx-n|bbx-y|cbx-n|cbx-y|dbx|lbx-gb>
%<*bbx-0>
\ProvidesFile{oxref.bbx}
    [\Version\space Base settings for bibliography styles inspired by the Oxford Guide to Style]
%</bbx-0>
%<*bbx-n>
\ProvidesFile{oxnotes.bbx}
    [\Version\space Footnote-based bibliography style inspired by the Oxford Guide to Style]
%</bbx-n>
%<*bbx-y>
\ProvidesFile{oxyear.bbx}
    [\Version\space Author-year bibliography style inspired by the Oxford Guide to Style]
%</bbx-y>
%<*cbx-n>
\ProvidesFile{oxnotes.cbx}
    [\Version\space Footnote-based citation style inspired by the Oxford Guide to Style]
%</cbx-n>
%<*cbx-y>
\ProvidesFile{oxyear.cbx}
    [\Version\space Author-year citation style inspired by the Oxford Guide to Style]
%</cbx-y>
%<*dbx>
\ProvidesFile{oxref.dbx}
    [\Version\space Data model for the Oxref family of styles]
%</dbx>
%<*lbx-gb>
\ProvidesFile{british-oxref.lbx}
    [\Version\space British English conventions required by the biblatex-oxref styles]
%</lbx-gb>
%<*doc-n>
\ProvidesFile{oxnotes-doc.tex}
    [\Version\space Footnote-based biblatex style inspired by the Oxford Guide to Style]
%</doc-n>
%<*doc-y>
\ProvidesFile{oxyear-doc.tex}
    [\Version\space Author-year biblatex style inspired by the Oxford Guide to Style]
%</doc-y>
%<*readme>
# biblatex-oxref: Biblatex styles inspired by the *Oxford Guide to Style*
%</readme>
%<*bib>
@reference{ritter2002ogs,
  author = {R. M. Ritter},
  title = {The {Oxford} Guide to Style},
  date = {2002},
  publisher = {Oxford University Press},
  location = {Oxford, UK}
}

@reference{waddingham2014nhr,
  editor = {Anne Waddingham},
  title = {New {Hart's} Rules},
  subtitle = {The {Oxford} Guide to Style},
  date = {2014},
  edition = {2},
  publisher = {Oxford University Press},
  location = {Oxford, UK}
}
%</bib>
%<*internal>
\fi
\def\nameofplainTeX{plain}
\ifx\fmtname\nameofplainTeX\else
  \expandafter\begingroup
\fi
%</internal>
%<*install>
\input docstrip.tex
\keepsilent
\askforoverwritefalse

\nopreamble\nopostamble

\usedir{doc/latex/\jobname}
\generate{
  \file{README.md}{\from{\jobname.dtx}{readme}}
  \file{\jobname.bib}{\from{\jobname.dtx}{bib}}
}

\preamble
----------------------------------------------------------------
biblatex-oxref --- Biblatex styles inspired by the Oxford Guide to Style
Author:  Alex Ball
E-mail:  a.j.ball@bath.ac.uk
License: Released under the LaTeX Project Public License v1.3c or later
See:     http://www.latex-project.org/lppl.txt
----------------------------------------------------------------

\endpreamble
\postamble

Copyright (C) 2016 Alex Ball
\endpostamble

\usedir{tex/latex/\jobname}
\generate{
  \file{\jobname.bbx}{\from{\jobname.dtx}{bbx-0}}
  \file{oxnotes.bbx}{\from{\jobname.dtx}{bbx-n}}
  \file{oxyear.bbx}{\from{\jobname.dtx}{bbx-y}}
  \file{oxnotes.cbx}{\from{\jobname.dtx}{cbx-n}}
  \file{oxyear.cbx}{\from{\jobname.dtx}{cbx-y}}
  \file{\jobname.dbx}{\from{\jobname.dtx}{dbx}}
  \file{british-\jobname.lbx}{\from{\jobname.dtx}{lbx-gb}}
  \file{oxnotes-doc.tex}{\from{\jobname.dtx}{doc-n}}
  \file{oxyear-doc.tex}{\from{\jobname.dtx}{doc-y}}
}
%</install>
%<install>\endbatchfile
%<*internal>
\usedir{source/latex/\jobname}
\generate{
  \file{\jobname.ins}{\from{\jobname.dtx}{install}}
}
\ifx\fmtname\nameofplainTeX
  \expandafter\endbatchfile
\else
  \expandafter\endgroup
\fi
%</internal>
%<*driver>
\ProvidesFile{oxref.dtx}
    [\Version\space Biblatex styles inspired by the Oxford Guide to Style]
\PassOptionsToPackage{style=oxnotes}{biblatex}
%</driver>
%<doc-n>\PassOptionsToPackage{style=oxnotes,scnames}{biblatex}
%<doc-y>\PassOptionsToPackage{style=oxyear}{biblatex}
%<*driver|doc-n|doc-y>
\documentclass[extrafontsizes,11pt,a4paper,oneside]{memoir}
\setlrmarginsandblock{4cm}{2cm}{*}
\setulmarginsandblock{2.5cm}{2.5cm}{*}
\checkandfixthelayout
% Document divisions
\chapterstyle{ell}
\hangsecnum
\setsecheadstyle{\Large\bfseries\raggedright}
\setsubsecheadstyle{\large\bfseries\scshape\raggedright}
\setsecnumdepth{subsection}
% Pagination and headers
\nouppercaseheads
\makeoddhead{myheadings}{\textsc{\leftmark}}{}{\thepage}
%\makeevenhead{myheadings}{\thepage}{}{\textsc{\leftmark}}
\makepsmarks{myheadings}{%
  \def\chaptermark##1{\markboth{##1}{##1}}%
  \def\sectionmark##1{\markright{##1}}%
}
\pagestyle{myheadings}
\aliaspagestyle{title}{empty}
% Paragraphs and lists
\setlength{\parindent}{0pt}\nonzeroparskip
\firmlists

\usepackage[british]{babel}
\usepackage[mono=false,defaultfeatures={SmallCapsFeatures={Letters=SmallCaps,Renderer=Basic,Ligatures=NoCommon}}]{libertine}
\usepackage{fontawesome}[2015/07/07]
\newcommand{\booksym}{\makebox[1em][c]{\faicon{book}}}
\newcommand{\cogsym}{\makebox[1em][c]{\faicon{cog}}}
\makeatletter
\@ifpackageloaded{fontspec}{%
  \setmonofont[Scale=MatchLowercase,StylisticSet=1,AutoFakeSlant]{Inconsolatazi4}
}{%
  \usepackage[utf8]{inputenc}
  \usepackage[varl]{zi4}
}
\makeatother

\usepackage{xpatch,csquotes,xcolor,xparse,multicol}
\definecolor{Green}{rgb}{0,.5,0}
\colorlet{ok}{Green}
\colorlet{todo}{red}
\colorlet{hacked}{orange}
\colorlet{manual}{purple}

\usepackage{tcolorbox}
\tcbuselibrary{listings,breakable,skins,xparse}
\lstloadlanguages{[LaTeX]TeX}
\lstset
  { columns=fullflexible
  , basicstyle=\ttfamily
  , language={[LaTeX]TeX}
  , texcsstyle=*\color{red!75!black}
  , moredelim=**[s][\color{violet}]{[}{]}
  , moredelim=**[s][\color{blue!75!black}]{\{}{\}}
  , mathescape
  , escapechar=~
  }

\usepackage
[backend=biber%
,hyperref=false%
%,isbn=false%
]{biblatex}
\bibliography{oxref}
\usepackage[pdfborder={0 0 0},pdfencoding=auto]{hyperref}
\usepackage[noabbrev,capitalize,nameinlink]{cleveref}

\newcommand*{\pkg}[1]{\href{http://www.ctan.org/pkg/#1}{\textsf{#1}}}
\newcommand*{\lit}[1]{\textsf{#1}}
\newcommand*{\code}[1]{`\texttt{#1}'}
\newcommand*{\aside}[1]{\textcolor{violet}{[\textsc{tip:} #1]}}

% For debugging; userd is either ‘todo’, ‘hacked’, ‘manual’
\makeatletter
\def\CiteStatus{todo}
\newcommand{\dbgcolor}[2]{%
  \bgroup
  \blx@citecmdinit
  \blx@citeinit
  \def\blx@precode{}%
  \def\blx@postcode{}%
  \def\blx@loopcode{%
    \iffieldundef{userd}
    {\xdef\CiteStatus{ok}}
    {\xdef\CiteStatus{\abx@field@userd}}}%
  \blx@citeloop{#1}%
  \textcolor{\CiteStatus}{#2}%
  \egroup
}
\makeatother
\NewTColorBox{bibexbox}{D(){ok}d<>om}%
  {bicolor
  ,colframe = #1
  ,colback = #1!5!white
  ,colbacklower = white
  ,fontlower = \footnotesize
  ,before upper = {\hangfrom{\booksym\space}}
  ,IfNoValueTF={#3}%
    {after upper = {\par\hangfrom{\cogsym\space}\fullcite{#4}.}
    }%
    {after upper = {\par\hangfrom{\cogsym\space}\fullcite[#3]{#4}.}
    ,title = {\texttt{\string\fullcite[#3]\{#4\}}}
    }
  ,IfNoValueTF={#2}{}%
    {overlay = {
      \node[anchor=south east,text=teal] at (frame.south east) {#2};
      }
    }
  }
\NewTotalTColorBox{\spec}{m}%
  {enhanced
  ,sharp corners = west
  ,colframe = teal
  ,colback = teal!5!white
  ,toprule = 0pt
  ,bottomrule = 0pt
  ,rightrule = 0pt
  }{#1}
% \egcite gives a citation as an example. Arguments:
% #1 = page range
% #2 = bib key
\NewDocumentCommand\egcite{om}{%
  \makebox[0pt][r]{\footnotesize\textcolor{teal}{#2}\quad}%
  \IfNoValueTF{#1}{\dbgcolor{#2}{\fullcite{#2}.}}%
  {\dbgcolor{#2}{\fullcite[#1]{#2}}.}}

\frenchspacing
%</driver|doc-n|doc-y>
%<*driver>

% For typesetting the implementation
\makeatletter
\let\theglossary\@undefined
\let\endtheglossary\@undefined
\makeatother
\usepackage{doc}
\makeatletter
\renewenvironment{theglossary}{%
  \bgroup
    \glossary@prologue
    \GlossaryParms \let\item\@idxitem \ignorespaces
}{%
  \egroup
}
\makeatother
\tcbuselibrary{documentation,breakable,skins,minted}
\colorlet{Option}{violet}
\colorlet{Command}{red!75!black}
\colorlet{Environment}{blue!75!black}
\colorlet{Value}{olive!75!black}
\colorlet{Color}{cyan!75!black}
\tcbset
  { enhanced
  , listing engine=minted
  , minted options=
    { breaklines
    , fontsize=\footnotesize
    , linenos
    , numbersep=20pt
    , firstnumber=last
    }
  , index format=pgf
  , color command=Command
  , color environment=Environment
  , color key=Option
  , color value=Value
  , color color=Color
  }
\renewcommand{\theFancyVerbLine}{\footnotesize\itshape\color{gray}\arabic{FancyVerbLine}}
\let\tcbcs=\cs
\renewcommand*{\cs}[1]{\textcolor{Command}{\tcbcs{#1}}}
\def\sqbrackets#1{%
  \texttt{\textcolor{Option}{[}#1\textcolor{Option}{]}}}
\def\brackets#1{%
  \texttt{\textcolor{Environment}{\char`\{}#1\textcolor{Environment}{\char`\}}}}
\def\marg#1{%
  \textcolor{Environment}{\ttfamily\char`\{}\meta{#1}\textcolor{Environment}{\ttfamily\char`\}}}
\newcommand*{\env}[1]{\textcolor{Environment}{\ttfamily #1}}
\newcommand*{\key}[1]{\textcolor{Option}{\ttfamily #1}}
\newcommand*{\val}[1]{\textcolor{Value}{\ttfamily #1}}
\renewenvironment{macro}[1]{%
  \def\MyName{#1}%
  \index{\MyName@\tcbIndexPrintComC {\MyName}|(emph}%
}{%
  \ifdef{\MyName}{}{%
    \errmessage{You have closed a macro environment you have not opened on \the\inputlineno.}%
  }
  \index{\MyName@\tcbIndexPrintComC {\MyName}|)}%
}
\renewenvironment{environment}[1]{%
  \def\MyName{#1}%
  \index{\MyName@\tcbIndexPrintEnvCA {\MyName}|(emph}%
  \index{Environments!\MyName@\tcbIndexPrintEnvC {\MyName}|(emph}%
}{%
  \ifdef{\MyName}{}{%
    \errmessage{You have closed an environment environment you have not opened on \the\inputlineno.}%
  }
  \index{Environments!\MyName@\tcbIndexPrintEnvC {\MyName}|)}%
  \index{\MyName@\tcbIndexPrintEnvCA {\MyName}|)}%
}
\newenvironment{optionkey}[1]{%
  \def\MyName{#1}%
  \index{\MyName@\tcbIndexPrintKeyCA {\MyName}|(emph}%
  \index{Keys!\MyName@\tcbIndexPrintKeyC {\MyName}|(emph}%
}{%
  \ifdef{\MyName}{}{%
    \errmessage{You have closed an optionkey environment you have not opened on \the\inputlineno.}%
  }
  \index{Keys!\MyName@\tcbIndexPrintKeyC {\MyName}|)}%
  \index{\MyName@\tcbIndexPrintKeyCA {\MyName}|)}%
}
\newenvironment{optionvalue}[1]{%
  \def\MyName{#1}%
  \index{\MyName@\tcbIndexPrintValCA {\MyName}|(emph}%
  \index{Values!\MyName@\tcbIndexPrintValC {\MyName}|(emph}%
}{%
  \ifdef{\MyName}{}{%
    \errmessage{You have closed an optionvalue environment you have not opened on \the\inputlineno.}%
  }
  \index{Values!\MyName@\tcbIndexPrintValC {\MyName}|)}%
  \index{\MyName@\tcbIndexPrintValCA {\MyName}|)}%
}
\makeatletter
\newcommand{\resetmintedformat}{%
  % Comments
  \expandafter\def\csname PYGdefault@tok@c\endcsname{\let\PYGdefault@it=\textit\def\PYGdefault@tc####1{\textcolor{gray}{####1}}}
  % Command sequences
  \expandafter\def\csname PYGdefault@tok@k\endcsname{\def\PYGdefault@tc####1{\textcolor{Command}{####1}}}
  % Optional arguments
  \expandafter\def\csname PYGdefault@tok@na\endcsname{\def\PYGdefault@tc####1{\textcolor{Option}{####1}}}
  % Braces
  \expandafter\def\csname PYGdefault@tok@nb\endcsname{\def\PYGdefault@tc####1{\textcolor{Environment}{####1}}}
}
\apptocmd{\minted@checkstyle}{\resetmintedformat}{}{}
\makeatother
\MakeShortVerb{\|}
\makeatletter
\let\PrintMacroName\@gobble
\let\PrintEnvName\@gobble
\renewenvironment{tcb@manual@entry}{\begin{list}{}{%
  \setlength{\topsep}{0pt}
  \setlength{\partopsep}{0pt}
  \setlength{\leftmargin}{\kvtcb@doc@left}%
  \setlength{\itemindent}{0pt}%
  \setlength{\itemsep}{0pt}%
  \setlength{\parsep}{0pt}%
  \setlength{\rightmargin}{\kvtcb@doc@right}%
  }\item}{\end{list}}
\makeatother

% This bit inspired by ydoc
\makeatletter
\newwrite\ydocwrite
\def\ydocfname{\jobname.listing}
\def\ydoc@catcodes{%
  \let\do\@makeother
  \dospecials
  \catcode`\\=\active
  \catcode`\^^M=\active
  \catcode`\ =\active
}
\def\macrocode{%
  \begingroup
  \ydoc@catcodes
  \macro@code
}
\def\endmacrocode{}
\begingroup
\endlinechar\m@ne
\@firstofone{%
\catcode`\|=0\relax
\catcode`\(=1\relax
\catcode`\)=2\relax
\catcode`\*=14\relax
\catcode`\{=12\relax
\catcode`\}=12\relax
\catcode`\ =12\relax
\catcode`\%=12\relax
\catcode`\\=\active
\catcode`\^^M=\active
\catcode`\ =\active
}*
|gdef|macro@code#1^^M%    \end{macrocode}(*
|endgroup|expandafter|macro@@code|expandafter(|ydoc@removeline#1|noexpand|lastlinemacro)*
)*
|gdef|ydoc@removeline#1^^M(|noexpand|firstlinemacro)*
|gdef|ydoc@defspecialmacros(*
|def^^M(|noexpand|newlinemacro)*
|def (|noexpand|spacemacro)*
|def\(|noexpand|bslashmacro)*
)*
|gdef|ydoc@defrevspecialmacros(*
|def|newlinemacro(|noexpand^^M)*
|def|spacemacro(|noexpand )*
|def|bslashmacro(|noexpand\)*
)*
|endgroup
\def\macro@@code#1{%
  {\ydoc@defspecialmacros
  \xdef\themacrocode{#1}}%
  \PrintMacroCode
  \end{macrocode}%
}
\def\PrintMacroCode{%
  \begingroup
  \let\firstlinemacro\empty
  \let\lastlinemacro\empty
  \def\newlinemacro{^^J}%
  \let\bslashmacro\bslash
  \let\spacemacro\space
  \immediate\openout\ydocwrite=\ydocfname\relax
  \immediate\write\ydocwrite{\themacrocode}%
  \immediate\closeout\ydocwrite
  \let\input\@input
  \tcbinputlisting{breakable,listing only,docexample,listing file=\ydocfname}%
  \endgroup
}
\makeatother

\title{OXREF -- Biblatex styles inspired by the \emph{Oxford Guide to Style}}
%</driver>
%<doc-n>\title{OXNOTES -- A notes-based style for Biblatex}
%<doc-y>\title{OXYEAR -- An author–year style for Biblatex}
%<*driver|doc-n|doc-y>
\author{Alex Ball}
\begin{document}
\thispagestyle{empty}
\begin{adjustwidth}{.2\textwidth}{0pt}
  \sffamily\setlength{\parindent}{0pt}%
  \LARGE\textsc{oxref bundle}

  \vspace{\stretch{1}}
  \LARGE\thetitle

  \bigskip
  \Large\theauthor

  \bigskip
  \Large\ifx\undefined\thedate\today\else\thedate\fi
\end{adjustwidth}

\vspace{\stretch{3}}
\noindent
\hspace*{.1\textwidth}\raisebox{0pt}[0pt][0pt]{\rule{\normalrulethickness}{\textheight}}

%</driver|doc-n|doc-y>
%<*driver>
\chapter*{Preface}

This document provides the documented sources for the \textsf{oxref}
biblography styles:
\begin{itemize}
  \item
  \textsf{oxnotes}: a style similar to the standard \textsf{verbose}, intended for use with footnotes;
  \item
  \textsf{oxyear}: a style similar to the standard \textsf{authoryear}, intended for use with parenthetical in-text citations.
\end{itemize}
For information on how to use the styles, please refer to the separate documentation files
\texttt{oxnotes-doc.pdf} and \texttt{oxyear-doc.pdf} respectively.

\tableofcontents*

\iffalse % syntax highlighting \fix
%</driver>
%<*doc-n|doc-y>
\newpage
\tableofcontents*

\chapter{Introduction}

\section{Loading the style}\label{sec:loading}

The style is self-contained, so you can load it with \pkg{biblatex}:
\begin{tcblisting}{listing only}
%<doc-n>\usepackage[style=oxnotes]{biblatex}
%<doc-y>\usepackage[style=oxyear]{biblatex}
\end{tcblisting}

The style has some options additional to the regular \pkg{biblatex} ones:

\begin{description}
  \item[altthesis]
  Puts the thesis type inside, rather than outside, the parenthetical publication block.
  \item[issuestyle]
  Affects how journal volumes and numbers are printed.
  \begin{description}
  \item[slash]
  (default) Separates the two with a solidus, e.g. ‘23/2’.
  \item[colon]
  Separates the two with a colon and space, e.g. ‘23: 2’.
%<doc-y>  It is intended for use with \textsf{oxnotes} and not recommended for this style.
  \item[comma]
  Separates the two with a comma and space, e.g. ‘23, 2’.
  \item[parens]
  Sets off the issue number in parentheses, e.g. ‘23 (2)’.
  \end{description}
  \item[nopublisher]
  Removes publisher name from all entries.
%<doc-n>  \item[scnames]
%<doc-n>  Prints initial author or editor names in bibliography entries in small capitals.
\end{description}

\textsf{Oxref} makes of Biber-specific techniques to solve some of the challenges
presented by Oxford style. If you resort to using Bib\TeX\ instead,
you will have to correct the sort order of certain entries manually with
\texttt{sortkey}, and use some messy workarounds to get things like reviews
working properly.

\section{How to use this document}

Bibliographical items are given throughout this document, and serve three purposes:

\begin{itemize}
  \item
  To demonstrate which of the variations discussed by the
  \emph{Oxford Guide to Style} and \emph{New Hart's Rules} have been chosen
  in this style.
  \item
  To help me, as package author, check that the style is working as intended.
  \item
  To show you, as document author, how to use the style to get the effect you want.
\end{itemize}

Examples that follow standard \textsf{biblatex} semantics, and are therefore
(mostly) portable to other styles, are in green:

\begin{tcolorbox}%
  [bicolor
  ,colframe = ok
  ,colback = ok!5!white
  ,colbacklower = white
  ,fontlower = \footnotesize\ttfamily
  ,overlay = {\node[anchor=south east,text=teal] at (frame.south east) {Source};}
  ]
  \hangfrom{\booksym\space} Reference text as it should look.\par
  \hangfrom{\cogsym\space} Reference text as generated by \textsf{biblatex}.
  \tcblower
  Code used in bibliography file.
\end{tcolorbox}

Examples that have been ‘hacked’ in some way, and are therefore not portable
to other styles, are in amber:

\begin{tcolorbox}%
  [bicolor
  ,colframe = hacked
  ,colback = hacked!5!white
  ,colbacklower = white
  ,fontlower = \footnotesize\ttfamily
  ,overlay = {\node[anchor=south east,text=teal] at (frame.south east) {Source};}
  ]
  \hangfrom{\booksym\space} Reference text as it should look.\par
  \hangfrom{\cogsym\space} Reference text as generated by \textsf{biblatex}.
  \tcblower
  Code used in bibliography file.
\end{tcolorbox}

Where a source is provided, it refers to a section from one of the reference works below:

\begin{description}
\item[OGS]
\fullcite{ritter2002ogs}

\item[NHR]
\fullcite{waddingham2014nhr}
\end{description}

Where the source is starred, this indicates the example is not quite how it appears in the book,
usually because the original is demonstrating an option that is not the \textsf{oxref} default.
%<doc-y>In addition, some reference texts are marked as \emph{[conv.]}:
%<doc-y>these have been coverted from \textsf{oxnotes}-style examples,
%<doc-y>since the guides provide relatively few for author-year references.

\section{Design philosophy}

The stipulations given by the \emph{Oxford Guide to Style} regarding citations and references
amount not so much to a consistent style as a body of advice for creating one.
Unlike the style manuals published by the American Psychological Association,
the Modern Language Association, and the University of Chicago,
the emphasis of the Oxford guide is to describe good practice rather than prescribe it.
This means that when it comes to \textsf{oxref}, there are decisions to be made
as to which variations to support by default, which to support as options,
and which to quietly ignore.

The situation is complicated further by the nature of the available versions.
The 2002 guide remains the most comprehensive in terms of rules, principles and examples.
The succeeding versions, under the title \emph{New Hart's Rules},
update the aspects of the 2002 guide that now seem somewhat dated,
such as its handling of URLs and DOIs,
and have more of an eye on machine processing of bibliographies.
They do, however, introduce some more variations with less of a steer on what is preferred,
and are considerably shorter with fewer examples.

The approach of \textsf{oxref} is to follow the 2014 \emph{New Hart's Rules} as much as possible,
but where variations are given without strong preference, or where guidance is lacking,
to follow the preferences of the 2002 guide.
Where the practices of the humanities and the sciences are contrasted,
the former are followed for \textsf{oxnotes} and the latter for \textsf{oxyear}.
Where neither version gives explicit guidance on citing a resource supported by \textsf{biblatex},
\textsf{oxref} extrapolates from what is provided, guided by standard \textsf{biblatex}
and other major referencing styles.

Where it is practical to do so,
ways and means of achieving the variations defined by the two guides are provided,
but by no means all of them will be supported.

\section{To do}

The \textsf{oxref} family of \textsf{biblatex} styles does not yet support the following standard entry types
(or at least, they have not yet been styled appropriately):

\begin{multicols}{2}
  \begin{itemize}\firmlist
    \item booklet
    \item manual (code written but untested)
    \item misc
    \item patent
    \item periodical
    \item suppperiodical
  \end{itemize}
\end{multicols}

Support is also planned for the following additional types (starred types are not defined by \textsf{biblatex}):

\begin{multicols}{2}
  \begin{itemize}\firmlist
    \item dataset*
    \item image
    \item letter
    \item manuscript*
    \item play*
    \item poem*
    \item software
    \item standard
    \item video
  \end{itemize}
\end{multicols}

Additional tasks:

\begin{itemize}
  \item Provide mechanism for properly rendering pseudepigraphical names.
  \item Provide mechanism for using ‘Anonymous’ as an author name.
  \item Provide citation style that matches the \emph{OGS} suggestions for shorthand expressions in notes.
\end{itemize}

%</doc-n|doc-y>
%<*driver>
\fi
\DocInput{\jobname.dtx}
%</driver>
%<*driver|doc-n|doc-y>
\end{document}
%</driver|doc-n|doc-y>
%<*bbx-0>
% \fi
% \chapter{Bibliography styles}
%
% \section{Base style: oxref.bbx}
%
% \setcounter{FancyVerbLine}{19}
%
% \subsection{Preliminaries}
%
% Dependencies:
% \begin{itemize}
% \item
% For ease of maintenance, we will patch some definitions with \pkg{xpatch}
% instead of writing out our own in full.
% \item
% We will manipulate strings with \pkg{xstring}.
% \item
% We will use \pkg{graphicx} for stretching \cs{bibnamedash}es
% \end{itemize}
%
%    \begin{macrocode}
\RequirePackage{xpatch}
\RequirePackage{xstring}
\RequirePackage{graphicx}
%    \end{macrocode}
%
% Here we set some defaults different to the standard ones. The author
% can still override them.
%
%    \begin{macrocode}
\ExecuteBibliographyOptions{urldate=comp,pagetracker,timezeros=false,time=12h}
%    \end{macrocode}
%
% Language support may be widened in future, but for now we support the
% following:
%
%    \begin{macrocode}
\DeclareLanguageMapping{english}{british-oxref}
\DeclareLanguageMapping{british}{british-oxref}
\DeclareLanguageMapping{american}{british-oxref}
%    \end{macrocode}
%
% We provide some additional bibliography strings.
%
%    \begin{macrocode}
\NewBibliographyString{at}
\NewBibliographyString{under}
\NewBibliographyString{facsimile}
\NewBibliographyString{revised}
\NewBibliographyString{byperformer}
\NewBibliographyString{bydirector}
\NewBibliographyString{byreader}
%    \end{macrocode}
%
% We provide some magic values for triggering formatting changes.
%
% \begin{itemize}
% \item \cs{isInferred} relates to author names;
% \item \cs{whenPublished} relates to URL dates.
% \end{itemize}
%
%    \begin{macrocode}
\def\isInferred{inferred}
\def\whenPublished{published}
%    \end{macrocode}
%
% We base our styles on the standard on the principle of least surprise
% (and to aid with maintenance in the face of new \pkg{biblatex} versions).
%
%    \begin{macrocode}
\RequireBibliographyStyle{standard}
%    \end{macrocode}
%
% Here are the default punctuation conventions:
%
%    \begin{macrocode}
\renewcommand*{\labelnamepunct}{\addcomma\space}
\renewcommand*{\newunitpunct}{\addcomma\space}
\renewcommand*{\subtitlepunct}{\addcolon\space}
\renewcommand*{\intitlepunct}{\nopunct\space}
\renewcommand*{\bibnamedash}{\resizebox{2em}{\height}{\textemdash}\addthinspace}
\renewcommand*{\revsdnamedelim}{\addcomma}
%    \end{macrocode}
%
% \subsection{Name handling}
%
% We declare some new formats so that authors/editors/others who appear
% mid-reference can be handled differently from those that appear at the
% head of the reference
%
%    \begin{macrocode}
\DeclareNameAlias{bookauthor}{default}
\DeclareNameAlias{bookeditor}{default}
%    \end{macrocode}
%
% Traditionally, Oxford style (for the humanities) prints author names in
% small capitals in the bibliography, but in normal case in citations. This is
% falling out of fashion, so we introduce this as an option.
%
%    \begin{macrocode}
\newtoggle{blx@ox@scnames}
\DeclareBibliographyOption[boolean]{scnames}[true]{%
  \settoggle{blx@ox@scnames}{#1}
}
%    \end{macrocode}
%
% Oxford style (for the humanities) is to write author names surname first in
% the bibliography, but in natural order in citations. We implement this with
% the `bib-family-given/cite-given-family' name format. The case changing option
% only affects this name format.
%
%    \begin{macrocode}
\DeclareNameFormat{bib-family-given/cite-given-family}{%
  \ifbibliography{%
    \iftoggle{blx@ox@scnames}{%
      \renewcommand*{\mkbibnamefamily}[1]{\textsc{##1}}%
      \renewcommand*{\mkbibnamegiven}[1]{\textsc{##1}}%
      \renewcommand*{\mkbibnameprefix}[1]{\textsc{##1}}%
      \renewcommand*{\mkbibnamesuffix}[1]{\textsc{##1}}%
    }{}%
    \ifgiveninits{%
      \usebibmacro{name:family-given}%
        {\namepartfamily}%
        {\namepartgiveni}%
        {\namepartprefix}%
        {\namepartsuffix}%
    }{%
      \usebibmacro{name:family-given}%
        {\namepartfamily}%
        {\namepartgiven}%
        {\namepartprefix}%
        {\namepartsuffix}%
    }%
    \iftoggle{blx@ox@scnames}{%
      \renewcommand*{\mkbibnamefamily}[1]{##1}%
      \renewcommand*{\mkbibnamegiven}[1]{##1}%
      \renewcommand*{\mkbibnameprefix}[1]{##1}%
      \renewcommand*{\mkbibnamesuffix}[1]{##1}%
    }{}%
  }{%
    \ifgiveninits{%
      \usebibmacro{name:given-family}%
        {\namepartfamily}%
        {\namepartgiveni}%
        {\namepartprefix}%
        {\namepartsuffix}%
    }{%
      \usebibmacro{name:given-family}%
        {\namepartfamily}%
        {\namepartgiven}%
        {\namepartprefix}%
        {\namepartsuffix}%
    }%
  }%
  \usebibmacro{name:andothers}%
}
%    \end{macrocode}
%
% When using natural order for names, there is no comma before numeric
% suffixes, but there is before textual ones (e.g. ‘junior’).
%
%    \begin{macrocode}
\xpatchbibmacro{name:given-family}%
  {\bibnamedelimd\mkbibnamesuffix{#4}}%
  {\ifnumeral{#4}{}{\addcomma}\bibnamedelimd\mkbibnamesuffix{#4}}{}{}
%    \end{macrocode}
%
% For pseudonyms, Oxford style is to put the other name in parentheses
% after the main one. \pkg{Biblatex} provides the `nameaddon' field for this
% use case.
%
%    \begin{macrocode}
\DeclareFieldFormat{nameaddon}{\mkbibparens{#1}}
%    \end{macrocode}
%
% Traditional Oxford style is to use dashes instead of repeating author
% names, though NHR recommends abandoning the practice as it interferes
% with text mining. As per the standard `authortitle' and `authoryear'
% styles, we implement dashes by default but allow authors to switch
% this behaviour off with `dashed=false'.
%
%    \begin{macrocode}
\newbibmacro*{bbx:savehash}{%
  \savefield{fullhash}{\bbx@lasthash}}
\DeclareBibliographyOption{dashed}[true]{%
  \ifstrequal{#1}{true}{%
    \ExecuteBibliographyOptions{pagetracker}%
    \renewbibmacro*{bbx:savehash}{\savefield{fullhash}{\bbx@lasthash}}%
  }{%
    \renewbibmacro*{bbx:savehash}{}%
  }%
}
\InitializeBibliographyStyle{%
  \global\undef\bbx@lasthash}
\newbibmacro*{bbx:dashcheck}[2]{%
  \ifboolexpr{
    test {\iffieldequals{fullhash}{\bbx@lasthash}}
    and
    not test \iffirstonpage
    and
    (
      not bool {bbx@inset}
      or
      test {\iffieldequalstr{entrysetcount}{1}}
    )
  }{#1}{#2}%
}
%    \end{macrocode}
%
% The following is used in the `authortitle' and `authoryear' styles to
% switch off the dash check macro for reference sets.
%
%    \begin{macrocode}
\newbool{bbx@inset}
\DeclareBibliographyDriver{set}{%
  \booltrue{bbx@inset}%
  \entryset{}{}%
  \newunit\newblock
  \usebibmacro{setpageref}%
  \finentry}
%    \end{macrocode}
%
% The `author' macro is enhanced from the standard version by
% \begin{itemize}
%   \item
%   including a dash check, to see if the name(s) should be replaced
%   with a dash (as in `authortitle' and `authoryear');
%   \item
%   checking if the author name(s) should be formatted as if inferred;
%   \item
%   inserting the `nameaddon' field if provided;
%   \item
%   inserting the author type if provided (as in `authoryear').
% \end{itemize}
%
%    \begin{macrocode}
\renewbibmacro*{author}{%
  \ifboolexpr{
    test \ifuseauthor
    and
    not test {\ifnameundef{author}}
  }
  {\usebibmacro{bbx:dashcheck}
    {\bibnamedash}
    {\usebibmacro{bbx:savehash}%
      \iffieldequals{authortype}{\isInferred}
      {\mkbibbrackets{\printnames{author}}\clearfield{authortype}}%
      {\printnames{author}}%
      \iffieldundef{nameaddon}
      {}
      {\setunit{\addspace}%
        \printfield{nameaddon}}%
      \iffieldundef{authortype}
      {\setunit{\addspace}}
      {\newunit}}%
    \iffieldundef{authortype}
    {}
    {\usebibmacro{authorstrg}%
      \setunit{\addspace}}}%
  {\global\undef\bbx@lasthash}}
%    \end{macrocode}
%
% Following `authortitle' and `authoryear', we redefine the `editor' and
% `editor+others' macros to use a common `bbx:editor' macro. The macro
% we use is the same as the usual one except the editor type is set off
% using parentheses instead of a comma.
%
%    \begin{macrocode}
\renewbibmacro*{editor}{%
  \usebibmacro{bbx:editor}{editorstrg}}
\renewbibmacro*{editor+others}{%
  \usebibmacro{bbx:editor}{editor+othersstrg}}
\newbibmacro*{bbx:editor}[1]{%
  \ifboolexpr{
    test \ifuseeditor
    and
    not test {\ifnameundef{editor}}
  }
  {\usebibmacro{bbx:dashcheck}
    {\bibnamedash}
    {\printnames{editor}%
     \setunit{\addspace}%
     \usebibmacro{bbx:savehash}}%
    \usebibmacro{#1}%
    \clearname{editor}}
  {\global\undef\bbx@lasthash}}
\DeclareFieldFormat{editortype}{\mkbibparens{#1}}
%    \end{macrocode}
%
% We do the same for `translator' as well.
%
%    \begin{macrocode}
\renewbibmacro*{translator}{%
  \usebibmacro{bbx:translator}{translatorstrg}}
\renewbibmacro*{translator+others}{%
  \usebibmacro{bbx:translator}{translator+othersstrg}}
\newbibmacro*{bbx:translator}[1]{%
  \ifboolexpr{
    test \ifusetranslator
    and
    not test {\ifnameundef{translator}}
  }
  {\usebibmacro{bbx:dashcheck}
    {\bibnamedash}
    {\printnames{translator}%
      \setunit{\addspace}%
      \usebibmacro{bbx:savehash}}%
    \usebibmacro{#1}%
    \clearname{translator}%
    \setunit{\addspace}}%
  {\global\undef\bbx@lasthash}}
\xpatchbibmacro{translatorstrg}%
  {\bibstring}%
  {\bibstring[\mkbibparens]}{}{}%
\xpatchbibmacro{translator+othersstrg}%
  {\bibstring}%
  {\bibstring[\mkbibparens]}{}{}%
%    \end{macrocode}
%
% When referencing one work from a collection by the same author,
% Oxford style traditionally puts \enquote{id.} in the `bookauthor'
% position (instead of omitting it as in standard \pkg{biblatex}).
%
%    \begin{macrocode}
\renewbibmacro*{bybookauthor}{%
  \ifnamesequal{author}{bookauthor}%
  {\bibstring{idem\thefield{gender}}}%
  {\printnames{bookauthor}}}
%    \end{macrocode}
%
% Similarly, with mixed collections, if referencing a work by the editor
% of the collection, the editor name is replaced with \enquote{id.}. We
% insert this logic into a copy of the `editor+others' macro, which will
% \emph{not} get a year inserted into it by Oxyear.
%
%    \begin{macrocode}
\newbibmacro*{bookeditor}{%
  \global\undef\bbx@lasthash
  \ifboolexpr{
    test \ifuseeditor
    and
    not test {\ifnameundef{editor}}
  }{%
    \ifnamesequal{author}{editor}{%
      \bibstring{idem\thefield{gender}}%
      \addspace
      \usebibmacro{editor+othersstrg}%
      \clearname{editor}%
    }{%
      \printnames[bookeditor]{editor}%
      \newunit%
      \usebibmacro{editor+othersstrg}%
      \clearname{editor}%
    }%
  }{}}
%    \end{macrocode}
%
% \subsection{Date handling}
%
% Year ranges are truncated (e.g.~2012–3). Therefore, we provide a
% command that compares two years, and outputs a full or truncated
% version of the second year, depending on how similar it is to the
% first. It is extracted and adapted from code by Marco Daniel.%
% \footnote{\url{http://tex.stackexchange.com/questions/23483/}}
%
%    \begin{macrocode}
\newcommand{\blx@ox@compyear}[2]{%
  \def\num@one{#1}%
  \def\num@two{#2}%
  \StrLen{\num@one}[\num@one@len]%
  \StrLen{\num@two}[\num@two@len]%
  \IfEq{\num@one@len}{\num@two@len}%
  {\StrCompare{\num@one}{\num@two}[\Result]%
    \ifnum\num@two@len>3%
    \IfStrEq{\Result}{2}{\def\Result{1}}{}%
    \fi
    \StrGobbleLeft{0\num@two}{\Result}}%
  {\num@two}%
}
%    \end{macrocode}
%
% We patch the truncated date formats from `biblatex.sty' to use the
% above function.
%
%    \begin{macrocode}
\patchcmd{\mkdaterangetrunc}{%
  \csuse{mkbibdate#1}{#2endyear}{#2endmonth}{#2endday}%
}{%
  \iffieldundef{#2endmonth}%
    {\blx@ox@compyear{\thefield{#2year}}{\thefield{#2endyear}}}%
    {\csuse{mkbibdate#1}{#2endyear}{#2endmonth}{#2endday}}%
}{}{}
\patchcmd{\mkdaterangetruncextra}{%
  \csuse{mkbibdate#1}{#2endyear}{#2endmonth}{#2endday}%
}{%
  \iffieldundef{#2endmonth}%
    {\blx@ox@compyear{\thefield{#2year}}{\thefield{#2endyear}}}%
    {\csuse{mkbibdate#1}{#2endyear}{#2endmonth}{#2endday}}%
}{}{}
%    \end{macrocode}
%
% We provide a way of prefacing dates with a type. This is mainly
% intended for online resources.
%
%    \begin{macrocode}
\def\dateType{}
\DeclareEntryOption{datetype}{\def\dateType{#1}}
\renewbibmacro*{date}{%
  \ifdefvoid{\dateType}{}{\printtext{\dateType\addspace}}%
  \printdate}
%    \end{macrocode}
%
% \subsection{Other number-like fields}
%
% We let `edition' take a language key as well as a number
%
%    \begin{macrocode}
\xpatchfieldformat{edition}{%
  #1\isdot
}{%
  \ifbibstring{#1}{\bibstring{#1}}{#1\isdot}%
}{}{}
%    \end{macrocode}
%
% Page ranges are compressed, but are not usually marked with
% \enquote{pp.}. The exception is if the page numbers are not obviously
% numbers.
%
%    \begin{macrocode}
\DeclareFieldFormat{pages}{%
  \iffieldundef{bookpagination}%
    {\mkcomprange{#1}}
    {\mkcomprange[{\mkpageprefix[bookpagination]}]{#1}}%
}

%    \end{macrocode}
%
% \subsection{Publishers}
%
% The \emph{Oxford Guide to Style} says it is fine to omit publisher
% names uniformly from bibliographic information. This is odd, but we
% can support it with a simple option.
%
%    \begin{macrocode}
\DeclareBibliographyOption{nopublisher}[true]{%
  \DeclareFieldInputHandler{publisher}{\def\NewValue{}}%
}
%    \end{macrocode}
%
% \subsection{URLs}
%
% The OGS recommends the ISO convention of enclosing URLs in angle
% brackets, but NHR recommends leaving URLs bare so as not to interfere
% with text-mining. The latter is the default.
%
%    \begin{macrocode}
\DeclareBibliographyOption{isourls}[true]{%
  \ifstrequal{#1}{true}
    {\DeclareFieldFormat{url}{$\langle$\url{##1}$\rangle$}}
    {\DeclareFieldFormat{url}{\url{##1}}}%
}
\ExecuteBibliographyOptions{isourls=false}
%    \end{macrocode}
%
% URL dates are set off with a comma rather than parentheses.
%
%    \begin{macrocode}
\DeclareFieldFormat{urldate}{\bibstring{urlseen}\space#1}
\xpatchbibmacro{url+urldate}%
  {\setunit*{\addspace}}%
  {\setunit*{\addcomma\addspace}}{}{}
%    \end{macrocode}
%
% The `urldatetype' implementation below was intended to support a use
% case from the OGS, but it is not something promoted by NHR, and that
% sort of use case is probably better met with a note of some kind.
% It will be removed.
%
%    \begin{macrocode}
\DeclareEntryOption{urldatetype}{%
  \DeclareFieldFormat{urldate}{\printtext{#1}\space##1}%
  \renewbibmacro*{url+urldate}{%
    \iffieldundef{url}{}{%
      \newunit
      \printurldate%
      \setunit*{\addspace}%
      \printfield{url}%
      \nopunct
    }%
  }%
}
%    \end{macrocode}
%
% The DOI is introduced by \enquote{doi} in lowercase.
%
%    \begin{macrocode}
\xpatchfieldformat{doi}%
  {\mkbibacro{DOI}}%
  {\printtext{doi}}{}{}
%    \end{macrocode}
%
% Unlike URLs, DOIs are preceded by a full stop.
%
%    \begin{macrocode}
\xpatchbibmacro{doi+eprint+url}%
  {\printfield{doi}}%
  {\setunit{\addperiod\space}\printfield{doi}}{}{}

%    \end{macrocode}
%
% \subsection{Addenda}
%
% Publication state is given in parentheses. Other addenda are added
% plain.
%
%    \begin{macrocode}
\renewbibmacro*{addendum+pubstate}{%
  \iffieldundef{pubstate}{}{\nopunct}%
  \printfield[parens]{pubstate}%
  \setunit{\addperiod\addspace}\newblock
  \printfield{addendum}}
%    \end{macrocode}
%
% \subsection{Periodical-style entries}
%
% Subtypes for articles and similar are in square brackets.
%
%    \begin{macrocode}
\DeclareFieldFormat[article,periodical,review]{entrysubtype}{\mkbibbrackets{#1}}
%    \end{macrocode}
%
% Journal types are also in square brackets. Note that when OGS was
% published, online journals were a novelty, but since they are now
% commonplace there is no purpose to pointing them out any more, so NHR
% doesn't. This complication will be removed.
%
%    \begin{macrocode}
\newbibmacro*{journaltype}{%
  \iffieldequalstr{entrysubtype}{online journal}%
  {\nopunct\printfield{entrysubtype}}{}%
}
%    \end{macrocode}
%
% We patch the `journal+issuetitle' macro so that if a work takes up a
% whole issue (signified by using `issuetitle' \emph{instead} of
% `title'), the title and journal title are separated by \enquote{=}
% instead of the usual punctuation.
% We also insert the above macro for journal types,
% and insert a new unit after the journal title regardless of what
% follows.
%
%    \begin{macrocode}
\xpatchbibmacro{journal+issuetitle}{%
  \usebibmacro{journal}%
  \setunit*{\addspace}%
}{
  \ifboolexpr{
    test {\iffieldundef{title}}
    and
    not test {\iffieldundef{issuetitle}}
  }{%
    \usebibmacro{issue}%
    \clearfield{issuetitle}%
    \setunit{\addspace =\addspace}}{}%
  \usebibmacro{journal}%
  \usebibmacro{journaltype}%
  \newunit
}{}{}
\xpatchbibmacro*{journal+issuetitle}{%
  \newunit\printfield{series}%
}{%
  \printfield{series}%
}{}{}
%    \end{macrocode}
%
% Similarly, we patch the `title+issuetitle' macro to insert
% the `journaltype' macro and insert
% a new unit after the journal title regardless of what follows.
%
%    \begin{macrocode}
\xpatchbibmacro{title+issuetitle}{%
  \usebibmacro{periodical}%
  \setunit*{\addspace}%
}{%
  \usebibmacro{periodical}%
  \usebibmacro{journaltype}%
  \newunit
}{}{}
\xpatchbibmacro*{title+issuetitle}{%
  \newunit\printfield{series}%
}{%
  \printfield{series}%
}{}{}
%    \end{macrocode}
%
% On last inspection, \pkg{biblatex} hard-coded the punctuation for volumes
% and issue numbers into the \texttt{title+issuetitle} macro instead of
% delegating to the appropriate macro. We expect this to change in
% future but in the meantime correct this ourselves.
%
%    \begin{macrocode}
\xpatchbibmacro*{title+issuetitle}{%
  \printfield{volume}%
  \setunit*{\adddot}%
  \printfield{number}%
  \newunit%
  \printfield{eid}%
}{%
  \usebibmacro{volume+number+eid}%
}{}{}
%    \end{macrocode}
%
% OGS and NHR provide plentiful options for formatting volume and issue
% numbers. We implement four of them here as options. The default is to
% use a slash between volume and issue number. Note that we prevent the
% `volume+number+eid' macro from actually printing the eid.
%
%    \begin{macrocode}
\DeclareBibliographyOption{issuestyle}[slash]{%
  \ifcsdef{blx@ox@issuestyle@#1}{%
    \csuse{blx@ox@issuestyle@#1}%
  }{%
    \PackageError{biblatex}
      {Invalid option 'issuestyle=#1'}
      {Valid values are 'slash', 'colon', 'comma', 'parens'}%
  }%
}
\csdef{blx@ox@issuestyle@slash}{%
  \renewbibmacro*{volume+number+eid}{%
    \printfield{volume}%
    \setunit*{\addslash}%
    \printfield{number}%
  }%
}
\csdef{blx@ox@issuestyle@colon}{%
  \renewbibmacro*{volume+number+eid}{%
    \printfield{volume}%
    \setunit*{\addcolon\space}%
    \printfield{number}%
  }%
}
\csdef{blx@ox@issuestyle@comma}{%
  \renewbibmacro*{volume+number+eid}{%
    \printfield{volume}%
    \setunit*{\addcomma\space}%
    \printfield{number}%
  }%
}
\csdef{blx@ox@issuestyle@parens}{%
  \renewbibmacro*{volume+number+eid}{%
    \printfield{volume}%
    \setunit*{\addspace}%
    \printfield[parens]{number}%
  }%
}
\ExecuteBibliographyOptions{issuestyle=slash}
%    \end{macrocode}
%
% If an issue has no volume or issue numbers (first block), the season
% and date are printed bare. Otherwise (second block) they are printed
% in parentheses just as in the standard version of the macro.
%
%    \begin{macrocode}
\renewbibmacro*{issue+date}{%
  \ifboolexpr{
    test {\iffieldundef{volume}}
    and
    test {\iffieldundef{number}}
  }{%
    \newunit\printtext{%
      \iffieldundef{issue}{%
        \usebibmacro{date}
      }{%
        \printfield{issue}%
        \setunit*{\addspace}%
        \usebibmacro{date}%
      }%
    }%
  }{%
    \ifboolexpr{
      test {\iffieldundef{issue}}
      and (
        test {\iffieldundef{year}}
        and
        test {\iffieldundef{month}}
      )
    }{}{%
      \printtext[parens]{%
        \iffieldundef{issue}{%
          \usebibmacro{date}%
        }{%
          \printfield{issue}%
          \setunit*{\addspace}%
          \usebibmacro{date}%
        }%
      }%
    }%
  }%
  \newunit%
  \printfield{eid}%
}
%    \end{macrocode}
%
% We patch the `article' driver to remove the \enquote{in} macro and
% adjust punctuation after editors and page references.
%
%    \begin{macrocode}
\xpatchbibdriver{article}{%
  \usebibmacro{in:}%
  \usebibmacro{journal+issuetitle}%
}{%
  \usebibmacro{journal+issuetitle}%
}{}{}
\xpatchbibdriver{article}{%
  \usebibmacro{byeditor+others}%
  \newunit
}{%
  \usebibmacro{byeditor+others}%
  \iffieldundef{note}{\newunit}{\setunit{\addsemicolon\addspace}}
}{}{}
\xpatchbibdriver{article}{%
  \usebibmacro{pageref}%
  \newunit\newblock
}{%
  \usebibmacro{pageref}%
  \setunit{\addperiod\addspace}\newblock
}{}{}
%    \end{macrocode}
%
% \subsection{Monograph-style entries}
%
% The titles of books that have been collected into an anthology are
% treated like regular chapters and set in quotes.
%
%    \begin{macrocode}
\DeclareFieldFormat[bookinbook]{citetitle}{\mkbibquote{#1\isdot}}
\DeclareFieldFormat[bookinbook]{title}{\mkbibquote{#1\isdot}}
%    \end{macrocode}
%
% Volume numbers in monograph-style entries are formatted as roman
% numerals.
%
%    \begin{macrocode}
\DeclareFieldFormat[book,bookinbook,inbook,mvbook,%
collection,incollection,mvcollection,%
proceedings,inproceedings,mvproceedings,%
reference,inreference,mvreference]{volume}{\Rn{#1}}
%    \end{macrocode}
%
% Where a multi-volume work is more like a series, the volume number and
% main title are put in a bracketted block between the volume title
% and the usual publication block.
%
%    \begin{macrocode}
\newbibmacro*{maintitle+volume}{%
  \ifboolexpr{
    test {\iffieldundef{maintitle}}
    or
    test {\iffieldundef{volume}}
  }{}%
  {\printtext[brackets]{%
    \bibstring{volume}\addspace
    \printfield{volume}\printfield{part}\addspace
    \bibstring{ofseries}\addspace
    \usebibmacro{maintitle}}}
}
%    \end{macrocode}
%
% Oxford style signifies formal publication by putting the relevant
% details in parentheses.
%
%    \begin{macrocode}
\DeclareFieldFormat{publication}{\mkbibparens{#1}}
%    \end{macrocode}
%
% Standard \pkg{biblatex} puts a space between series name and number.
% OGS separates them with a comma.
%
%    \begin{macrocode}
\renewbibmacro*{series+number}{%
  \printfield{series}%
  \newunit
  \printfield{number}%
  \newunit}
%    \end{macrocode}
%
% When citing both the first and a later edition, the first one comes
% first, and the later one comes after a semicolon. As per standard
% \pkg{biblatex}, the elements of a single edition are separated by commas
% except that the publisher is preceded by a colon.
%
%    \begin{macrocode}
\newcounter{locpubpairs}
\newbibmacro*{edition+publisher+location+date}{%
  \printlist{origlocation}%
  \iflistundef{origpublisher}%
  {\setunit*{\addcomma\space}}%
  {\setunit*{\addcolon\space}}%
  \printlist{origpublisher}%
  \setunit*{\addcomma\space}%
  \printorigdate%
  \setunit*{\addsemicolon\space}%
  \printfield{edition}%
  \setunit*{\addcomma\space}%
%    \end{macrocode}
%
% If there are the same number of locations and publishers, and there
% are more than one pair, we print them pairwaise rather than in two
% separate lists.
%
%    \begin{macrocode}
  \ifboolexpr{%
    test {\ifnumcomp{\value{publisher}}{>}{1}}
    and
    test {\ifnumequal{\value{location}}{\value{publisher}}}
  }{%
    \setcounter{locpubpairs}{0}%
    \savebibmacro{list:andothers}%
    \renewbibmacro*{list:andothers}{}%
    \whileboolexpr{%
      test {\ifnumcomp{\value{locpubpairs}}{<}{\value{publisher}}}
    }{%
      \stepcounter{locpubpairs}%
      \ifnumcomp{\value{locpubpairs}}{>}{1}{%
        \ifnumequal{\value{publisher}}{2}{%
          \setunit*{\addspace\bibstring{and}\addspace}%
        }{%
          \ifnumequal{\value{locpubpairs}}{\value{publisher}}{%
            \setunit*{\addcomma\space\bibstring{and}\addspace}%
          }{%
            \setunit*{\addcomma\space}%
          }%
        }%
      }{}%
      \printlist[][\value{locpubpairs}-\value{locpubpairs}]{location}%
      \setunit*{\addcolon\space}%
      \printlist[][\value{locpubpairs}-\value{locpubpairs}]{publisher}%
    }%
    \restorebibmacro{list:andothers}%
  }{%
    \printlist{location}%
    \iflistundef{publisher}%
    {\setunit*{\addcomma\space}}%
    {\setunit*{\addcolon\space}}%
    \printlist{publisher}%
  }%
  \setunit*{\addcomma\space}%
  \usebibmacro{date}%
}
%    \end{macrocode}
%
% Oxford style is to provide publication details –
% series name and number, edition, publisher, location, date –
% in a parenthetical block after the title information.
%
%    \begin{macrocode}
\newbibmacro*{series+number+edition+publisher+location+date}{%
  \nopunct\printtext[publication]{%
    \usebibmacro{series+number}%
    \setunit*{\addsemicolon\addspace}%
    \usebibmacro{edition+publisher+location+date}%
  }%
}
%    \end{macrocode}
%
% The edition information for reference works is recorded earlier in the
% reference, so we provide a variant that excludes it from the
% publication block.
%
%    \begin{macrocode}
\newbibmacro*{series+number+publisher+location+date}{%
  \nopunct\printtext[publication]{%
    \usebibmacro{series+number}%
    \setunit*{\addsemicolon\addspace}%
    \usebibmacro{publisher+location+date}%
  }%
}
%    \end{macrocode}
%
% The changes to the `book' driver compared to the standard style are as
% follows:
% \begin{itemize}
%   \item
%   only `author' appears before the title, never `editor'.
%   \item
%   `maintitle' is processed with `volume' just before `series', instead
%   of with `title';
%   \item
%   `edition' is processed where the standard style processes `note',
%   and vice versa;
%   \item
%   `volumes' is omitted;
%   \item
%   `series' to `date' information is delegated to a separate macro;
%   \item
%   support is added for `howpublished' field;
%   \item
%   before `related', there is a full stop instead of a comma.
% \end{itemize}
%
%    \begin{macrocode}
\DeclareBibliographyDriver{book}{%
  \usebibmacro{bibindex}%
  \usebibmacro{begentry}%
  \usebibmacro{author}%
  \setunit{\labelnamepunct}\newblock
  \usebibmacro{title}%
  \newunit
  \printlist{language}%
  \newunit\newblock
  \usebibmacro{byauthor}%
  \newunit\newblock
  \usebibmacro{byeditor+others}%
  \newunit\newblock
  \printfield{note}%
  \newunit\newblock
  \usebibmacro{maintitle+volume}%
  \newunit
  \usebibmacro{series+number+edition+publisher+location+date}%
  \setunit{\addspace}%
  \printfield[brackets]{howpublished}%
  \newunit\newblock
  \usebibmacro{chapter+pages}%
  \newunit
  \printfield{pagetotal}%
  \newunit\newblock
  \iftoggle{bbx:isbn}
    {\printfield{isbn}}
    {}%
  \newunit\newblock
  \usebibmacro{doi+eprint+url}%
  \newunit\newblock
  \usebibmacro{addendum+pubstate}%
  \setunit{\bibpagerefpunct}\newblock
  \usebibmacro{pageref}%
  \setunit{\addperiod\addspace}\newblock
  \iftoggle{bbx:related}
    {\usebibmacro{related:init}%
     \usebibmacro{related}}
    {}%
  \usebibmacro{finentry}}
%    \end{macrocode}
%
% Unlike the standard styles, we have a separate driver for `mvbook'
% which behaves slightly differently. It is in fact closer to the
% standard `book' driver. The changes are as follows:
% \begin{itemize}
%   \item
%   only `author' appears before the title, never `editor';
%   \item
%   `volume'/`part' is processed just after `maintitle+title';
%   \item
%   `edition' is processed where the standard style processes `note';
%   \item
%   `note' is processed after `volumes';
%   \item
%   `series' to `date' information is delegated to a separate macro;
%   \item
%   before `related', there is a full stop instead of a comma.
% \end{itemize}
%
%    \begin{macrocode}
\DeclareBibliographyDriver{mvbook}{%
  \usebibmacro{bibindex}%
  \usebibmacro{begentry}%
  \usebibmacro{author}%
  \setunit{\labelnamepunct}\newblock
  \usebibmacro{maintitle+title}%
  \newunit
  \iffieldundef{maintitle}
  {\printfield{volume}%
    \printfield{part}}
  {}%
  \newunit
  \printlist{language}%
  \newunit\newblock
  \usebibmacro{byauthor}%
  \newunit\newblock
  \usebibmacro{byeditor+others}%
  \newunit\newblock
  \printfield{volumes}%
  \newunit\newblock
  \printfield{note}%
  \newunit\newblock%
  \usebibmacro{series+number+edition+publisher+location+date}%
  \newunit\newblock
  \usebibmacro{chapter+pages}%
  \newunit
  \printfield{pagetotal}%
  \newunit\newblock
  \iftoggle{bbx:isbn}
  {\printfield{isbn}}
  {}%
  \newunit\newblock
  \usebibmacro{doi+eprint+url}%
  \newunit\newblock
  \usebibmacro{addendum+pubstate}%
  \setunit{\bibpagerefpunct}\newblock
  \usebibmacro{pageref}%
  \setunit{\addperiod\addspace}\newblock
  \iftoggle{bbx:related}
  {\usebibmacro{related:init}%
    \usebibmacro{related}}
  {}%
  \usebibmacro{finentry}}
%    \end{macrocode}
%
% Our `inbook' driver modifies the standard one in just the same way as
% our `mvbook' driver modifies the standard `book'.
%
%    \begin{macrocode}
\DeclareBibliographyDriver{inbook}{%
  \usebibmacro{bibindex}%
  \usebibmacro{begentry}%
  \usebibmacro{author/translator+others}%
  \setunit{\labelnamepunct}\newblock
  \usebibmacro{title}%
  \newunit
  \printlist{language}%
  \newunit\newblock
  \usebibmacro{byauthor}%
  \newunit\newblock
  \usebibmacro{in:}%
  \usebibmacro{bybookauthor}%
  \newunit\newblock
  \usebibmacro{maintitle+booktitle}%
  \newunit
  \iffieldundef{maintitle}
  {\printfield{volume}%
    \printfield{part}}
  {}%
  \newunit\newblock
  \usebibmacro{byeditor+others}%
  \newunit\newblock
  \printfield{volumes}%
  \newunit\newblock
  \printfield{note}%
  \newunit\newblock%
  \usebibmacro{series+number+edition+publisher+location+date}%
  \newunit\newblock
  \usebibmacro{chapter+pages}%
  \newunit\newblock
  \iftoggle{bbx:isbn}
  {\printfield{isbn}}
  {}%
  \newunit\newblock
  \usebibmacro{doi+eprint+url}%
  \newunit\newblock
  \usebibmacro{addendum+pubstate}%
  \setunit{\bibpagerefpunct}\newblock
  \usebibmacro{pageref}%
  \setunit{\addperiod\addspace}\newblock
  \iftoggle{bbx:related}
  {\usebibmacro{related:init}%
    \usebibmacro{related}}
  {}%
  \usebibmacro{finentry}}
%    \end{macrocode}
%
% Our `collection' driver modifies the standard one in just the same way
% as our `book' driver.
%
%    \begin{macrocode}
\DeclareBibliographyDriver{collection}{%
  \usebibmacro{bibindex}%
  \usebibmacro{begentry}%
  \usebibmacro{editor+others}%
  \setunit{\labelnamepunct}\newblock
  \usebibmacro{title}%
  \newunit
  \printlist{language}%
  \newunit\newblock
  \usebibmacro{byeditor+others}%
  \newunit\newblock
  \printfield{note}%
  \newunit\newblock
  \usebibmacro{maintitle+volume}%
  \newunit
  \usebibmacro{series+number+edition+publisher+location+date}%
  \newunit\newblock
  \usebibmacro{chapter+pages}%
  \newunit
  \printfield{pagetotal}%
  \newunit\newblock
  \iftoggle{bbx:isbn}
  {\printfield{isbn}}
  {}%
  \newunit\newblock
  \usebibmacro{doi+eprint+url}%
  \newunit\newblock
  \usebibmacro{addendum+pubstate}%
  \setunit{\bibpagerefpunct}\newblock
  \usebibmacro{pageref}%
  \setunit{\addperiod\addspace}\newblock
  \iftoggle{bbx:related}
  {\usebibmacro{related:init}%
    \usebibmacro{related}}
  {}%
  \usebibmacro{finentry}}
%    \end{macrocode}
%
% Our `mvcollection' driver modifies the standard `collection' driver
% in just the same way as our `mvbook' driver modifies the standard
% `book'.
%
%    \begin{macrocode}
\DeclareBibliographyDriver{mvcollection}{%
  \usebibmacro{bibindex}%
  \usebibmacro{begentry}%
  \usebibmacro{editor+others}%
  \setunit{\labelnamepunct}\newblock
  \usebibmacro{maintitle+title}%
  \newunit
  \iffieldundef{maintitle}
  {\printfield{volume}%
    \printfield{part}}
  {}%
  \newunit
  \printlist{language}%
  \newunit\newblock
  \usebibmacro{byeditor+others}%
  \newunit\newblock
  \printfield{volumes}%
  \newunit\newblock
  \printfield{note}%
  \newunit\newblock%
  \usebibmacro{series+number+edition+publisher+location+date}%
  \newunit\newblock
  \usebibmacro{chapter+pages}%
  \newunit
  \printfield{pagetotal}%
  \newunit\newblock
  \iftoggle{bbx:isbn}
  {\printfield{isbn}}
  {}%
  \newunit\newblock
  \usebibmacro{doi+eprint+url}%
  \newunit\newblock
  \usebibmacro{addendum+pubstate}%
  \setunit{\bibpagerefpunct}\newblock
  \usebibmacro{pageref}%
  \setunit{\addperiod\addspace}\newblock
  \iftoggle{bbx:related}
  {\usebibmacro{related:init}%
    \usebibmacro{related}}
  {}%
  \usebibmacro{finentry}}
%    \end{macrocode}
%
% Our `incollection' driver modifies the standard one in the same way as
% our `mvbook' driver modifies the standard `book'. In addition, the
% `bookeditor' macro precedes `maintitle' instead of `byeditor+others'
% following it.
%
%    \begin{macrocode}
\DeclareBibliographyDriver{incollection}{%
  \usebibmacro{bibindex}%
  \usebibmacro{begentry}%
  \usebibmacro{author/translator+others}%
  \setunit{\labelnamepunct}\newblock
  \usebibmacro{title}%
  \newunit
  \printlist{language}%
  \newunit\newblock
  \usebibmacro{byauthor}%
  \newunit\newblock
  \usebibmacro{in:}%
  \usebibmacro{bookeditor}%
  \newunit\newblock
  \usebibmacro{maintitle+booktitle}%
  \newunit
  \iffieldundef{maintitle}
  {\printfield{volume}%
    \printfield{part}}
  {}%
  \newunit\newblock
  \printfield{volumes}%
  \newunit\newblock
  \printfield{note}%
  \newunit\newblock%
  \usebibmacro{series+number+edition+publisher+location+date}%
  \newunit\newblock
  \usebibmacro{chapter+pages}%
  \newunit\newblock
  \iftoggle{bbx:isbn}
  {\printfield{isbn}}
  {}%
  \newunit\newblock
  \usebibmacro{doi+eprint+url}%
  \newunit\newblock
  \usebibmacro{addendum+pubstate}%
  \setunit{\bibpagerefpunct}\newblock
  \usebibmacro{pageref}%
  \setunit{\addperiod\addspace}\newblock
  \iftoggle{bbx:related}
  {\usebibmacro{related:init}%
    \usebibmacro{related}}
  {}%
  \usebibmacro{finentry}}
%    \end{macrocode}
%
% Unlike the standard styles, we provide a separate driver for
% `reference'. It is just like `book' except that `edition' is given
% between `byauthor' and `byeditor+others'.
%
%    \begin{macrocode}
\DeclareBibliographyDriver{reference}{%
  \usebibmacro{bibindex}%
  \usebibmacro{begentry}%
  \usebibmacro{author}%
  \setunit{\labelnamepunct}\newblock
  \usebibmacro{title}%
  \newunit
  \printlist{language}%
  \newunit\newblock
  \usebibmacro{byauthor}%
  \newunit\newblock
  \printfield{edition}%
  \newunit
  \usebibmacro{byeditor+others}%
  \newunit\newblock
  \printfield{note}%
  \newunit\newblock
  \usebibmacro{maintitle+volume}%
  \newunit
  \usebibmacro{series+number+publisher+location+date}%
  \newunit\newblock
  \usebibmacro{chapter+pages}%
  \newunit
  \printfield{pagetotal}%
  \newunit\newblock
  \iftoggle{bbx:isbn}
  {\printfield{isbn}}
  {}%
  \newunit\newblock
  \usebibmacro{doi+eprint+url}%
  \newunit\newblock
  \usebibmacro{addendum+pubstate}%
  \setunit{\bibpagerefpunct}\newblock
  \usebibmacro{pageref}%
  \setunit{\addperiod\addspace}\newblock
  \iftoggle{bbx:related}
  {\usebibmacro{related:init}%
    \usebibmacro{related}}
  {}%
  \usebibmacro{finentry}}
%    \end{macrocode}
%
% Similarly, `mvreference' is just like `mvbook' except that `edition'
% is given between `byauthor' and `byeditor+others'.
%
%    \begin{macrocode}
\DeclareBibliographyDriver{mvreference}{%
  \usebibmacro{bibindex}%
  \usebibmacro{begentry}%
  \usebibmacro{author}%
  \setunit{\labelnamepunct}\newblock
  \usebibmacro{maintitle+title}%
  \newunit
  \iffieldundef{maintitle}
  {\printfield{volume}%
    \printfield{part}}
  {}%
  \newunit
  \printlist{language}%
  \newunit\newblock
  \usebibmacro{byauthor}%
  \newunit\newblock
  \printfield{edition}%
  \newunit
  \usebibmacro{byeditor+others}%
  \newunit\newblock
  \printfield{volumes}%
  \newunit\newblock
  \printfield{note}%
  \newunit\newblock
  \usebibmacro{series+number+publisher+location+date}%
  \newunit\newblock
  \usebibmacro{chapter+pages}%
  \newunit
  \printfield{pagetotal}%
  \newunit\newblock
  \iftoggle{bbx:isbn}
  {\printfield{isbn}}
  {}%
  \newunit\newblock
  \usebibmacro{doi+eprint+url}%
  \newunit\newblock
  \usebibmacro{addendum+pubstate}%
  \setunit{\bibpagerefpunct}\newblock
  \usebibmacro{pageref}%
  \setunit{\addperiod\addspace}\newblock
  \iftoggle{bbx:related}
  {\usebibmacro{related:init}%
    \usebibmacro{related}}
  {}%
  \usebibmacro{finentry}}
%    \end{macrocode}
%
% Our `inreference' driver is just like `inbook' except `bookeditor'
% replaces `bybookauthor' and `edition' is given just before
% `byeditor+others'.
%
%    \begin{macrocode}
\DeclareBibliographyDriver{inreference}{%
  \usebibmacro{bibindex}%
  \usebibmacro{begentry}%
  \usebibmacro{author/translator+others}%
  \setunit{\labelnamepunct}\newblock
  \usebibmacro{title}%
  \newunit
  \printlist{language}%
  \newunit\newblock
  \usebibmacro{byauthor}%
  \newunit\newblock
  \usebibmacro{in:}%
  \usebibmacro{bookeditor}%
  \newunit\newblock
  \usebibmacro{maintitle+booktitle}%
  \newunit
  \iffieldundef{maintitle}
  {\printfield{volume}%
    \printfield{part}}
  {}%
  \newunit\newblock
  \printfield{edition}%
  \newunit
  \usebibmacro{byeditor+others}%
  \newunit\newblock
  \printfield{volumes}%
  \newunit\newblock
  \printfield{note}%
  \newunit\newblock%
  \usebibmacro{series+number+publisher+location+date}%
  \newunit\newblock
  \usebibmacro{chapter+pages}%
  \newunit
  \printfield{pagetotal}%
  \newunit\newblock
  \iftoggle{bbx:isbn}
  {\printfield{isbn}}
  {}%
  \newunit\newblock
  \usebibmacro{doi+eprint+url}%
  \newunit\newblock
  \usebibmacro{addendum+pubstate}%
  \setunit{\bibpagerefpunct}\newblock
  \usebibmacro{pageref}%
  \setunit{\addperiod\addspace}\newblock
  \iftoggle{bbx:related}
  {\usebibmacro{related:init}%
    \usebibmacro{related}}
  {}%
  \usebibmacro{finentry}}
%    \end{macrocode}
%
% \subsection{Proceedings}
%
% In our version of the `event+venue+date' macro, the venue and date
% are set off with a colon and comma respectively instead of
% parentheses. A bit of jiggery-pokery is used to make the colon italic
% if it follows italic text.
%
%    \begin{macrocode}
\renewbibmacro*{event+venue+date}{%
  \printfield{eventtitle}%
  \newunit
  \printfield{eventtitleaddon}%
  \ifboolexpr{
    test {\iffieldundef{venue}}
    and
    test {\iffieldundef{eventyear}}
  }
  {}
  {\ifboolexpr{
      test {\iffieldundef{language}}
      and
      test {\iffieldundef{eventtitle}}
    }{\setunit{\mkbibemph{\addcolon}\addspace}}%
    {\setunit{\addcolon\addspace}}%
    \printfield{venue}%
    \newunit%
    \printeventdate}%
  \newunit}
%    \end{macrocode}
%
% The changes to the `proceedings' driver compared to the standard style
% are as follows:
% \begin{itemize}
%   \item
%   `organization' is moved before the title;
%   \item
%   `editor' comes after the title, not before;
%   \item
%   `volume'/`part' is processed just after `maintitle+title';
%   \item
%   `note' is moved to before `series';
%   \item
%   `series' to `date' information is delegated to a separate macro;
%   \item
%   before `related', there is a full stop instead of a comma.
% \end{itemize}
%
%    \begin{macrocode}
\DeclareBibliographyDriver{proceedings}{%
  \usebibmacro{bibindex}%
  \usebibmacro{begentry}%
  \printlist{organization}%
  \newunit
  \usebibmacro{maintitle+title}%
  \newunit
  \iffieldundef{maintitle}
  {\printfield{volume}%
    \printfield{part}}
  {}%
  \newunit
  \printlist{language}%
  \newunit\newblock
  \usebibmacro{event+venue+date}%
  \newunit\newblock
  \usebibmacro{byeditor+others}%
  \newunit\newblock
  \printfield{volumes}%
  \newunit\newblock
  \printfield{note}%
  \newunit\newblock
  \usebibmacro{series+number+edition+publisher+location+date}%
  \newunit\newblock
  \usebibmacro{chapter+pages}%
  \newunit
  \printfield{pagetotal}%
  \newunit\newblock
  \iftoggle{bbx:isbn}
  {\printfield{isbn}}
  {}%
  \newunit\newblock
  \usebibmacro{doi+eprint+url}%
  \newunit\newblock
  \usebibmacro{addendum+pubstate}%
  \setunit{\bibpagerefpunct}\newblock
  \usebibmacro{pageref}%
  \setunit{\addperiod\addspace}\newblock
  \iftoggle{bbx:related}
  {\usebibmacro{related:init}%
    \usebibmacro{related}}
  {}%
  \usebibmacro{finentry}}
%    \end{macrocode}
%
% Our `inproceedings' driver modifies the standard one in the same way
% as our `proceedings' driver modifies the standard one.
%
%    \begin{macrocode}
\DeclareBibliographyDriver{inproceedings}{%
  \usebibmacro{bibindex}%
  \usebibmacro{begentry}%
  \usebibmacro{author/translator+others}%
  \setunit{\labelnamepunct}\newblock
  \usebibmacro{title}%
  \newunit
  \printlist{language}%
  \newunit\newblock
  \usebibmacro{byauthor}%
  \newunit\newblock
  \usebibmacro{in:}%
  \usebibmacro{maintitle+booktitle}%
  \newunit
  \iffieldundef{maintitle}
  {\printfield{volume}%
    \printfield{part}}
  {}%
  \newunit\newblock
  \usebibmacro{event+venue+date}%
  \newunit\newblock
  \usebibmacro{byeditor+others}%
  \newunit\newblock
  \printfield{volumes}%
  \newunit\newblock
  \printfield{note}%
  \newunit\newblock
  \printlist{organization}%
  \newunit
  \usebibmacro{series+number+publisher+location+date}%
  \newunit\newblock
  \usebibmacro{chapter+pages}%
  \newunit\newblock
  \iftoggle{bbx:isbn}
  {\printfield{isbn}}
  {}%
  \newunit\newblock
  \usebibmacro{doi+eprint+url}%
  \newunit\newblock
  \usebibmacro{addendum+pubstate}%
  \setunit{\bibpagerefpunct}\newblock
  \usebibmacro{pageref}%
  \setunit{\addperiod\addspace}\newblock
  \iftoggle{bbx:related}
  {\usebibmacro{related:init}%
    \usebibmacro{related}}
  {}%
  \usebibmacro{finentry}}
%    \end{macrocode}
%
% \subsection{Grey literature}
%
% We provide a macro for handling `type+number'. We provide some extra
% logic to insert `No.' before the number if there is no type.
%
%    \begin{macrocode}
\newbibmacro*{series+type+number}{%
  \printfield{series}%
  \newunit
  \ifboolexpr{%
      test {\iffieldundef{type}}
      and
      not test {\iffieldundef{number}}
  }{%
    \bibcpstring{number}
  }{%
    \printfield{type}%
  }%
  \setunit*{\addspace}%
  \printfield{number}}
%    \end{macrocode}
%
% Our version of the `institution+location+date' macro is just like the
% standard one except it is wrapped in parentheses, and the preceding
% punctuation is therefore suppressed. This is used by the standard
% `thesis' driver, which also works well for us, and the `report' driver.
%
%    \begin{macrocode}
\renewbibmacro*{institution+location+date}{%
  \nopunct\printtext[publication]{%
    \printlist{location}%
    \iflistundef{institution}%
    {\setunit*{\addcomma\space}}%
    {\setunit*{\addcolon\space}}%
    \printlist{institution}%
    \setunit*{\addcomma\space}%
    \usebibmacro{date}%
  }%
}
%    \end{macrocode}
%
% We provide a slightly more complicated version with the type included
% at the beginning. It is used by the `thesis' driver. We provide two
% versions: one with the type outside and one with it inside the
% parentheses.
%
%    \begin{macrocode}
\newbibmacro*{type+institution+location+date}{%
  \printfield{type}%
  \newunit
  \usebibmacro{institution+location+date}%
}
\DeclareBibliographyOption{altthesis}[true]{%
  \renewbibmacro*{type+institution+location+date}{%
    \nopunct\printtext[publication]{%
      \printfield{type}%
      \setunit*{\addcomma\space}%
      \printlist{location}%
      \iflistundef{institution}%
      {\setunit*{\addcomma\space}}%
      {\setunit*{\addcolon\space}}%
      \printlist{institution}%
      \setunit*{\addcomma\space}%
      \usebibmacro{date}%
    }%
  }%
}
%    \end{macrocode}
%
% We provide an even more convoluted version that also includes `series',
% `title' and `number', and uses the more common `publisher' in place of
% `institution'. We also include some logic that means `organization'
% is used instead of `publisher' if no `publisher' is provided;
% otherwise it is printed before `location'. This is used by the
% `manual' driver.
%
%    \begin{macrocode}
\newbibmacro*{type+series+number+edition+organization+publisher+location+date}{%
  \nopunct\printtext[publication]{%
    \usebibmacro{series+type+number}%
    \setunit{\addsemicolon\space}%
    \printfield{edition}%
    \setunit*{\addcomma\space}%
    \iflistundef{publisher}%
    {}%
    {\printlist{organization}%
      \setunit*{\addcomma\space}}%
    \printlist{location}%
    \iflistundef{publisher}%
    {\iflistundef{organization}%
      {\setunit*{\addcomma\space}}%
      {\setunit*{\addcolon\space}%
        \printlist{organization}}}%
    {\setunit*{\addcolon\space}
      \printlist{publisher}}%
    \setunit*{\addcomma\space}%
    \usebibmacro{date}%
  }%
}
%    \end{macrocode}
%
% The changes to the `report' driver compared to the standard style
% are as follows:
% \begin{itemize}
%   \item
%   support is added for `volume', `maintitle' and `series';
%   \item
%   `type' and `number' are moved to just before the publication
%   information block';
%   \item
%   before `related', there is a full stop instead of a comma.
% \end{itemize}
%
%    \begin{macrocode}
\DeclareBibliographyDriver{report}{%
  \usebibmacro{bibindex}%
  \usebibmacro{begentry}%
  \usebibmacro{author}%
  \setunit{\labelnamepunct}\newblock
  \usebibmacro{maintitle+title}%
  \newunit
  \iffieldundef{maintitle}
  {\printfield{volume}%
    \printfield{part}}
  {}%
  \newunit
  \printlist{language}%
  \newunit\newblock
  \usebibmacro{byauthor}%
  \newunit\newblock
  \printfield{version}%
  \newunit\newblock
  \printfield{note}%
  \newunit\newblock
  \usebibmacro{series+type+number}%
  \setunit{\addspace}%
  \usebibmacro{institution+location+date}%
  \newunit\newblock
  \usebibmacro{chapter+pages}%
  \newunit
  \printfield{pagetotal}%
  \newunit\newblock
  \iftoggle{bbx:isbn}
  {\printfield{isrn}}
  {}%
  \newunit\newblock
  \usebibmacro{doi+eprint+url}%
  \newunit\newblock
  \usebibmacro{addendum+pubstate}%
  \setunit{\bibpagerefpunct}\newblock
  \usebibmacro{pageref}%
  \setunit{\addperiod\addspace}\newblock
  \iftoggle{bbx:related}
  {\usebibmacro{related:init}%
    \usebibmacro{related}}
  {}%
  \usebibmacro{finentry}}
%    \end{macrocode}
%
% We patch the thesis driver to use our slightly more complex version.
%
%    \begin{macrocode}
\xpatchbibdriver{thesis}{%
  \printfield{type}%
  \newunit
  \usebibmacro{institution+location+date}%
}{%
  \usebibmacro{type+institution+location+date}%
}{}{}
%    \end{macrocode}
%
% The changes to the `manual' driver compared to the standard style
% are as follows:
% \begin{itemize}
%   \item
%   `type', `series', `number', `edition' and `organization' are moved
%   into the publication information block with `publisher', `location'
%   and `date';
%   \item
%   before `related', there is a full stop instead of a comma.
% \end{itemize}
%
%    \begin{macrocode}
\DeclareBibliographyDriver{manual}{%
  \usebibmacro{bibindex}%
  \usebibmacro{begentry}%
  \usebibmacro{author/editor}%
  \setunit{\labelnamepunct}\newblock
  \usebibmacro{title}%
  \newunit
  \printlist{language}%
  \newunit\newblock
  \usebibmacro{byauthor}%
  \newunit\newblock
  \usebibmacro{byeditor}%
  \newunit\newblock
  \printfield{type}%
  \newunit
  \printfield{version}%
  \newunit
  \printfield{note}%
  \newunit\newblock
  \usebibmacro{type+series+number+edition+organization+publisher+location+date}%
  \newunit\newblock
  \usebibmacro{chapter+pages}%
  \newunit
  \printfield{pagetotal}%
  \newunit\newblock
  \iftoggle{bbx:isbn}
  {\printfield{isbn}}
  {}%
  \newunit\newblock
  \usebibmacro{doi+eprint+url}%
  \newunit\newblock
  \usebibmacro{addendum+pubstate}%
  \setunit{\bibpagerefpunct}\newblock
  \usebibmacro{pageref}%
  \setunit{\addperiod\addspace}\newblock
  \iftoggle{bbx:related}
  {\usebibmacro{related:init}%
    \usebibmacro{related}}
  {}%
  \usebibmacro{finentry}}
%    \end{macrocode}
%
% \subsection{Intrinsically online resources}
%
% The titles of web pages and similar intrinsically online resources
% are written in roman text within quotes.
%
%    \begin{macrocode}
\DeclareFieldFormat[online]{title}{\mkbibquote{#1\isdot}}
\DeclareFieldFormat[online]{citetitle}{\mkbibquote{#1\isdot}}
%    \end{macrocode}
%
% The `entrysubtype' field is wrapped in square brackets.
%
%    \begin{macrocode}
\DeclareFieldFormat[online]{entrysubtype}{\mkbibbrackets{#1}}
%    \end{macrocode}
%
% The changes to the `online' driver compared to the standard style
% are as follows:
% \begin{itemize}
%   \item
%   We support the use of `maintitle' for, say, the title of the wesbite
%   in which the web page is located. This comes after `title'.
%   \item
%   We support the use of `entrysubtype' to specify the nature of the
%   online resource (following the practice of 'biblatex-apa' and
%   others). This comes after `maintitle'.
%   \item
%   In most cases, the date is wrapped in parentheses if present.
%   With Twitter posts, the time and date are printed bare.
%   \item
%   We support displaying a publisher after the date.
%   \item
%   before `related', there is a full stop instead of a comma.
% \end{itemize}
%
%    \begin{macrocode}
\DeclareBibliographyDriver{online}{%
  \usebibmacro{bibindex}%
  \usebibmacro{begentry}%
  \usebibmacro{author/editor+others/translator+others}%
  \setunit{\labelnamepunct}\newblock
  \usebibmacro{title}%
  \newunit
  \usebibmacro{maintitle}
  \setunit{\addspace}\newblock
  \printfield{entrysubtype}%
  \newunit
  \printlist{language}%
  \newunit\newblock
  \usebibmacro{byauthor}%
  \newunit\newblock
  \usebibmacro{byeditor+others}%
  \newunit\newblock
  \printfield{version}%
  \newunit
  \printfield{note}%
  \newunit\newblock
  \printlist{organization}%
  \iffieldequalstr{entrysubtype}{Twitter post}{%
    \newunit\newblock
    \usebibmacro{time+date}%
  }{%
    \iffieldundef{year}{}{%
      \setunit{\addspace}\newblock
      \printtext[parens]{\usebibmacro{date}}%
    }%
  }
  \newunit\newblock
  \printlist{publisher}%
  \newunit\newblock
  \iftoggle{bbx:eprint}
  {\usebibmacro{eprint}}
  {}%
  \newunit\newblock
  \usebibmacro{url+urldate}%
  \newunit\newblock
  \usebibmacro{addendum+pubstate}%
  \setunit{\bibpagerefpunct}\newblock
  \usebibmacro{pageref}%
  \setunit{\addperiod\addspace}\newblock
  \iftoggle{bbx:related}
  {\usebibmacro{related:init}%
    \usebibmacro{related}}
  {}%
  \usebibmacro{finentry}}
%    \end{macrocode}
%
% We define a new macro for printing both time and date.
%
%    \begin{macrocode}
\newbibmacro*{time+date}{%
  \printtime
  \newunit
  \printdate
}
%    \end{macrocode}
%
% \subsection{Audiovisual resources}
%
% The publication block for audiovisual resources is quite different
% from the normal, as the `type', `series' and `number' come
% between the `publisher' and `date'. We do not use `location'.
% Specification of the medium goes at the start, which we implement
% using the `howpublished' field.
%
%    \begin{macrocode}
\newbibmacro*{publisher+type+series+number+date}{%
  \nopunct\printtext[publication]{%
    \printfield{howpublished}%
    \setunit*{\addcomma\space}%
    \printlist{publisher}%
    \setunit*{\addspace}%
    \printfield{type}%
    \setunit*{\addspace}%
    \printfield{series}%
    \setunit*{\addcomma\space}%
    \printfield{number}%
    \newunit%
    \usebibmacro{date}%
  }%
}
%    \end{macrocode}
%
% We provide a specialist `audio'/`music' driver. It is currently based
% closely on `book'; it could do with a more thoughtful implementation.
%
%    \begin{macrocode}
\DeclareBibliographyDriver{audio}{%
  \usebibmacro{bibindex}%
  \usebibmacro{begentry}%
  \usebibmacro{author}%
  \setunit{\labelnamepunct}\newblock
  \usebibmacro{maintitle+title}%
  \newunit
  \iffieldundef{maintitle}
  {\printfield{volume}%
    \printfield{part}}
  {}%
  \newunit
  \printlist{language}%
  \newunit\newblock
  \usebibmacro{byauthor}%
  \newunit\newblock
  \usebibmacro{byeditor+others}%
  \newunit\newblock
  \printfield{volumes}%
  \newunit\newblock
  \printfield{note}%
  \newunit\newblock%
  \usebibmacro{publisher+type+series+number+date}%
  \newunit\newblock
  \usebibmacro{chapter+pages}%
  \newunit
  \printfield{pagetotal}%
  \newunit\newblock
  \iftoggle{bbx:isbn}
  {\printfield{isbn}}
  {}%
  \newunit\newblock
  \usebibmacro{doi+eprint+url}%
  \newunit\newblock
  \usebibmacro{addendum+pubstate}%
  \setunit{\bibpagerefpunct}\newblock
  \usebibmacro{pageref}%
  \setunit{\addperiod\addspace}\newblock
  \iftoggle{bbx:related}
  {\usebibmacro{related:init}%
    \usebibmacro{related}}
  {}%
  \usebibmacro{finentry}}

\DeclareBibliographyAlias{music}{audio}
%    \end{macrocode}
%
% \subsection{Related entries}
%
% \subsubsection{Reprints}
%
% We take advantage of the Biber-specific mechanism of relating items
% in order to handle reprints. The presence or otherwise of the
% `journaltitle' field is used to determine whether this is a reprint
% from a book (first block) or a periodical (second block).
%
%    \begin{macrocode}
\renewbibmacro*{related:reprintfrom}[1]{%
  \entrydata*{#1}{\unspace
    \iffieldundef{journaltitle}{%
      \iffieldundef{maintitle}
      {\printfield[title:hook]{booktitle}}
      {\printfield[title:hook]{maintitle}}%
      \newunit\newblock
      \usebibmacro{byeditor+others}%
      \newunit\newblock
      \iffieldundef{volume}{%
      }{%
        \printfield{volume}%
        \printfield{part}%
      }%
      \newunit\newblock
      \printfield{note}%
      \newunit\newblock
      \usebibmacro{series+number+edition+publisher+location+date}%
      \newunit\newblock
      \usebibmacro{chapter+pages}%
    }{%
      \usebibmacro{journal+issuetitle}%
      \newunit
      \usebibmacro{byeditor+others}%
      \iffieldundef{note}{%
        \newunit
      }{%
        \setunit{\addsemicolon\addspace}%
      }%
      \usebibmacro{note+pages}%
    }}}
%    \end{macrocode}
%
% \subsubsection{Reviews}
%
% We ensure that the field formatting for `review' entries matches that
% for `article' entries.
%
%    \begin{macrocode}
\DeclareFieldFormat[review]{title}{\mkbibquote{#1\isdot}}
\DeclareFieldFormat[review]{volume}{#1}% volume of a journal
\DeclareFieldFormat[review]{number}{#1}% number of a journal
\DeclareFieldFormat[review]{series}{% series of a journal
  \ifinteger{#1}
  {\mkbibordseries{#1}~\bibstring{jourser}}
  {\ifbibstring{#1}{\bibstring{#1}}{#1}}}
%    \end{macrocode}
%
% We provide some alternative macros for related authors
%
%    \begin{macrocode}
\DeclareNameAlias{related}{given-family}
\newbibmacro*{relatedauthor}{%
  \ifboolexpr{
    test \ifuseauthor
    and
    not test {\ifnameundef{author}}
  }
    {\printnames[related]{author}%
     \iffieldundef{authortype}
       {}
       {\newunit%
        \usebibmacro{authorstrg}}}
    {}}
\newbibmacro*{relatedauthor/translator+others}{%
  \ifboolexpr{
    test \ifuseauthor
    and
    not test {\ifnameundef{author}}
  }
    {\usebibmacro{relatedauthor}}
    {\usebibmacro{translator+others}}}
%    \end{macrocode}
%
% This macro provides a non-Biber method of inserting a formatted
% reference within a review reference. It is deprecated and will
% be removed in a future release.
%
%    \begin{macrocode}
\newbibmacro*{reviewof}{}
\DeclareBibliographyCategory{reviewed}
\DeclareEntryOption{reviewed}{%
  \renewbibmacro*{reviewof}{%
    \bibstring{reviewof}%
    \setunit{\addspace}%
    \printtext{{\settoggle{blx@bibliography}{false}\fullcite{#1}}}%
  }%
  \addtocategory{reviewed}{#1}%
}
%    \end{macrocode}
%
% We take advantage of the Biber-specific mechanism of relating items
% in order to handle reviews. The presence or otherwise of the
% `journaltitle' field is used to determine whether this is a review
% of a book (first block) or a periodical (second block).
%
%    \begin{macrocode}
\newbibmacro*{related:reviewof}[1]{%
  \entrydata*{#1}{%
    \iffieldundef{journaltitle}{%
      \usebibmacro{relatedauthor}%
      \setunit{\labelnamepunct}\newblock
      \usebibmacro{maintitle+title}%
      \newunit
      \iffieldundef{maintitle}{%
        \printfield{volume}%
        \printfield{part}%
      }{}%
      \newunit
      \printlist{language}%
      \newunit\newblock
      \usebibmacro{byauthor}%
      \newunit\newblock
      \usebibmacro{byeditor+others}%
      \newunit\newblock
      \printfield{volumes}%
      \newunit\newblock
      \printfield{note}%
      \newunit\newblock
      \usebibmacro{series+number+edition+publisher+location+date}%
    }{%
      \usebibmacro{relatedauthor/translator+others}%
      \setunit{\labelnamepunct}\newblock
      \usebibmacro{title}%
      \newunit
      \printlist{language}%
      \newunit\newblock
      \usebibmacro{byauthor}%
      \newunit\newblock
      \usebibmacro{bytranslator+others}%
      \newunit\newblock
      \printfield{version}%
      \newunit\newblock
      \usebibmacro{journal+issuetitle}%
      \newunit
      \usebibmacro{byeditor+others}%
      \iffieldundef{note}%
        {\newunit}%
        {\setunit{\addsemicolon\addspace}}%
      \usebibmacro{note+pages}%
    }%
  }%
}
%    \end{macrocode}
%
% Since a review is a special kind of journal article, the `review'
% driver is based on the `article' one, the difference being that the
% `related' macro comes just before `journal+issuetitle'. Oxford style
% links the two with \enquote{in}.
%
%    \begin{macrocode}
\DeclareBibliographyDriver{review}{%
  \usebibmacro{bibindex}%
  \usebibmacro{begentry}%
  \usebibmacro{author/translator+others}%
  \setunit{\labelnamepunct}\newblock
  \usebibmacro{title}%
  \newunit
  \printlist{language}%
  \newunit\newblock
  \usebibmacro{byauthor}%
  \newunit\newblock
  \usebibmacro{bytranslator+others}%
  \newunit\newblock
  \printfield{version}%
  \newunit\newblock
  \usebibmacro{reviewof}%
  \newunit\newblock
  \iftoggle{bbx:related}{%
    \usebibmacro{related:init}%
    \usebibmacro{related}%
  }{}%
  \newunit\newblock
  \usebibmacro{in:}%
  \unspace
  \usebibmacro{journal+issuetitle}%
  \newunit
  \usebibmacro{byeditor+others}%
  \iffieldundef{note}%
    {\newunit}%
    {\setunit{\addsemicolon\addspace}}%
  \usebibmacro{note+pages}%
  \newunit\newblock
  \iftoggle{bbx:isbn}
    {\printfield{issn}}
    {}%
  \newunit\newblock
  \usebibmacro{doi+eprint+url}%
  \newunit\newblock
  \usebibmacro{addendum+pubstate}%
  \setunit{\bibpagerefpunct}\newblock
  \usebibmacro{pageref}%
  \usebibmacro{finentry}}
%    \end{macrocode}
%
% \subsection{Sort order}
%
% We exclude editor and translator from the sort order for books,
% reference works, and proceedings, music, since they come after the
% title.
%
%    \begin{macrocode}
\DeclareSortExclusion{book,reference,mvreference,proceedings,music}{editor,translator}
%    \end{macrocode}
%
% \iffalse
%</bbx-0>
%<*bbx-n>
% \fi
%
% \section{Notes style: oxnotes.bbx}
%
% \setcounter{FancyVerbLine}{19}
%
% \subsection{Preliminaries}
%
% First we load the common oxref features.
%
%    \begin{macrocode}
\RequireBibliographyStyle{oxref}
%    \end{macrocode}
%
% We apply our special name format to names likely to appear at the start of
% an entry.
%
%    \begin{macrocode}
\DeclareNameAlias{author}{bib-family-given/cite-given-family}
\DeclareNameAlias[related:reviewof]{author}{given-family}
\DeclareNameAlias[book,booklet,collection,manual,misc,mvbook,mvcollection,%
  mvproceedings,mvreference,online,proceedings,reference]%
  {editor}{bib-family-given/cite-given-family}
%    \end{macrocode}
%
% \subsection{Source mappings}
%
% This is used by Biber to fix the following sorting issues:
% \begin{itemize}
%   \item
%   `proceedings' entries print the organization at the beginning.
%   \item
%   `proceedings' entries print the maintitle before the title.
% \end{itemize}
%
%    \begin{macrocode}
\DeclareStyleSourcemap{%
  \maps[datatype=bibtex]{%
    \map{%
      \pertype{proceedings}
      \step[fieldsource=organization]%
      \step[fieldset=sortname, origfieldval]%
      \step[fieldsource=maintitle]%
      \step[fieldset=sorttitle, origfieldval]%
      \step[fieldsource=volume]%
      \step[fieldset=sorttitle, append, origfieldval]%
    }%
  }%
}
%    \end{macrocode}
%
% \subsection{Page ranges in citations}
%
% If specifying a page within a range in the course of a citation note,
% Oxford style is to format that as \enquote{\emph{range} at
% \emph{page}}.
%
%    \begin{macrocode}
\DeclareFieldFormat{postnote}{%
  \iffieldundef{pages}%
    {}%
    {\addspace\bibstring{at}\addspace}%
  \mkcomprange{#1}}
%    \end{macrocode}
%
% The idea here is that if a postnote is provided but the entry does not
% include a page range, we print the normal delimiter and the postnote;
% If there is a page range, we replace the intervening punctuation with
% \enquote{at}.
%
%    \begin{macrocode}
\renewbibmacro*{postnote}{%
  \iffieldundef{postnote}
  {}
  {\iffieldundef{pages}
    {\printtext{\unspace}\setunit{\postnotedelim}\newblock\printfield{postnote}}
    {\printfield{postnote}}%
  }}
%    \end{macrocode}
%
% \iffalse
%</bbx-n>
%<*bbx-y>
% \fi
%
% \section{Author–year style: oxyear.bbx}
%
% \setcounter{FancyVerbLine}{19}
%
% \subsection{Preliminaries}
%
% First we load the common oxref features.
%
%    \begin{macrocode}
\RequireBibliographyStyle{oxref}
%    \end{macrocode}
%
% Here we set some defaults different to the standard ones. The author
% can still override them.
%
%    \begin{macrocode}
\ExecuteBibliographyOptions{uniquename=init,giveninits,labeldateparts,sorting=nyt,pagetracker}
%    \end{macrocode}
%
% \subsection{Date merging option}
%
% We provide the `mergedate' option from the `authoryear' style. The
% OGS and NHR consistently use a variant that is between basic and
% compact, in that the year is always moved but the month and day are
% never moved; this is implemented as the `year' value.
%
%    \begin{macrocode}
\DeclareBibliographyOption{mergedate}[true]{%
  \ifcsdef{bbx@opt@mergedate@#1}
    {\csuse{bbx@opt@mergedate@#1}}
    {\PackageError{biblatex}
       {Invalid option 'mergedate=#1'}
       {Valid values are 'maximum', 'compact', 'basic', 'minimum',\MessageBreak
        'year', 'true' (=year), and 'false'.}}}
%    \end{macrocode}
%
% The various values affect the definition of the following macro.
%
%    \begin{macrocode}
\providebibmacro*{date+extrayear}{}
%    \end{macrocode}
%
% Here we provide the various possible definitions.
%
% \begin{itemize}
%   \item `true' is the same as `year' (see below).
% \end{itemize}
%
%    \begin{macrocode}
\def\bbx@opt@mergedate@true{\bbx@opt@mergedate@year}
%    \end{macrocode}
%
% \begin{itemize}
%   \item `maximum' merges date/issue with date label. Note that we
%   clear fields rather than the bibmacros as the `authoryear' style does.
% \end{itemize}
%
%    \begin{macrocode}
\def\bbx@opt@mergedate@maximum{%
  \renewbibmacro*{date+extrayear}{%
    \iffieldundef{labelyear}
      {}
      {\printtext[parens]{%
         \printfield{issue}%
         \setunit*{\addspace}%
         \iffieldsequal{year}{labelyear}
           {\printdateextralabel}%
           {\printfield{labelyear}%
            \printfield{extrayear}}}%
       \clearfield{year}\clearfield{endyear}%
       \clearfield{month}\clearfield{endmonth}%
       \clearfield{day}\clearfield{endday}%
       \clearfield{issue}}}}%
%    \end{macrocode}
%
% \begin{itemize}
%   \item `compact' merges whole date with date label. Note that we
%   clear fields rather than the bibmacros as the `authoryear' style does.
% \end{itemize}
%
%    \begin{macrocode}
\def\bbx@opt@mergedate@compact{%
  \renewbibmacro*{date+extrayear}{%
    \iffieldundef{labelyear}
      {}
      {\printtext[parens]{%
         \iffieldsequal{year}{labelyear}
           {\printdateextralabel}%
           {\printfield{labelyear}%
            \printfield{extrayear}}}%
       \clearfield{year}\clearfield{endyear}%
       \clearfield{month}\clearfield{endmonth}%
       \clearfield{day}\clearfield{endday}}}%
  \renewbibmacro*{issue+date}{%
    \iffieldundef{issue}
      {}
      {\ifboolexpr{(
          test {\iffieldundef{volume}}
          and
          test {\iffieldundef{number}}
          ) and
          test {\iffieldundef{eid}}
        }{%
          \newunit
          \printfield{issue}%
        }{%
          \printtext[parens]{\printfield{issue}}%
        }}%
    \newunit}}
%    \end{macrocode}
%
% \begin{itemize}
%   \item `basic' merges year-only date with date label.
% \end{itemize}
%
%    \begin{macrocode}
\def\bbx@opt@mergedate@basic{%
  \renewbibmacro*{date+extrayear}{%
    \iffieldundef{labelyear}
      {}
      {\printtext[parens]{%
         \printfield{labelyear}%
         \printfield{extrayear}}}}%
  \renewbibmacro*{date}{%
    \iffieldundef{month}
      {}
      {\printdate}}%
  \renewbibmacro*{issue+date}{%
    \ifboolexpr{
      test {\iffieldundef{issue}}
      and
      test {\iffieldundef{month}}
    }
    {}
    {\ifboolexpr{(
        test {\iffieldundef{volume}}
        and
        test {\iffieldundef{number}}
        ) and
        test {\iffieldundef{eid}}
      }{%
        \newunit
        \printfield{issue}%
        \setunit*{\addspace}%
        \printdate
      }{%
        \printtext[parens]{%
          \printfield{issue}%
          \setunit*{\addspace}%
          \printdate}%
      }}%
    \newunit}}
%    \end{macrocode}
%
% \begin{itemize}
%   \item `year' merges only the year with date label.
% \end{itemize}
%
%    \begin{macrocode}
\def\bbx@opt@mergedate@year{%
  \renewbibmacro*{date+extrayear}{%
    \iffieldundef{labelyear}
    {}
    {\printtext[parens]{%
        \printfield{labelyear}%
        \printfield{extrayear}}%
      \iffieldundef{\thefield{labeldatesource}year}{%
        \iffieldundef{\thefield{labeldatesource}}{}{%
          \clearfield{\thefield{labeldatesource}}%
        }%
      }{%
        \clearfield{\thefield{labeldatesource}year}%
        \clearfield{end\thefield{labeldatesource}year}%
      }}}%
  \renewbibmacro*{issue+date}{%
    \ifboolexpr{
      test {\iffieldundef{issue}}
      and
      test {\iffieldundef{month}}
    }
    {}
    {\ifboolexpr{(
        test {\iffieldundef{volume}}
        and
        test {\iffieldundef{number}}
        ) and
        test {\iffieldundef{eid}}
      }{%
        \newunit
        \printfield{issue}%
        \setunit*{\addspace}%
        \printdate
      }{%
        \printtext[parens]{%
          \printfield{issue}%
          \setunit*{\addspace}%
          \printdate}%
      }}%
    \newunit}}
%    \end{macrocode}
%
% \begin{itemize}
%   \item `minimum' merges year-only date with year-only date label.
% \end{itemize}
%
%    \begin{macrocode}
\def\bbx@opt@mergedate@minimum{%
  \renewbibmacro*{date+extrayear}{%
    \iffieldundef{labelyear}
      {}
      {\printtext[parens]{%
         \printfield{labelyear}%
         \printfield{extrayear}}}}%
  \renewbibmacro*{date}{%
    \ifboolexpr{
      test {\iffieldundef{month}}
      and
      test {\iffieldundef{extrayear}}
    }
      {}
      {\printdate}}%
  \renewbibmacro*{issue+date}{%
    \ifboolexpr{
      test {\iffieldundef{issue}}
      and
      test {\iffieldundef{month}}
      and
      test {\iffieldundef{extrayear}}
    }
    {}
    {\ifboolexpr{(
        test {\iffieldundef{volume}}
        and
        test {\iffieldundef{number}}
        ) and
        test {\iffieldundef{eid}}
      }{%
        \newunit
        \printfield{issue}%
        \setunit*{\addspace}%
        \printdate
      }{%
        \printtext[parens]{%
          \printfield{issue}%
          \setunit*{\addspace}%
          \printdate}%
      }}%
    \newunit}}
%    \end{macrocode}
%
% \begin{itemize}
%   \item `false' suppresses any merging).
% \end{itemize}
%
%    \begin{macrocode}
\def\bbx@opt@mergedate@false{%
  \renewbibmacro*{date+extrayear}{%
    \iffieldundef{labelyear}
      {}
      {\printtext[parens]{%
         \printfield{labelyear}%
         \printfield{extrayear}}}}%
  \renewbibmacro*{date}{\printdate}%
  \renewbibmacro*{issue+date}{%
    \ifboolexpr{(
      test {\iffieldundef{volume}}
      and
      test {\iffieldundef{number}}
      ) and
      test {\iffieldundef{eid}}
    }{%
      \newunit\printtext{%
        \iffieldundef{issue}{%
          \usebibmacro{date}
        }{%
          \printfield{issue}%
          \setunit*{\addspace}%
          \usebibmacro{date}%
        }%
      }%
    }{%
      \ifboolexpr{
        test {\iffieldundef{issue}}
        and (
        test {\iffieldundef{year}}
        and
        test {\iffieldundef{month}}
        )
      }{}{%
        \printtext[parens]{%
          \iffieldundef{issue}{%
            \usebibmacro{date}%
          }{%
            \printfield{issue}%
            \setunit*{\addspace}%
            \usebibmacro{date}%
          }%
        }%
      }%
    }%
  \newunit}
}
%    \end{macrocode}
%
% We set the default to be `true'/`year'.
%
%    \begin{macrocode}
\ExecuteBibliographyOptions{mergedate}
%    \end{macrocode}
%
% The above clearfield commands would suppress any further date
% information being printed, but we need to allow months and days to be
% printed in the absence of a year. This means adapting the range
% calculation commands defined in `biblatex.sty' for use by
% \cs{printdate}.
%
%    \begin{macrocode}
\DeclareRobustCommand*{\mkdaterangefull}[2]{%
  \iffieldundef{#2year}{%
    \iffieldundef{#2month}{}{%
      \printtext[#2date]{%
        \csuse{mkbibdate#1}{}{#2month}{#2day}%
        \iffieldundef{#2endmonth}{}{%
          \iffieldequalstr{#2endmonth}{}{%
            \mbox{\bibdaterangesep}%
          }{%
            \bibdaterangesep
            \csuse{mkbibdate#1}{}{#2endmonth}{#2endday}}}}}%
  }{%
    \printtext[#2date]{%
      \csuse{mkbibdate#1}{#2year}{#2month}{#2day}%
      \iffieldundef{#2endyear}{%
      }{%
        \iffieldequalstr{#2endyear}{}{%
          \mbox{\bibdaterangesep}%
        }{%
          \bibdaterangesep
          \csuse{mkbibdate#1}{#2endyear}{#2endmonth}{#2endday}}}}}}

\DeclareRobustCommand*{\mkdaterangetrunc}[2]{%
  \iffieldundef{#2year}{%
    \iffieldundef{#2month}{}{%
      \printtext[#2date]{%
        \iffieldsequal{#2month}{#2endmonth}{%
          \csuse{mkbibdate#1}{}{}{#2day}%
        }{%
          \csuse{mkbibdate#1}{}{#2month}{#2day}}%
        \iffieldundef{#2endmonth}{}{%
          \iffieldequalstr{#2endmonth}{}{%
            \mbox{\bibdaterangesep}%
          }{%
            \bibdaterangesep
            \iffieldundef{#2endday}{%
              \csuse{mkbibdate#1}{}{#2endmonth}{}%
            }{%
              \csuse{mkbibdate#1}{}{#2endmonth}{%
                \blx@ox@compyear{\thefield{#2day}}{\thefield{#2endday}}}}}}}}%
  }{%
    \printtext[#2date]{%
      \iffieldsequal{#2year}{#2endyear}{%
        \iffieldsequal{#2month}{#2endmonth}{%
          \csuse{mkbibdate#1}{}{}{#2day}%
        }{%
          \csuse{mkbibdate#1}{}{#2month}{#2day}}%
      }{%
        \csuse{mkbibdate#1}{#2year}{#2month}{#2day}}%
      \iffieldundef{#2endyear}{%
      }{%
        \iffieldequalstr{#2endyear}{}{%
          \mbox{\bibdaterangesep}%
        }{%
          \bibdaterangesep
          \iffieldundef{#2endmonth}{%
            \blx@ox@compyear{\thefield{#2year}}{\thefield{#2endyear}}%
          }{%
            \csuse{mkbibdate#1}{#2endyear}{#2endmonth}{#2endday}}}}}}}

\DeclareRobustCommand*{\mkdaterangefullextra}[2]{%
  \iffieldundef{#2year}{%
    \iffieldundef{#2month}{}{%
      \printtext[#2date]{%
        \csuse{mkbibdate#1}{}{#2month}{#2day}%
        \iffieldundef{#2endmonth}{}{%
          \iffieldequalstr{#2endmonth}{}{%
            \mbox{\bibdaterangesep}%
          }{%
            \bibdaterangesep
            \csuse{mkbibdate#1}{}{#2endmonth}{#2endday}}}}}%
  }{%
    \printtext[#2date]{%
      \printtext{%
        \csuse{mkbibdate#1}{#2year}{#2month}{#2day}}%
      \iffieldundef{#2endyear}{%
        \printfield{extrayear}%
      }{%
        \iffieldequalstr{#2endyear}{}{%
          \printfield{extrayear}%
          \printtext{\mbox{\bibdaterangesep}}%
        }{%
          \printtext{%
            \bibdaterangesep
            \csuse{mkbibdate#1}{#2endyear}{#2endmonth}{#2endday}%
            \printfield{extrayear}}}}}}}

\DeclareRobustCommand*{\mkdaterangetruncextra}[2]{%
  \iffieldundef{#2year}{%
    \iffieldundef{#2month}{}{%
      \printtext[#2date]{%
        \iffieldsequal{#2month}{#2endmonth}{%
          \csuse{mkbibdate#1}{}{}{#2day}%
        }{%
          \csuse{mkbibdate#1}{}{#2month}{#2day}}%
        \iffieldundef{#2endmonth}{}{%
          \iffieldequalstr{#2endmonth}{}{%
            \mbox{\bibdaterangesep}%
          }{%
            \bibdaterangesep
            \iffieldundef{#2endday}{%
              \csuse{mkbibdate#1}{}{#2endmonth}{}%
            }{%
              \csuse{mkbibdate#1}{}{#2endmonth}{%
                \blx@ox@compyear{\thefield{#2day}}{\thefield{#2endday}}}}}}}}%
  }{%
    \printtext[#2date]{%
      \printtext{%
        \iffieldsequal{#2year}{#2endyear}{%
          \iffieldsequal{#2month}{#2endmonth}{%
            \csuse{mkbibdate#1}{}{}{#2day}%
          }{%
            \csuse{mkbibdate#1}{}{#2month}{#2day}}%
        }{%
          \csuse{mkbibdate#1}{#2year}{#2month}{#2day}}}%
      \iffieldundef{#2endyear}{%
        \printfield{extrayear}%
      }{%
        \iffieldequalstr{#2endyear}{}{%
          \printfield{extrayear}%
          \printtext{\mbox{\bibdaterangesep}}%
        }{%
          \printtext{%
            \bibdaterangesep
            \iffieldundef{#2endmonth}{%
              \blx@ox@compyear{\thefield{#2year}}{\thefield{#2endyear}}%
            }{%
              \csuse{mkbibdate#1}{#2endyear}{#2endmonth}{#2endday}}%
            \printfield{extrayear}}}}}}}
%    \end{macrocode}
%
% The `labelyear' falls back to `pubstate' if provided.
%
%    \begin{macrocode}
\DeclareLabeldate{%
  \field{origdate}
  \field{date}
  \field{year}
  \field{eventdate}
  \field{pubstate}
  \field{urldate}
  \literal{nodate}
}

%    \end{macrocode}
%
% \subsection{Bibliography formatting}
%
% We let \pkg{biblatex} measure shorthands so we can use
% \cs{shorthandwidth} later.
%
%    \begin{macrocode}
\DeclareFieldFormat{shorthandwidth}{#1}
%    \end{macrocode}
%
% We declare a variant of the `family-given' name format which exports the hash
% of the name, so we can compare `shortauthor' with `author' and `shorteditor'
% with `editor'.
%
%    \begin{macrocode}
\DeclareNameFormat{family-given-hash}{%
  \ifgiveninits{%
    \usebibmacro{name:family-given}%
      {\namepartfamily}%
      {\namepartgiveni}%
      {\namepartprefix}%
      {\namepartsuffix}%
  }{%
    \usebibmacro{name:family-given}%
      {\namepartfamily}%
      {\namepartgiven}%
      {\namepartprefix}%
      {\namepartsuffix}%
  }%
  \ifboolexpr{
    test {\ifcurrentname{shortauthor}}
    or
    test {\ifcurrentname{shorteditor}}
  }{%
    \savefield{hash}{\bbx@ox@firsthash}%
  }{%
    \savefield{hash}{\bbx@ox@secondhash}%
  }%
  \usebibmacro{name:andothers}%
}
\DeclareNameAlias{shortauthor}{family-given-hash}
\DeclareNameAlias{shorteditor}{family-given-hash}
%    \end{macrocode}
%
% Names at the head of the reference are family-given, but names elsewhere
% are given-family.
%
%    \begin{macrocode}
\DeclareNameAlias{sortname}{family-given}
\DeclareNameAlias{author}{family-given}
\DeclareNameAlias{editor}{family-given}
\DeclareNameAlias{translator}{family-given}
%    \end{macrocode}
%
% The bibliography formatting is just like `authoryear' except we do not
% eliminate item separation by default.
%
%    \begin{macrocode}
\defbibenvironment{bibliography}
  {\list
     {}
     {\setlength{\leftmargin}{\bibhang}%
      \setlength{\itemindent}{-\leftmargin}%
      \setlength{\itemsep}{\bibitemsep}%
      \setlength{\parsep}{\bibparsep}}}
  {\endlist}
  {\item}
%    \end{macrocode}
%
% The shorthand formatting is just like `authoryear'.
%
%    \begin{macrocode}
\defbibenvironment{shorthand}
  {\list
     {\printfield[shorthandwidth]{shorthand}}
     {\setlength{\labelwidth}{\shorthandwidth}%
      \setlength{\leftmargin}{\labelwidth}%
      \setlength{\labelsep}{\biblabelsep}%
      \addtolength{\leftmargin}{\labelsep}%
      \setlength{\itemsep}{\bibitemsep}%
      \setlength{\parsep}{\bibparsep}%
      \renewcommand*{\makelabel}[1]{##1\hss}}}
  {\endlist}
  {\item}
%    \end{macrocode}
%
% \subsection{Name and date formatting}
%
% We set up hashing just as in `authoryear'.
%
%    \begin{macrocode}
\InitializeBibliographyStyle{\global\undef\bbx@lasthash}
%    \end{macrocode}
%
% We ensure related entries do not interfere with the hashing.
%
%    \begin{macrocode}
\renewbibmacro*{begrelated}{%
  \booltrue{bbx@inset}}
\renewbibmacro*{endrelated}{%
  \usebibmacro*{bbx:savehash}}
%    \end{macrocode}
%
% We patch the `author' macro so that the date label information appears
% at the end (as in `authoryear').
%
%    \begin{macrocode}
\xpatchbibmacro{author}{%
  \global\undef\bbx@lasthash
}{%
  \global\undef\bbx@lasthash
  \usebibmacro{labeltitle}%
  \setunit*{\addspace}%
}{}{}%
\xapptobibmacro{author}{%
  \usebibmacro{date+extrayear}%
}{}{}%
%    \end{macrocode}
%
% We patch it further so that, where an author name has a corresponding
% (different) short author name, the short name is given first and the long
% name given in parentheses.
%
% This technique uses saveboxes to extract the hashes.
%
% An unfortunate side effect of this is that where short author names are
% provided, the author list is no longer truncated, but I hope this represents
% something of an edge case.
%
%    \begin{macrocode}
\newcounter{namepairs}
\newsavebox{\bbx@ox@shortnamebox}
\newsavebox{\bbx@ox@namebox}
\xpatchbibmacro{author}{%
  {\printnames{author}}\iffieldundef%
}{{%
  \ifboolexpr{%
    ( not test {\ifnameundef{shortauthor}} )
    and
    test {\ifnumequal{\value{shortauthor}}{\value{author}}}
  }{%
    \setcounter{namepairs}{0}%
    \savebibmacro{name:andothers}%
    \renewbibmacro*{name:andothers}{}%
    \whileboolexpr{%
      test {\ifnumcomp{\value{namepairs}}{<}{\value{author}}}
    }{%
      \stepcounter{namepairs}%
      \ifnumcomp{\value{namepairs}}{>}{1}{%
        \ifnumequal{\value{author}}{2}{%
          \setunit*{\addspace\bibstring{and}\addspace}%
        }{%
          \ifnumequal{\value{namepairs}}{\value{author}}{%
            \setunit*{\addcomma\space\bibstring{and}\addspace}%
          }{%
            \setunit*{\addcomma\space}%
          }%
        }%
      }{}%
      \savebox{\bbx@ox@shortnamebox}{%
        \printnames[family-given-hash][\value{namepairs}-\value{namepairs}]{shortauthor}%
      }%
      \savebox{\bbx@ox@namebox}{%
        \printnames[family-given-hash][\value{namepairs}-\value{namepairs}]{author}%
      }%
      \ifdefstrequal{\bbx@ox@firsthash}{\bbx@ox@secondhash}{%
        \unhbox\bbx@ox@namebox
      }{%
        \unhbox\bbx@ox@shortnamebox
        \addspace\mkbibparens{\unhbox\bbx@ox@namebox}%
      }%
    }%
    \restorebibmacro{name:andothers}%
  }{%
    \printnames{author}%
  }}\iffieldundef%
}{}{}%
%    \end{macrocode}
%
% We apply the same patches to `bbx:editor', but also move the editor
% string to after the date label.
%
%    \begin{macrocode}
\xpatchbibmacro{bbx:editor}{%
  \usebibmacro{#1}%
}{%
  \usebibmacro{date+extrayear}%
  \setunit*{\addspace}%
  \usebibmacro{#1}%
}{}{}%
\xpatchbibmacro{bbx:editor}{%
  \global\undef\bbx@lasthash
}{%
  \global\undef\bbx@lasthash
  \usebibmacro{labeltitle}%
  \setunit*{\addspace}%
}{}{}%
\xpatchbibmacro{bbx:editor}{%
  \printnames{editor}%
}{%
  \ifboolexpr{%
    ( not test {\ifnameundef{shorteditor}} )
    and
    test {\ifnumequal{\value{shorteditor}}{\value{editor}}}
  }{%
    \setcounter{namepairs}{0}%
    \savebibmacro{name:andothers}%
    \renewbibmacro*{name:andothers}{}%
    \whileboolexpr{%
      test {\ifnumcomp{\value{namepairs}}{<}{\value{editor}}}
    }{%
      \stepcounter{namepairs}%
      \ifnumcomp{\value{namepairs}}{>}{1}{%
        \ifnumequal{\value{editor}}{2}{%
          \setunit*{\addspace\bibstring{and}\addspace}%
        }{%
          \ifnumequal{\value{namepairs}}{\value{editor}}{%
            \setunit*{\addcomma\space\bibstring{and}\addspace}%
          }{%
            \setunit*{\addcomma\space}%
          }%
        }%
      }{}%
      \savebox{\bbx@ox@shortnamebox}{%
        \printnames[family-given-hash][\value{namepairs}-\value{namepairs}]{shorteditor}%
      }%
      \savebox{\bbx@ox@namebox}{%
        \printnames[family-given-hash][\value{namepairs}-\value{namepairs}]{editor}%
      }%
      \ifdefstrequal{\bbx@ox@firsthash}{\bbx@ox@secondhash}{%
        \unhbox\bbx@ox@namebox
      }{%
        \unhbox\bbx@ox@shortnamebox
        \addspace\mkbibparens{\unhbox\bbx@ox@namebox}%
      }%
    }%
    \restorebibmacro{name:andothers}%
  }{%
    \printnames{editor}%
  }%
}{}{}%
%    \end{macrocode}
%
% We apply the same patches to `bbx:translator' as we do for
% `bbx:editor', except for the shortening one.
%
%    \begin{macrocode}
\xpatchbibmacro{bbx:translator}{%
  \global\undef\bbx@lasthash
}{%
\global\undef\bbx@lasthash
\usebibmacro{labeltitle}%
\setunit*{\addspace}%
}{}{}%
\xpatchbibmacro{bbx:translator}{%
  \usebibmacro{#1}%
}{}{}{}%
\xapptobibmacro{bbx:translator}{%
  \usebibmacro{date+extrayear}%
  \setunit*{\addspace}%
  \usebibmacro{#1}%
}{}{}%
%    \end{macrocode}
%
% The `labeltitle' macro is exactly the same as the one from
% `authoryear'.
%
%    \begin{macrocode}
\newbibmacro*{labeltitle}{%
  \iffieldundef{label}
    {\iffieldundef{shorttitle}
       {\printfield{title}%
        \clearfield{title}}
       {\printfield[title]{shorttitle}}}
    {\printfield{label}}}
%    \end{macrocode}
%
% The date of `online' entries is printed in parentheses; as the year
% is moved after the author, this can lead to empty parentheses unless
% a month is printed, so we change the test accordingly.
%
%    \begin{macrocode}
\xpatchbibdriver{online}{%
  \iffieldundef{year}{}{%
    \printtext[parens]{\usebibmacro{date}}%
  }%
}{%
  \iffieldundef{month}{}{%
    \printtext[parens]{\usebibmacro{date}}%
  }%
}{}{}
%    \end{macrocode}
%
% \subsection{Journal division formatting}
%
% Oxyear uses a colon to demarcate page numbers
%
%    \begin{macrocode}
\renewcommand*{\bibpagespunct}{\addcolon\space}
%    \end{macrocode}
%
% \subsection{Source mappings}
%
% This is used by Biber to fix the following sorting issues:
% \begin{itemize}
%   \item
%   `proceedings' entries print the organization at the beginning.
%   \item
%   `proceedings' entries print the maintitle before the title.
% \end{itemize}
%
%    \begin{macrocode}
\DeclareStyleSourcemap{%
  \maps[datatype=bibtex]{%
    \map{%
      \pertype{proceedings}
      \step[fieldsource=organization]%
      \step[fieldset=sortname, origfieldval]%
      \step[fieldsource=maintitle]%
      \step[fieldset=sorttitle, origfieldval]%
      \step[fieldsource=volume]%
      \step[fieldset=sorttitle, append, origfieldval]%
    }%
  }%
}
%    \end{macrocode}
%
% \chapter{Citation styles}
%
% \iffalse
%</bbx-y>
%<*cbx-n>
% \fi
%
% \section{Notes style: oxnotes.cbx}
%
% \setcounter{FancyVerbLine}{19}
%
%    \begin{macrocode}
\RequireCitationStyle{verbose}
%    \end{macrocode}
%
% \iffalse
%</cbx-n>
%<*cbx-y>
% \fi
%
% \section{Author–year style: oxyear.cbx}
%
% \setcounter{FancyVerbLine}{19}
%
%    \begin{macrocode}
\RequireCitationStyle{authoryear}
%    \end{macrocode}
%
% \iffalse
%</cbx-y>
%<*lbx-gb>
% \fi
%
% \chapter{Localization modules}
%
% \section{British English: british-oxref.lbx}
%
% \setcounter{FancyVerbLine}{19}
%
%    \begin{macrocode}
\InheritBibliographyExtras{british}
\DeclareBibliographyExtras{%
  \protected\def\mkbibdatelong#1#2#3{%
    \iffieldundef{#3}
      {}
      {\stripzeros{\thefield{#3}}%
       \iffieldundef{#2}{}{\nobreakspace}}%
    \iffieldundef{#2}
      {}
      {\mkbibmonth{\thefield{#2}}%
       \iffieldundef{#1}{}{\space}}%
    \stripzeros{\thefield{#1}}}%
  \def\finalandcomma{\addcomma}%
  \def\bibtimesep{\addperiod}%
}
\DeclareBibliographyStrings{%
  inherit          = {british},
  % Peculiar to Oxref
  at               = {{at}{at}},
  under            = {{s.v.}{s.v.}},
  facsimile        = {{facsimile edition}{facs\adddotspace edn\adddot}},
  revised          = {{revised edition}{rev\adddotspace edn\adddot}},
  reviewof         = {{review of}{review of}},
  byperformer      = {{}{}},
  bydirector       = {{directed by}{dir\adddot}},
  byreader         = {{read by}{read by}},
  % Standard (NB. Oxford style is more sparing with `by')
  introduction     = {{introduction}{introduction}},
  withintroduction = {{with an introduction by}{with an introduction by}},
  byeditor         = {{edited by}{ed\adddot}},
  bycompiler       = {{compiled by}{comp\adddot}},
  byredactor       = {{redacted by}{red\adddot}},
  byfounder        = {{founded by}{found\adddot}},
  bycontinuator    = {{continued by}{cont\adddot}},
  bycollaborator   = {{in collaboration with}{in collab\adddotspace with}}, % FIXME: unsure
  bytranslator     = {{translated \lbx@lfromlang\ by}{trans\adddot\ \lbx@sfromlang}},
  bycommentator    = {{commented by}{comm\adddot}},
  byannotator      = {{annotated by}{annot\adddot}},
  withcommentator  = {{with a commentary by}{with a commentary by}},
  withannotator    = {{with annotations by}{with annotations by}},
  withintroduction = {{with an introduction by}{with an introduction by}},
  withforeword     = {{with a foreword by}{with a foreword by}},
  withafterword    = {{with an afterword by}{with an afterword by}},
  byeditortr       = {{edited and translated \lbx@lfromlang\ by}%
                      {ed\adddotspace and trans\adddot\ \lbx@sfromlang}},
  byeditorco       = {{edited and commented by}%
                      {ed\adddotspace and comm\adddot}},
  byeditoran       = {{edited and annotated by}%
                      {ed\adddotspace and annot\adddot}},
  byeditorin       = {{edited, with an introduction, by}%
                      {ed.\adddotspace with introduction}},
  byeditorfo       = {{edited, with a foreword, by}%
                      {ed.\adddotspace with foreword}},
  byeditoraf       = {{edited, with an afterword, by}%
                      {ed.\adddotspace with afterword}},
  byeditortrco     = {{edited, translated \lbx@lfromlang\finalandcomma\ and commented by}%
                      {ed.,\addabbrvspace trans\adddot\ \lbx@sfromlang\finalandcomma\ and comm\adddot}},
  byeditortran     = {{edited, translated \lbx@lfromlang\finalandcomma\ and annotated by}%
                      {ed.,\addabbrvspace trans\adddot\ \lbx@sfromlang\finalandcomma\ and annot\adddot}},
  byeditortrin     = {{edited and translated \lbx@lfromlang, with an introduction, by}%
                      {ed\adddotspace and trans\adddot\ \lbx@sfromlang\ with introduction}},
  byeditortrfo     = {{edited and translated \lbx@lfromlang, with a foreword, by}%
                      {ed\adddotspace and trans\adddot\ \lbx@sfromlang\ with foreword}},
  byeditortraf     = {{edited and translated \lbx@lfromlang, with an afterword, by}%
                      {ed\adddotspace and trans\adddot\ \lbx@sfromlang\ with afterword}},
  byeditorcoin     = {{edited and commented, with an introduction, by}%
                      {ed\adddotspace and comm\adddotspace with introduction}},
  byeditorcofo     = {{edited and commented, with a foreword, by}%
                      {ed\adddotspace and comm\adddotspace with foreword}},
  byeditorcoaf     = {{edited and commented, with an afterword, by}%
                      {ed\adddotspace and comm\adddotspace with afterword}},
  byeditoranin     = {{edited and annotated, with an introduction, by}%
                      {ed\adddotspace and annot\adddotspace with introduction}},
  byeditoranfo     = {{edited and annotated, with a foreword, by}%
                      {ed\adddotspace and annot\adddotspace with foreword}},
  byeditoranaf     = {{edited and annotated, with an afterword, by}%
                      {ed\adddotspace and annot\adddotspace with afterword}},
  byeditortrcoin   = {{edited, translated \lbx@lfromlang\finalandcomma\ and commented, with an introduction, by}%
                      {ed.,\addabbrvspace trans\adddot\ \lbx@sfromlang\finalandcomma\ and comm\adddotspace with introduction}},
  byeditortrcofo   = {{edited, translated \lbx@lfromlang\finalandcomma\ and commented, with a foreword, by}%
                      {ed.,\addabbrvspace trans\adddot\ \lbx@sfromlang\finalandcomma\ and comm\adddotspace with foreword}},
  byeditortrcoaf   = {{edited, translated \lbx@lfromlang\finalandcomma\ and commented, with an afterword, by}%
                      {ed.,\addabbrvspace trans\adddot\ \lbx@sfromlang\finalandcomma\ and comm\adddotspace with afterword}},
  byeditortranin   = {{edited, translated \lbx@lfromlang\finalandcomma\ and annotated, with an introduction, by}%
                      {ed.,\addabbrvspace trans\adddot\ \lbx@sfromlang\finalandcomma\ and annot\adddotspace with introduction}},
  byeditortranfo   = {{edited, translated \lbx@lfromlang\finalandcomma\ and annotated, with a foreword, by}%
                      {ed.,\addabbrvspace trans\adddot\ \lbx@sfromlang\finalandcomma\ and annot\adddotspace with foreword}},
  byeditortranaf   = {{edited, translated \lbx@lfromlang\finalandcomma\ and annotated, with an afterword, by}%
                      {ed.,\addabbrvspace trans\adddot\ \lbx@sfromlang\finalandcomma\ and annot\adddotspace with afterword}},
  bytranslatorco   = {{translated \lbx@lfromlang\ and commented by}%
                      {trans\adddot\ \lbx@sfromlang\ and comm\adddot\ by}},
  bytranslatoran   = {{translated \lbx@lfromlang\ and annotated by}%
                      {trans\adddot\ \lbx@sfromlang\ and annot\adddot\ by}},
  bytranslatorin   = {{translated \lbx@lfromlang, with an introduction, by}%
                      {trans\adddot\ \lbx@sfromlang\ with introduction}},
  bytranslatorfo   = {{translated \lbx@lfromlang, with a foreword, by}%
                      {trans\adddot\ \lbx@sfromlang\ with foreword}},
  bytranslatoraf   = {{translated \lbx@lfromlang, with an afterword, by}%
                      {trans\adddot\ \lbx@sfromlang\ with afterword}},
  bytranslatorcoin = {{translated \lbx@lfromlang\ and commented, with an introduction, by}%
                      {trans\adddot\ \lbx@sfromlang\ and comm\adddotspace with introduction}},
  bytranslatorcofo = {{translated \lbx@lfromlang\ and commented, with a foreword, by}%
                      {trans\adddot\ \lbx@sfromlang\ and comm\adddotspace with foreword}},
  bytranslatorcoaf = {{translated \lbx@lfromlang\ and commented, with an afterword, by}%
                      {trans\adddot\ \lbx@sfromlang\ and comm\adddotspace with afterword}},
  bytranslatoranin = {{translated \lbx@lfromlang\ and annotated, with an introduction, by}%
                      {trans\adddot\ \lbx@sfromlang\ and annot\adddotspace with introduction}},
  bytranslatoranfo = {{translated \lbx@lfromlang\ and annotated, with a foreword, by}%
                      {trans\adddot\ \lbx@sfromlang\ and annot\adddotspace with foreword}},
  bytranslatoranaf = {{translated \lbx@lfromlang\ and annotated, with an afterword, by}%
                      {trans\adddot\ \lbx@sfromlang\ and annot\adddotspace with afterword}},
  idem             = {{idem}{id\adddot}},
  idemsm           = {{idem}{id\adddot}},
  idemsf           = {{eadem}{ead\adddot}},
  idemsn           = {{idem}{id\adddot}},
  idempm           = {{eidem}{eid\adddot}},
  idempf           = {{eaedem}{eaed\adddot}},
  idempn           = {{eadem}{ead\adddot}},
  idempp           = {{eidem}{eid\adddot}},
  newseries        = {{new series}{\mkbibacro{NS}}},
  oldseries        = {{old series}{\mkbibacro{OS}}},
  edition          = {{edition}{edn\adddot}},
  reprint          = {{reprint\nopunct}{repr\adddot\nopunct}},
  reviewof         = {{review of}{review of}},
  reprintfrom      = {{from}{from}},
  urlseen          = {{accessed}{accessed}},
  am               = {{a\adddot m\adddot}{a\adddot m\adddot}},
  pm               = {{p\adddot m\adddot}{p\adddot m\adddot}},
}
%    \end{macrocode}
% \iffalse
%</lbx-gb>
% \fi

\def\Version{2015/05/25 v0.1}
\ProvidesFile{oxref.tex}
    [\Version\space Biblatex styles inspired by the Oxford Guide to Style]

\documentclass[extrafontsizes,11pt,a4paper,oneside]{memoir}

\newoutputstream{src}
\newoutputstream{oxnotes}
\newoutputstream{oxyear}

%%%%%%%%%%%%%%%%%%%%%%%%%%%%%%%%%%%%%%%%%%%%%%%%%%%%%%%%%%%%%%%%%%%%%%%%
%%% COMMON PREAMBLE
%%%%%%%%%%%%%%%%%%%%%%%%%%%%%%%%%%%%%%%%%%%%%%%%%%%%%%%%%%%%%%%%%%%%%%%%

\openoutputfile{oxref-preamble.tex}{src}
\begin{writeverbatim}{src}
\setlrmarginsandblock{4cm}{2cm}{*}
\setulmarginsandblock{2.5cm}{2.5cm}{*}
\checkandfixthelayout
% Document divisions
\chapterstyle{hangnum}
\hangsecnum
\setsecheadstyle{\Large\bfseries\raggedright}
\setsubsecheadstyle{\large\bfseries\scshape\raggedright}
\setsecnumdepth{subsection}
% Pagination and headers
\nouppercaseheads
\makeoddhead{myheadings}{\textsc{\leftmark}}{}{\thepage}
%\makeevenhead{myheadings}{\thepage}{}{\textsc{\leftmark}}
\makepsmarks{myheadings}{%
  \def\chaptermark##1{\markboth{##1}{##1}}%
  \def\sectionmark##1{\markright{##1}}%
}
\pagestyle{myheadings}
\aliaspagestyle{title}{empty}
% Paragraphs and lists
\setlength{\parindent}{0pt}\nonzeroparskip
\firmlists

\usepackage[british]{babel}
%   \DeclareUnicodeCharacter{2212}{\textminus}
\usepackage[mono=false,defaultfeatures={SmallCapsFeatures={Letters=SmallCaps,Renderer=Basic,Ligatures=NoCommon}}]{libertine}
\makeatletter
\@ifpackageloaded{fontspec}{%
  \setmonofont[Scale=MatchLowercase,StylisticSet=1]{Inconsolatazi4}
  \usepackage{fontawesome}
  \newcommand{\booksym}{\makebox[1em][c]{\faicon{book}}}
  \newcommand{\cogsym}{\makebox[1em][c]{\faicon{cog}}}
}{%
  \usepackage[utf8]{inputenc}
  \usepackage[varl]{zi4}
  \usepackage{dictsym}
  \newcommand{\booksym}{\makebox[1em][c]{\dsliterary}}
  \newcommand{\cogsym}{\makebox[1em][c]{\dstechnical}}
}
\makeatother

\usepackage{microtype,xpatch,csquotes,xcolor,xparse,multicol}
\definecolor{Green}{rgb}{0,.5,0}
\colorlet{ok}{Green}
\colorlet{bad}{red}
\colorlet{hacked}{orange}
\colorlet{manual}{purple}

\usepackage{tcolorbox}
\tcbuselibrary{listingsutf8,breakable,skins,xparse}
\lstloadlanguages{[LaTeX]TeX}
\lstset
  { columns=fullflexible
  , basicstyle=\ttfamily
  , language={[LaTeX]TeX}
  , texcsstyle=*\color{red!75!black}
  , moredelim=**[s][\color{violet}]{[}{]}
  , moredelim=**[s][\color{blue!75!black}]{\{}{\}}
  , mathescape
  , escapechar=~
  }

\usepackage
[backend=biber%
,hyperref=false%
,isbn=false]{biblatex}
\bibliography{oxref}
\usepackage[pdfborder={0 0 0}]{hyperref}
\usepackage[noabbrev,capitalize,nameinlink]{cleveref}

\newcommand*{\lit}[1]{\textsf{#1}}
\newcommand*{\code}[1]{`\textsf{#1}'}
\newcommand*{\aside}[1]{\textcolor{violet}{[\textsc{tip:} #1]}}

% For debugging; userd is either ‘bad’, ‘hacked’, ‘manual’
\makeatletter
\def\CiteStatus{bad}
\newcommand{\dbgcolor}[2]{%
  \bgroup
  \blx@citecmdinit
  \blx@citeinit
  \def\blx@precode{}%
  \def\blx@postcode{}%
  \def\blx@loopcode{%
    \iffieldundef{userd}
    {\xdef\CiteStatus{ok}}
    {\xdef\CiteStatus{\abx@field@userd}}}%
  \blx@citeloop{#1}%
  \textcolor{\CiteStatus}{#2}%
  \egroup
}
\makeatother
\NewTColorBox{bibexbox}{D(){ok}d<>om}%
  {bicolor
  ,colframe = #1
  ,colback = #1!5!white
  ,colbacklower = white
  ,fontlower = \footnotesize
  ,before upper = {\hangfrom{\booksym\space}}
  ,IfNoValueTF={#3}%
    {after upper = {\par\hangfrom{\cogsym\space}\fullcite{#4}.}
    }%
    {after upper = {\par\hangfrom{\cogsym\space}\fullcite[#3]{#4}.}
    ,title = {\texttt{\string\fullcite[#3]\{#4\}}}
    }
  ,IfNoValueTF={#2}{}%
    {overlay = {
      \node[anchor=south east,text=teal] at (frame.south east) {#2};
      }
    }
  }
\NewTotalTColorBox{\spec}{m}%
  {enhanced
  ,sharp corners = west
  ,colframe = teal
  ,colback = teal!5!white
  ,toprule = 0pt
  ,bottomrule = 0pt
  ,rightrule = 0pt
  }{#1}
% \testrow compares target format with generated format. Arguments:
% #1 = target format
% #2 = page range
% #3 = bib key
\NewDocumentCommand\testrow{mom}{\biburlsetup #1\\%
  \makebox[0pt][r]{\footnotesize\textcolor{teal}{#3}\quad}%
  \IfNoValueTF{#2}{\dbgcolor{#3}{\fullcite{#3}.}}%
  {\dbgcolor{#3}{\fullcite[#2]{#3}.}}\par}
% \egcite gives a citation as an example. Arguments:
% #1 = page range
% #2 = bib key
\NewDocumentCommand\egcite{om}{%
  \makebox[0pt][r]{\footnotesize\textcolor{teal}{#2}\quad}%
  \IfNoValueTF{#1}{\dbgcolor{#2}{\fullcite{#2}.}}%
  {\dbgcolor{#2}{\fullcite[#1]{#2}}.}}
\newcommand{\egauthor}[1]{%
  \makebox[0pt][r]{\footnotesize\textcolor{teal}{#1}\quad}%
  \dbgcolor{#1}{\citeauthor{#1}.}%
}

\frenchspacing
\end{writeverbatim}
\closeoutputstream{src}

\PassOptionsToPackage{bibstyle=oxnotes,citestyle=verbose}{biblatex}
\input{oxref-preamble}

%%%%%%%%%%%%%%%%%%%%%%%%%%%%%%%%%%%%%%%%%%%%%%%%%%%%%%%%%%%%%%%%%%%%%%%%
%%% OXREF BBX (Common settings)
%%%%%%%%%%%%%%%%%%%%%%%%%%%%%%%%%%%%%%%%%%%%%%%%%%%%%%%%%%%%%%%%%%%%%%%%

\openoutputfile{oxref.bbx}{src}
\addtostream{src}{\string\ProvidesFile{oxref.bbx}}
\addtostream{src}{\space\space\space\space[\Version\space Base settings for biblatex styles inspired by the Oxford Guide to Style]}
\begin{writeverbatim}{src}
%
% Dependencies:
% \begin{itemize}
% \item
% For ease of maintenance, we will patch some definitions with 'xpatch'
% instead of writing out our own in full.
% \item
% We will manipulate strings with 'xstring'.
% \end{itemize}
%
\RequirePackage{xpatch}
\RequirePackage{xstring}
%
% Here we set some defaults different to the standard ones. The author
% can still override them.
%
\ExecuteBibliographyOptions{urldate=comp,pagetracker}
%
% Language support may be widened in future, but for now we support the
% following:
%
\DeclareLanguageMapping{english}{british-oxref}
\DeclareLanguageMapping{british}{british-oxref}
\DeclareLanguageMapping{american}{british-oxref}
%
% We provide some additional bibliography strings.
%
\NewBibliographyString{at}
\NewBibliographyString{under}
\NewBibliographyString{facsimile}
\NewBibliographyString{revised}
\NewBibliographyString{byperformer}
\NewBibliographyString{bydirector}
\NewBibliographyString{byreader}
%
% We provide some magic values for triggering formatting changes.
%
% \begin{itemize}
% \item |\isInferred| relates to author names;
% \item |\whenPublished| relates to URL dates.
% \end{itemize}
%
\def\isInferred{inferred}
\def\whenPublished{published}
%
% We base our styles on the standard on the principle of least surprise
% (and to aid with maintenance in the face of new 'biblatex' versions).
%
\RequireBibliographyStyle{standard}
%
% Here are the default punctuation conventions:
%
\renewcommand*{\labelnamepunct}{\addcomma\space}
\renewcommand*{\newunitpunct}{\addcomma\space}
\renewcommand*{\subtitlepunct}{\addcolon\space}
\renewcommand*{\intitlepunct}{\nopunct\space}
\renewcommand*{\bibnamedash}{\rule[0.5ex]{2em}{\normalrulethickness}\addthinspace}
\renewcommand*{\revsdnamedelim}{\addcomma}
%
% Traditionally, Oxford style (for the humanities) is to write author
% names in small capitals in the bibliography, surname first; but in
% citations, they are written in normal case in natural order.
%
\DeclareNameFormat{scstartname}{%
  \ifbibliography{%
    \ifnumequal{\value{listcount}}{1}{%
      \iffirstinits{%
        \usebibmacro{name:last-first}%
          {\ifblank{#1}{}{\textsc{#1}}}%
          {\ifblank{#4}{}{\textsc{#4}}}%
          {\ifblank{#5}{}{\textsc{#5}}}%
          {\ifblank{#7}{}{\textsc{#7}}}%
      }{%
        \usebibmacro{name:last-first}%
          {\ifblank{#1}{}{\textsc{#1}}}%
          {\ifblank{#3}{}{\textsc{#3}}}%
          {\ifblank{#5}{}{\textsc{#5}}}%
          {\ifblank{#7}{}{\textsc{#7}}}%
      }%
      \ifblank{#3#5}
        {}
        {\usebibmacro{name:revsdelim}}%
    }{%
      \iffirstinits{%
        \usebibmacro{name:last-first}%
          {\ifblank{#1}{}{\textsc{#1}}}%
          {\ifblank{#4}{}{\textsc{#4}}}%
          {\ifblank{#5}{}{\textsc{#5}}}%
          {\ifblank{#7}{}{\textsc{#7}}}%
      }{%
        \usebibmacro{name:last-first}%
          {\ifblank{#1}{}{\textsc{#1}}}%
          {\ifblank{#3}{}{\textsc{#3}}}%
          {\ifblank{#5}{}{\textsc{#5}}}%
          {\ifblank{#7}{}{\textsc{#7}}}%
      }%
    }%
  }{%
    \iffirstinits
      {\usebibmacro{name:first-last}{#1}{#4}{#5}{#7}}%
      {\usebibmacro{name:first-last}{#1}{#3}{#5}{#7}}%
  }%
  \usebibmacro{name:andothers}%
}
%
% The small caps aspect of this seems to be falling out of favour so we
% provide a facility for authors to switch it off.
%
\DeclareNameFormat{lcstartname}{%
  \ifbibliography{%
    \ifnumequal{\value{listcount}}{1}{%
      \iffirstinits
        {\usebibmacro{name:last-first}{#1}{#4}{#5}{#7}}%
        {\usebibmacro{name:last-first}{#1}{#3}{#5}{#7}}%
      \ifblank{#3#5}
        {}
        {\usebibmacro{name:revsdelim}}%
    }{%
      \iffirstinits
        {\usebibmacro{name:last-first}{#1}{#4}{#5}{#7}}%
        {\usebibmacro{name:last-first}{#1}{#3}{#5}{#7}}%
    }%
  }{%
    \iffirstinits
      {\usebibmacro{name:first-last}{#1}{#4}{#5}{#7}}%
      {\usebibmacro{name:first-last}{#1}{#3}{#5}{#7}}%
  }%
  \usebibmacro{name:andothers}%
}
%
% When using natural order for names, there is no comma before numeric
% suffixes, but there is before textual ones (e.g. ‘junior’).
%
\xpatchbibmacro{name:first-last}%
  {\bibnamedelimd\mkbibnameaffix{#4}}%
  {\ifnumeral{#4}{}{\addcomma}\bibnamedelimd\mkbibnameaffix{#4}}{}{}
%
% For pseudonyms, Oxford style is to put the other name in parentheses
% after the main one. We use |nameaddon| for this.
%
\DeclareFieldFormat{nameaddon}{\mkbibparens{#1}}
%
% Oxford style is to use dashes instead of repeating author names.
% Note that 'biblatex' sets are treated differently.
%
\InitializeBibliographyStyle{%
  \global\undef\bbx@lasthash}
\newbibmacro*{bbx:savehash}{%
  \savefield{fullhash}{\bbx@lasthash}}
\newbool{bbx@inset}
\DeclareBibliographyDriver{set}{%
  \booltrue{bbx@inset}%
  \entryset{}{}%
  \newunit\newblock
  \usebibmacro{setpageref}%
  \finentry}
\newbibmacro*{bbx:dashcheck}[2]{%
  \ifboolexpr{
    test {\iffieldequals{fullhash}{\bbx@lasthash}}
    and
    not test \iffirstonpage
    and
    (
    not bool {bbx@inset}
    or
    test {\iffieldequalstr{entrysetcount}{1}}
    )
  }
  {#1}
  {#2}}
%
% Macro for author names.
%
\renewbibmacro*{author}{%
  \ifboolexpr{
    test \ifuseauthor
    and
    not test {\ifnameundef{author}}
  }
  {\usebibmacro{bbx:dashcheck}
    {\bibnamedash}
    {\usebibmacro{bbx:savehash}%
      \iffieldequals{authortype}{\isInferred}
      {\mkbibbrackets{\printnames{author}}\clearfield{authortype}}%
      {\printnames{author}}%
      \iffieldundef{nameaddon}
      {}
      {\setunit{\addspace}%
        \printfield{nameaddon}}%
      \iffieldundef{authortype}
      {\setunit{\addspace}}
      {\setunit{\addcomma\space}}}%
    \iffieldundef{authortype}
    {}
    {\usebibmacro{authorstrg}%
      \setunit{\addspace}}}%
  {\global\undef\bbx@lasthash}}
%
% Macro for editor names.
%
\renewbibmacro*{editor}{%
  \usebibmacro{bbx:editor}{editorstrg}}
\renewbibmacro*{editor+others}{%
  \usebibmacro{bbx:editor}{editor+othersstrg}}
\newbibmacro*{bbx:editor}[1]{%
  \ifboolexpr{
    test \ifuseeditor
    and
    not test {\ifnameundef{editor}}
  }
  {\usebibmacro{bbx:dashcheck}
    {\bibnamedash}
    {\usebibmacro{bbx:savehash}%
      \printnames{editor}%
      \addspace}%
    \mkbibparens{\usebibmacro{#1}}%
    \clearname{editor}%
    \setunit{\addspace}}%
  {\global\undef\bbx@lasthash}}
%
% Macro for author of a collection.
%
\renewbibmacro*{bybookauthor}{%
  \ifnamesequal{author}{bookauthor}%
  {\bibstring{idem\thefield{gender}}}%
  {\printnames{bookauthor}}}
%
% Macro for editor of a collection.
%
\newbibmacro*{bookeditor}{%
  \ifnamesequal{author}{editor}%
  {\bibstring{idem\thefield{gender}}%
    \addspace%
    \mkbibparens{\usebibmacro{editor+othersstrg}}%
    \clearname{editor}}%
  {\undef\bbx@lasthash\usebibmacro{editor+others}}}
%
% Year ranges are truncated (e.g.~2012–3). Therefore, we provide a
% command that compares two years, and outputs a full or truncated
% version of the second year, depending on how similar it is to the
% first. It is extracted and adapted from code by Marco Daniel.%
% \footnote{\url{http://tex.stackexchange.com/questions/23483/}}
%
\newcommand{\blx@ox@compyear}[2]{%
  \def\num@one{#1}%
  \def\num@two{#2}%
  \StrLen{\num@one}[\num@one@len]%
  \StrLen{\num@two}[\num@two@len]%
  \IfEq{\num@one@len}{\num@two@len}%
  {\StrCompare{\num@one}{\num@two}[\Result]%
    \ifnum\num@two@len>3%
    \IfStrEq{\Result}{2}{\def\Result{1}}{}%
    \fi
    \StrGobbleLeft{0\num@two}{\Result}}%
  {\num@two}%
}
%
% We use this to create some truncated date formats.
% TODO: convert these to patched commands.
%
\renewrobustcmd*{\mkbibrangetrunc}[2]{%
  \iffieldundef{#2year}%
  {}%
  {\printtext{%
      \iffieldsequal{#2year}{#2endyear}%
      {\iffieldsequal{#2month}{#2endmonth}%
        {\csuse{mkbibdate#1}{}{}{#2day}}%
        {\csuse{mkbibdate#1}{}{#2month}{#2day}}}%
      {\csuse{mkbibdate#1}{#2year}{#2month}{#2day}}%
      \iffieldundef{#2endyear}%
      {}%
      {\iffieldequalstr{#2endyear}{}%
        {\mbox{\bibdatedash}}%
        {\bibdatedash
          % In the original, only the false part of the following occurs
          \iffieldundef{#2month}%
          {\blx@ox@compyear{\thefield{#2year}}{\thefield{#2endyear}}}%
          {\csuse{mkbibdate#1}{#2endyear}{#2endmonth}{#2endday}}}}}}%
}
\renewrobustcmd*{\mkbibrangetruncextra}[2]{%
  \iffieldundef{#2year}%
  {}%
  {\printtext{%
      \iffieldsequal{#2year}{#2endyear}%
      {\iffieldsequal{#2month}{#2endmonth}%
        {\csuse{mkbibdate#1}{}{}{#2day}}%
        {\csuse{mkbibdate#1}{}{#2month}{#2day}}}%
      {\csuse{mkbibdate#1}{#2year}{#2month}{#2day}}}%
    \iffieldundef{#2endyear}%
    {\printfield{extrayear}}%
    {\iffieldequalstr{#2endyear}{}%
      {\printfield{extrayear}%
        \printtext{\mbox{\bibdatedash}}}%
      {\printtext{%
          \bibdatedash
          % In the original, only the false part of the following occurs
          \iffieldundef{#2month}%
          {\blx@ox@compyear{\thefield{#2year}}{\thefield{#2endyear}}}%
          {\csuse{mkbibdate#1}{#2endyear}{#2endmonth}{#2endday}}%
          \printfield{extrayear}}}}}%
}
% Dates generally

% - allow date to be prefaced by a type
\let\dateType\@empty
\DeclareEntryOption{datetype}{\def\dateType{#1\addspace}}
\renewbibmacro*{date}{%
  \ifx\@empty\dateType\else\printtext{\dateType}\fi
  \printdate}

% Editions

\DeclareFieldFormat{edition}{%
  \ifinteger{#1}
  {\mkbibordedition{#1}~\bibstring{edition}}
  {\ifbibstring{#1}{\bibstring{#1}}{#1\isdot}}}

% Pages (no pp.)

\DeclareFieldFormat{pages}{\mkcomprange{#1}}
\DeclareFieldFormat{postnote}{\iffieldundef{pages}{}%
  {\addspace\bibstring{at}\addspace}%
  \mkcomprange{#1}}

\renewbibmacro*{postnote}{%
  \iffieldundef{postnote}
  {}
  % This test makes a lot of assumptions and needs more sophistication
  {\iffieldundef{pages}
    % The \printtext following is a complete hack as print tracker has become unreliable
    {\printtext{\unspace}\setunit{\postnotedelim}\newblock\printfield{postnote}}
    {\printfield{postnote}}%
  }}

% URLs

\DeclareFieldFormat{url}{$\langle$\url{#1}$\rangle$}
\DeclareFieldFormat{urldate}{\bibstring{urlseen}\space#1}
\DeclareFieldFormat{urldatepublished}{\bibstring{epublished}\space#1}

\renewbibmacro*{url+urldate}{%
  \iffieldundef{url}{}{\nopunct}%
  \printfield{url}%
  \iffieldundef{urlyear}
  {\iffieldundef{url}{}{\nopunct}}
  {\newunit
    \printtext[urldate]{\printurldate}}}
% - allow custom urldate type (these have a different position)
\DeclareEntryOption{urldatetype}{%
  \DeclareFieldFormat{urldate}{\printtext{#1}\space##1}%
  \renewbibmacro*{url+urldate}{%
    \iffieldundef{url}{}{%
      \newunit
      \printtext[urldate]{\printurldate}%
      \setunit*{\addspace}%
      \printfield{url}%
      \nopunct
    }%
  }%
}

% Addenda

\renewbibmacro*{addendum+pubstate}{%
  \iffieldundef{pubstate}{}{\nopunct}%
  \printfield[parens]{pubstate}%
  \setunit{\addperiod\addspace}\newblock
  \printfield{addendum}}

% Relationships: reprints

\renewbibmacro*{related:reprintfrom}[1]{%
  \entrydata*{#1}{%
    \iffieldundef{journaltitle}{% Reprint from book
      \iffieldundef{maintitle}
      {\printfield[title:hook]{booktitle}}
      {\printfield[title:hook]{maintitle}}%
      \newunit\newblock
      \usebibmacro{byeditor+others}%
      \newunit\newblock
      \iffieldundef{volume}{%
      }{%
        \printfield{volume}%
        \printfield{part}%
      }%
      \newunit\newblock
      \printfield{note}%
      \newunit\newblock
      \usebibmacro{series+number+edition+publisher+location+date}%
      \newunit\newblock
      \usebibmacro{chapter+pages}
    }{% Reprint from periodical
      \usebibmacro{journal+issuetitle}%
      \newunit
      \usebibmacro{byeditor+others}%
      \iffieldundef{note}{%
        \newunit
      }{%
        \setunit{\addsemicolon\addspace}%
      }%
      \usebibmacro{note+pages}%
    }}}

% Adjustments to articles

\DeclareFieldFormat[article,periodical,review]{entrysubtype}{\mkbibbrackets{#1}}

\newbibmacro*{journaltype}{%
  \iffieldequalstr{entrysubtype}{online journal}%
  {\nopunct\printfield{entrysubtype}}{}%
}

\renewbibmacro*{journal+issuetitle}{%
  \ifboolexpr{
    test {\iffieldundef{title}}
    and
    not test {\iffieldundef{issuetitle}}
  }{%
    \usebibmacro{issue}%
    \clearfield{issuetitle}%
    \setunit{\addspace =\addspace}}{}%
  \usebibmacro{journal}%
  \usebibmacro{journaltype}%
  \setunit*{\addspace}%
  \iffieldundef{series}
  {\newunit}
  {\newunit
    \printfield{series}%
    \setunit{\addspace}}%
  \usebibmacro{volume+number+eid}%
  \setunit{\addspace}%
  \usebibmacro{issue+date}%
  \setunit{\addsemicolon\space}%
  \usebibmacro{issue}%
  \newunit}

\renewbibmacro*{title+issuetitle}{%
  \usebibmacro{periodical}%
  \usebibmacro{journaltype}%
  \setunit*{\addspace}%
  \iffieldundef{series}
  {\newunit}
  {\newunit
    \printfield{series}%
    \setunit{\addspace}}%
  \usebibmacro{volume+number+eid}%
  \setunit{\addspace}%
  \usebibmacro{issue+date}%
  \setunit{\addcolon\space}%
  \usebibmacro{issue}%
  \newunit}

\renewbibmacro*{volume+number+eid}{%
  \printfield{volume}%
  \setunit*{\addslash}%
  \printfield{number}%
  \setunit{\addcomma\space}%
  \printfield{eid}}

\renewbibmacro*{issue+date}{%
  \ifboolexpr{(
    test {\iffieldundef{volume}}
    and
    test {\iffieldundef{number}}
    ) and
    test {\iffieldundef{eid}}
  }%
  {\newunit\printtext{%
      \iffieldundef{issue}
      {\usebibmacro{date}}
      {\printfield{issue}%
        \setunit*{\addspace}%
        \usebibmacro{date}}}}%
  {\ifboolexpr{
      test {\iffieldundef{issue}}
      and
      test {\iffieldundef{year}}
    }{}{%
      \printtext[parens]{%
        \iffieldundef{issue}
        {\usebibmacro{date}}
        {\printfield{issue}%
          \setunit*{\addspace}%
          \usebibmacro{date}}}}}%
  \newunit}

% Adjustments to books

\DeclareFieldFormat[bookinbook]{citetitle}{\mkbibquote{#1\isdot}}
\DeclareFieldFormat[bookinbook]{title}{\mkbibquote{#1\isdot}}

\DeclareFieldFormat[book,bookinbook,inbook,mvbook,%
collection,incollection,mvcollection,%
proceedings,inproceedings,mvproceedings,%
reference,inreference,mvreference]{volume}{\Rn{#1}}

\renewbibmacro*{series+number}{%
  \printfield{series}%
  \newunit
  \printfield{number}}

\newbibmacro*{origdate}{\printorigdate}

\newbibmacro*{edition+publisher+location+date}{%
  \printlist{origlocation}%
  \iflistundef{origpublisher}%
  {\setunit*{\addcomma\space}}%
  {\setunit*{\addcolon\space}}%
  \printlist{origpublisher}%
  \setunit*{\addcomma\space}%
  \printorigdate%
  %\usebibmacro{origdate}%
  \setunit*{\addsemicolon\space}%
  \printfield{edition}%
  \setunit*{\addcomma\space}%
  \printlist{location}%
  \iflistundef{publisher}%
  {\setunit*{\addcomma\space}}%
  {\setunit*{\addcolon\space}}%
  \printlist{publisher}%
  \setunit*{\addcomma\space}%
  \usebibmacro{date}%
}

\DeclareFieldFormat{publication}{\mkbibparens{#1}}

\newbibmacro*{series+number+edition+publisher+location+date}{%
  \nopunct\printtext[publication]{%
    \usebibmacro{series+number}%
    \setunit*{\addsemicolon\addspace}%
    \usebibmacro{edition+publisher+location+date}%
  }%
}

\newbibmacro*{maintitle+volume}{%
  \ifboolexpr{
    test {\iffieldundef{maintitle}}
    or
    test {\iffieldundef{volume}}
  }{}%
  {\printtext[publication]{%
      \bibstring{volume}\addspace
      \printfield{volume}\addspace\bibstring{ofseries}\addspace
      \usebibmacro{maintitle}}}
}

% Adjustments to online

\DeclareFieldFormat[online]{title}{\mkbibquote{#1\isdot}}
\DeclareFieldFormat[online]{citetitle}{\mkbibquote{#1\isdot}}
\DeclareFieldFormat[online]{entrysubtype}{\mkbibbrackets{#1}}

% Adjustments to proceedings

\renewbibmacro*{event+venue+date}{%
  \printfield{eventtitle}%
  \ifboolexpr{
    test {\iffieldundef{venue}}
    and
    test {\iffieldundef{eventyear}}
  }
  {}
  {\ifboolexpr{
      test {\iffieldundef{language}}
      and
      test {\iffieldundef{eventtitle}}
    }{\setunit{\mkbibemph{\addcolon}\addspace}}%
    {\setunit{\addcolon\addspace}}%
    \printfield{venue}%
    \setunit*{\addcomma\space}%
    \printeventdate}%
  \newunit}

% Adjustments to reference

\newbibmacro*{series+number+publisher+location+date}{%
  \nopunct\printtext[publication]{%
    \usebibmacro{series+number}%
    \setunit{\addsemicolon\space}%
    \usebibmacro{publisher+location+date}%
  }%
}

% Adjustments to reports and theses

\newbibmacro*{type+number}{%
  \printfield{type}%
  \newunit
  \printfield{number}}

\renewbibmacro*{institution+location+date}{%
  \nopunct\printtext[publication]{%
    \printlist{location}%
    \iflistundef{institution}%
    {\setunit*{\addcomma\space}}%
    {\setunit*{\addcolon\space}}%
    \printlist{institution}%
    \setunit*{\addcomma\space}%
    \usebibmacro{date}%
  }%
}

\newbibmacro*{type+number+edition+institution+location+date}{%
  \nopunct\printtext[publication]{%
    \usebibmacro{type+number}%
    \setunit{\addsemicolon\space}%
    \printfield{edition}%
    \setunit*{\addcomma\space}%
    \printlist{location}%
    \iflistundef{institution}%
    {\setunit*{\addcomma\space}}%
    {\setunit*{\addcolon\space}}%
    \printlist{institution}%
    \setunit*{\addcomma\space}%
    \usebibmacro{date}%
  }%
}

\newbibmacro*{type+series+number+edition+organization+publisher+location+date}{%
  \nopunct\printtext[publication]{%
    \printfield{type}%
    \setunit*{\addcomma\space}%
    \usebibmacro{series+number}%
    \setunit{\addsemicolon\space}%
    \printfield{edition}%
    \setunit*{\addcomma\space}%
    \iflistundef{publisher}%
    {}%
    {\printlist{organization}%
      \setunit*{\addcomma\space}}%
    \printlist{location}%
    \iflistundef{publisher}%
    {\iflistundef{organization}%
      {\setunit*{\addcomma\space}}%
      {\setunit*{\addcolon\space}%
        \printlist{organization}}}%
    {\setunit*{\addcolon\space}
      \printlist{publisher}}%
    \setunit*{\addcomma\space}%
    \usebibmacro{date}%
  }%
}

% Adjustments to review

\DeclareFieldFormat[review]{title}{\mkbibquote{#1\isdot}}
\DeclareFieldFormat[review]{volume}{#1}% volume of a journal
\DeclareFieldFormat[review]{number}{#1}% number of a journal
\DeclareFieldFormat[review]{series}{% series of a journal
  \ifinteger{#1}
  {\mkbibordseries{#1}~\bibstring{jourser}}
  {\ifbibstring{#1}{\bibstring{#1}}{#1}}}

% - allow reference to reviewed item to be embedded within the reference to the review
% - TODO: switch to using biblatex's related item mechanism instead
\newbibmacro*{reviewof}{}
\DeclareBibliographyCategory{reviewed}
\DeclareEntryOption{reviewed}{%
  \renewbibmacro*{reviewof}{%
    \bibstring{reviewof}\addspace
    \fullcite{#1}\printtext{\unspace}% <- hack to ensure punctuation is triggered
  }%
  \addtocategory{reviewed}{#1}%
}

% Somewhat cleaner Biber method
\newbibmacro*{related:reviewof}[1]{%
  \entrydata*{#1}{%
    \iffieldundef{journaltitle}{% Review of book
      \usebibmacro{author}%
      \setunit{\labelnamepunct}\newblock
      \iffieldequalstr{entrysubtype}{sequence}%
      {\usebibmacro{title}}%
      {\usebibmacro{maintitle+title}}%
      \newunit
      \printlist{language}%
      \newunit\newblock
      \usebibmacro{byauthor}%
      \newunit\newblock
      \usebibmacro{byeditor+others}%
      \newunit\newblock
      \iffieldequalstr{entrysubtype}{sequence}{%
        \printfield{note}%
        \newunit\newblock
        \usebibmacro{maintitle+volume}%
        \newunit
      }{%
        \printfield{volumes}%
        \newunit\newblock
        \iffieldundef{maintitle}{%
          \printfield{volume}%
          \printfield{part}%
        }{}%
        \newunit
        \printfield{note}%
        \newunit\newblock
      }%
      \usebibmacro{series+number+edition+publisher+location+date}%
    }{% Review of periodical
      \usebibmacro{author/translator+others}%
      \setunit{\labelnamepunct}\newblock
      \usebibmacro{title}%
      \newunit
      \printlist{language}%
      \newunit\newblock
      \usebibmacro{byauthor}%
      \newunit\newblock
      \usebibmacro{bytranslator+others}%
      \newunit\newblock
      \printfield{version}%
      \newunit\newblock
      \usebibmacro{journal+issuetitle}%
      \newunit
      \usebibmacro{byeditor+others}%
      \iffieldundef{note}{\newunit}{\setunit{\addsemicolon\addspace}}
      \usebibmacro{note+pages}%
    }%
  }%
}

% Adjustments to music

\newbibmacro*{publisher+type+series+number+date}{%
  \nopunct\printtext[publication]{%
    \printfield{howpublished}%
    \setunit*{\addcomma\space}%
    \printlist{publisher}%
    \setunit*{\addspace}%
    \printfield{type}%
    \setunit*{\addspace}%
    \printfield{series}%
    \setunit*{\addcomma\space}%
    \printfield{number}%
    \setunit{\addcomma\space}%
    \usebibmacro{date}%
  }%
}

% Adjustments to aliases

\DeclareBibliographyAlias{mvreference}{reference}
%
% There is no more.
%
\endinput
\end{writeverbatim}
\closeoutputstream{src}

%%%%%%%%%%%%%%%%%%%%%%%%%%%%%%%%%%%%%%%%%%%%%%%%%%%%%%%%%%%%%%%%%%%%%%%%
%%% OXNOTES BBX
%%%%%%%%%%%%%%%%%%%%%%%%%%%%%%%%%%%%%%%%%%%%%%%%%%%%%%%%%%%%%%%%%%%%%%%%

\openoutputfile{oxnotes.bbx}{oxnotes}
\addtostream{oxnotes}{\string\ProvidesFile{oxnotes.bbx}}
\addtostream{oxnotes}{\space\space\space\space[\Version\space Footnote-based biblatex style inspired by the Oxford Guide to Style]}
\begin{writeverbatim}{oxnotes}
%
% First we load the common oxref features.
%
\RequireBibliographyStyle{oxref}
%
% To make it easier for authors to switch between name formats, we
% provide a layer of indirection. The default is to have names at the
% start of a bibliography entry in small capitals. Authors can switch
% this off with |\DeclareNameAlias{oxrefstartname}{lcstartname}|.
%
\DeclareNameAlias{oxrefstartname}{scstartname}
\DeclareNameAlias{author}{oxrefstartname}
\DeclareNameAlias[book]{editor}{oxrefstartname}
\DeclareNameAlias[booklet]{editor}{oxrefstartname}
\DeclareNameAlias[collection]{editor}{oxrefstartname}
\DeclareNameAlias[manual]{editor}{oxrefstartname}
\DeclareNameAlias[misc]{editor}{oxrefstartname}
\DeclareNameAlias[mvbook]{editor}{oxrefstartname}
\DeclareNameAlias[mvcollection]{editor}{oxrefstartname}
\DeclareNameAlias[mvproceedings]{editor}{oxrefstartname}
\DeclareNameAlias[mvreference]{editor}{oxrefstartname}
\DeclareNameAlias[online]{editor}{oxrefstartname}
\DeclareNameAlias[proceedings]{editor}{oxrefstartname}
\DeclareNameAlias[reference]{editor}{oxrefstartname}

% Adjustments to existing drivers

\DeclareBibliographyDriver{article}{%
  \usebibmacro{bibindex}%
  \usebibmacro{begentry}%
  \usebibmacro{author/translator+others}%
  \setunit{\labelnamepunct}\newblock
  \usebibmacro{title}%
  \newunit
  \printlist{language}%
  \newunit\newblock
  \usebibmacro{byauthor}%
  \newunit\newblock
  \usebibmacro{bytranslator+others}%
  \newunit\newblock
  \printfield{version}%
  \newunit\newblock
  \usebibmacro{journal+issuetitle}%
  \newunit
  \usebibmacro{byeditor+others}%
  \iffieldundef{note}{\newunit}{\setunit{\addsemicolon\addspace}}
  \usebibmacro{note+pages}%
  \newunit\newblock
  \iftoggle{bbx:isbn}
  {\printfield{issn}}
  {}%
  \newunit\newblock
  \usebibmacro{doi+eprint+url}%
  \newunit\newblock
  \usebibmacro{addendum+pubstate}%
  \setunit{\bibpagerefpunct}\newblock
  \usebibmacro{pageref}%
  \setunit{\addperiod\addspace}\newblock
  \iftoggle{bbx:related}
  {\usebibmacro{related:init}%
    \usebibmacro{related}}
  {}%
  \usebibmacro{finentry}}

\DeclareBibliographyDriver{book}{%
  \usebibmacro{bibindex}%
  \usebibmacro{begentry}%
  \usebibmacro{author}%
  \setunit{\labelnamepunct}\newblock
  \iffieldequalstr{entrysubtype}{sequence}%
  {\usebibmacro{title}}%
  {\usebibmacro{maintitle+title}}%
  \newunit
  \printlist{language}%
  \newunit\newblock
  \usebibmacro{byauthor}%
  \newunit\newblock
  \usebibmacro{byeditor+others}%
  \newunit\newblock
  \iffieldequalstr{entrysubtype}{sequence}%
  {\printfield{note}%
    \newunit\newblock
    \usebibmacro{maintitle+volume}%
    \newunit}
  {\printfield{volumes}%
    \newunit\newblock
    \iffieldundef{maintitle}
    {\printfield{volume}%
      \printfield{part}}
    {}%
    \newunit
    \printfield{note}%
    \newunit\newblock}%
  \usebibmacro{series+number+edition+publisher+location+date}%
  \newunit\newblock
  \usebibmacro{chapter+pages}%
  \newunit
  \printfield{pagetotal}%
  \newunit\newblock
  \iftoggle{bbx:isbn}
  {\printfield{isbn}}
  {}%
  \newunit\newblock
  \usebibmacro{doi+eprint+url}%
  \newunit\newblock
  \usebibmacro{addendum+pubstate}%
  \setunit{\bibpagerefpunct}\newblock
  \usebibmacro{pageref}%
  \setunit{\addperiod\addspace}\newblock
  \iftoggle{bbx:related}
  {\usebibmacro{related:init}%
    \usebibmacro{related}}
  {}%
  \usebibmacro{finentry}}

\DeclareBibliographyDriver{inbook}{%
  \usebibmacro{bibindex}%
  \usebibmacro{begentry}%
  \usebibmacro{author/translator+others}%
  \setunit{\labelnamepunct}\newblock
  \usebibmacro{title}%
  \newunit
  \printlist{language}%
  \newunit\newblock
  \usebibmacro{byauthor}%
  \newunit\newblock
  \usebibmacro{in:}%
  \usebibmacro{bybookauthor}%
  \newunit\newblock
  \iffieldequalstr{entrysubtype}{sequence}%
  {\usebibmacro{booktitle}}%
  {\usebibmacro{maintitle+booktitle}}%
  \newunit\newblock
  \usebibmacro{byeditor+others}%
  \newunit\newblock
  \iffieldequalstr{entrysubtype}{sequence}%
  {\printfield{note}%
    \newunit\newblock
    \usebibmacro{maintitle+volume}%
    \newunit}
  {\printfield{volumes}%
    \newunit\newblock
    \iffieldundef{maintitle}
    {\printfield{volume}%
      \printfield{part}}
    {}%
    \newunit
    \printfield{note}%
    \newunit\newblock}%
  \usebibmacro{series+number+edition+publisher+location+date}%
  \newunit\newblock
  \usebibmacro{chapter+pages}%
  \newunit\newblock
  \iftoggle{bbx:isbn}
  {\printfield{isbn}}
  {}%
  \newunit\newblock
  \usebibmacro{doi+eprint+url}%
  \newunit\newblock
  \usebibmacro{addendum+pubstate}%
  \setunit{\bibpagerefpunct}\newblock
  \usebibmacro{pageref}%
  \setunit{\addperiod\addspace}\newblock
  \iftoggle{bbx:related}
  {\usebibmacro{related:init}%
    \usebibmacro{related}}
  {}%
  \usebibmacro{finentry}}

\DeclareBibliographyDriver{collection}{%
  \usebibmacro{bibindex}%
  \usebibmacro{begentry}%
  \usebibmacro{editor+others}%
  \setunit{\labelnamepunct}\newblock
  \iffieldequalstr{entrysubtype}{sequence}%
  {\usebibmacro{title}}%
  {\usebibmacro{maintitle+title}}%
  \newunit
  \printlist{language}%
  \newunit\newblock
  \usebibmacro{byeditor+others}%
  \newunit\newblock
  \iffieldequalstr{entrysubtype}{sequence}%
  {\printfield{note}%
    \newunit\newblock
    \usebibmacro{maintitle+volume}%
    \newunit}
  {\printfield{volumes}%
    \newunit\newblock
    \iffieldundef{maintitle}
    {\printfield{volume}%
      \printfield{part}}
    {}%
    \newunit
    \printfield{note}%
    \newunit\newblock}%
  \usebibmacro{series+number+edition+publisher+location+date}%
  \newunit\newblock
  \usebibmacro{chapter+pages}%
  \newunit
  \printfield{pagetotal}%
  \newunit\newblock
  \iftoggle{bbx:isbn}
  {\printfield{isbn}}
  {}%
  \newunit\newblock
  \usebibmacro{doi+eprint+url}%
  \newunit\newblock
  \usebibmacro{addendum+pubstate}%
  \setunit{\bibpagerefpunct}\newblock
  \usebibmacro{pageref}%
  \setunit{\addperiod\addspace}\newblock
  \iftoggle{bbx:related}
  {\usebibmacro{related:init}%
    \usebibmacro{related}}
  {}%
  \usebibmacro{finentry}}

\DeclareBibliographyDriver{incollection}{%
  \usebibmacro{bibindex}%
  \usebibmacro{begentry}%
  \usebibmacro{author/translator+others}%
  \setunit{\labelnamepunct}\newblock
  \usebibmacro{title}%
  \newunit
  \printlist{language}%
  \newunit\newblock
  \usebibmacro{byauthor}%
  \newunit\newblock
  \usebibmacro{in:}%
  \usebibmacro{bookeditor}%
  \newunit\newblock
  \iffieldequalstr{entrysubtype}{sequence}%
  {\usebibmacro{booktitle}}%
  {\usebibmacro{maintitle+booktitle}}%
  \newunit\newblock
  \iffieldequalstr{entrysubtype}{sequence}%
  {\printfield{note}%
    \newunit\newblock
    \usebibmacro{maintitle+volume}%
    \newunit}
  {\printfield{volumes}%
    \newunit\newblock
    \iffieldundef{maintitle}
    {\printfield{volume}%
      \printfield{part}}
    {}%
    \newunit
    \printfield{note}%
    \newunit\newblock}%
  \usebibmacro{series+number+edition+publisher+location+date}%
  \newunit\newblock
  \usebibmacro{chapter+pages}%
  \newunit\newblock
  \iftoggle{bbx:isbn}
  {\printfield{isbn}}
  {}%
  \newunit\newblock
  \usebibmacro{doi+eprint+url}%
  \newunit\newblock
  \usebibmacro{addendum+pubstate}%
  \setunit{\bibpagerefpunct}\newblock
  \usebibmacro{pageref}%
  \setunit{\addperiod\addspace}\newblock
  \iftoggle{bbx:related}
  {\usebibmacro{related:init}%
    \usebibmacro{related}}
  {}%
  \usebibmacro{finentry}}

\DeclareBibliographyDriver{online}{%
  \usebibmacro{bibindex}%
  \usebibmacro{begentry}%
  \usebibmacro{author/editor+others/translator+others}%
  \setunit{\labelnamepunct}\newblock
  \usebibmacro{title}%
  \newunit
  \usebibmacro{maintitle}
  \setunit{\addspace}\newblock
  \printfield{entrysubtype}%
  \newunit
  \printlist{language}%
  \newunit\newblock
  \usebibmacro{byauthor}%
  \newunit\newblock
  \usebibmacro{byeditor+others}%
  \newunit\newblock
  \printfield{version}%
  \newunit
  \printfield{note}%
  \newunit\newblock
  \printlist{organization}%
  \newunit\newblock
  \iffieldundef{year}{}{%
    \printtext[parens]{\usebibmacro{date}}%
  }%
  \newunit\newblock
  \iftoggle{bbx:eprint}
  {\usebibmacro{eprint}}
  {}%
  \newunit\newblock
  \usebibmacro{url+urldate}%
  \newunit\newblock
  \usebibmacro{addendum+pubstate}%
  \setunit{\bibpagerefpunct}\newblock
  \usebibmacro{pageref}%
  \setunit{\addperiod\addspace}\newblock
  \iftoggle{bbx:related}
  {\usebibmacro{related:init}%
    \usebibmacro{related}}
  {}%
  \usebibmacro{finentry}}

\DeclareBibliographyDriver{proceedings}{%
  \usebibmacro{bibindex}%
  \usebibmacro{begentry}%
  \usebibmacro{maintitle+title}%
  \newunit
  \printlist{language}%
  \newunit\newblock
  \usebibmacro{event+venue+date}%
  \newunit\newblock
  \usebibmacro{byeditor+others}%
  \newunit\newblock
  \printfield{volumes}%
  \newunit\newblock
  \iffieldundef{maintitle}
  {\printfield{volume}%
    \printfield{part}}
  {}%
  \newunit
  \printfield{note}%
  \newunit\newblock
  \printlist{organization}%
  \newunit
  \usebibmacro{series+number+edition+publisher+location+date}%
  \newunit\newblock
  \usebibmacro{chapter+pages}%
  \newunit
  \printfield{pagetotal}%
  \newunit\newblock
  \iftoggle{bbx:isbn}
  {\printfield{isbn}}
  {}%
  \newunit\newblock
  \usebibmacro{doi+eprint+url}%
  \newunit\newblock
  \usebibmacro{addendum+pubstate}%
  \setunit{\bibpagerefpunct}\newblock
  \usebibmacro{pageref}%
  \setunit{\addperiod\addspace}\newblock
  \iftoggle{bbx:related}
  {\usebibmacro{related:init}%
    \usebibmacro{related}}
  {}%
  \usebibmacro{finentry}}

\DeclareBibliographyDriver{inproceedings}{%
  \usebibmacro{bibindex}%
  \usebibmacro{begentry}%
  \usebibmacro{author/translator+others}%
  \setunit{\labelnamepunct}\newblock
  \usebibmacro{title}%
  \newunit
  \printlist{language}%
  \newunit\newblock
  \usebibmacro{byauthor}%
  \newunit\newblock
  \usebibmacro{in:}%
  \usebibmacro{maintitle+booktitle}%
  \newunit\newblock
  \usebibmacro{event+venue+date}%
  \newunit\newblock
  \usebibmacro{byeditor+others}%
  \newunit\newblock
  \printfield{volumes}%
  \newunit\newblock
  \iffieldundef{maintitle}
  {\printfield{volume}%
    \printfield{part}}
  {}%
  \newunit
  \printfield{note}%
  \newunit\newblock
  \printlist{organization}%
  \newunit
  \usebibmacro{series+number+edition+publisher+location+date}%
  \newunit\newblock
  \usebibmacro{chapter+pages}%
  \newunit\newblock
  \iftoggle{bbx:isbn}
  {\printfield{isbn}}
  {}%
  \newunit\newblock
  \usebibmacro{doi+eprint+url}%
  \newunit\newblock
  \usebibmacro{addendum+pubstate}%
  \setunit{\bibpagerefpunct}\newblock
  \usebibmacro{pageref}%
  \setunit{\addperiod\addspace}\newblock
  \iftoggle{bbx:related}
  {\usebibmacro{related:init}%
    \usebibmacro{related}}
  {}%
  \usebibmacro{finentry}}

\DeclareBibliographyDriver{reference}{%
  \usebibmacro{bibindex}%
  \usebibmacro{begentry}%
  \usebibmacro{author}%
  \setunit{\labelnamepunct}\newblock
  \iffieldequalstr{entrysubtype}{sequence}%
  {\usebibmacro{title}}%
  {\usebibmacro{maintitle+title}}%
  \newunit
  \printlist{language}%
  \newunit\newblock
  \usebibmacro{byauthor}%
  \newunit\newblock
  \printfield{edition}%
  \newunit
  \usebibmacro{byeditor+others}%
  \newunit\newblock
  \iffieldequalstr{entrysubtype}{sequence}%
  {\printfield{note}%
    \newunit\newblock
    \usebibmacro{maintitle+volume}%
    \newunit}
  {\printfield{volumes}%
    \newunit\newblock
    \iffieldundef{maintitle}
    {\printfield{volume}%
      \printfield{part}}
    {}%
    \newunit
    \printfield{note}%
    \newunit\newblock}%
  \usebibmacro{series+number+publisher+location+date}%
  \newunit\newblock
  \usebibmacro{chapter+pages}%
  \newunit
  \printfield{pagetotal}%
  \newunit\newblock
  \iftoggle{bbx:isbn}
  {\printfield{isbn}}
  {}%
  \newunit\newblock
  \usebibmacro{doi+eprint+url}%
  \newunit\newblock
  \usebibmacro{addendum+pubstate}%
  \setunit{\bibpagerefpunct}\newblock
  \usebibmacro{pageref}%
  \setunit{\addperiod\addspace}\newblock
  \iftoggle{bbx:related}
  {\usebibmacro{related:init}%
    \usebibmacro{related}}
  {}%
  \usebibmacro{finentry}}

\DeclareBibliographyDriver{inreference}{%
  \usebibmacro{bibindex}%
  \usebibmacro{begentry}%
  \usebibmacro{author}%
  \setunit{\labelnamepunct}\newblock
  \usebibmacro{title}%
  \newunit
  \printlist{language}%
  \newunit\newblock
  \usebibmacro{byauthor}%
  \newunit\newblock
  \usebibmacro{in:}%
  \usebibmacro{bookeditor}%
  \newunit\newblock
  \iffieldequalstr{entrysubtype}{sequence}%
  {\usebibmacro{booktitle}}%
  {\usebibmacro{maintitle+booktitle}}%
  \newunit\newblock
  \printfield{edition}%
  \newunit
  \usebibmacro{byeditor+others}%
  \newunit\newblock
  \iffieldequalstr{entrysubtype}{sequence}%
  {\printfield{note}%
    \newunit\newblock
    \usebibmacro{maintitle+volume}%
    \newunit}
  {\printfield{volumes}%
    \newunit\newblock
    \iffieldundef{maintitle}
    {\printfield{volume}%
      \printfield{part}}
    {}%
    \newunit
    \printfield{note}%
    \newunit\newblock}%
  \usebibmacro{series+number+publisher+location+date}%
  \newunit\newblock
  \usebibmacro{chapter+pages}%
  \newunit
  \printfield{pagetotal}%
  \newunit\newblock
  \iftoggle{bbx:isbn}
  {\printfield{isbn}}
  {}%
  \newunit\newblock
  \usebibmacro{doi+eprint+url}%
  \newunit\newblock
  \usebibmacro{addendum+pubstate}%
  \setunit{\bibpagerefpunct}\newblock
  \usebibmacro{pageref}%
  \setunit{\addperiod\addspace}\newblock
  \iftoggle{bbx:related}
  {\usebibmacro{related:init}%
    \usebibmacro{related}}
  {}%
  \usebibmacro{finentry}}

\DeclareBibliographyDriver{report}{%
  \usebibmacro{bibindex}%
  \usebibmacro{begentry}%
  \usebibmacro{author}%
  \setunit{\labelnamepunct}\newblock
  \usebibmacro{title}%
  \newunit
  \printlist{language}%
  \newunit\newblock
  \usebibmacro{byauthor}%
  \newunit\newblock
  \printfield{version}%
  \newunit
  \printfield{note}%
  \newunit\newblock
  \usebibmacro{type+number+edition+institution+location+date}%
  \newunit\newblock
  \usebibmacro{chapter+pages}%
  \newunit
  \printfield{pagetotal}%
  \newunit\newblock
  \iftoggle{bbx:isbn}
  {\printfield{isrn}}
  {}%
  \newunit\newblock
  \usebibmacro{doi+eprint+url}%
  \newunit\newblock
  \usebibmacro{addendum+pubstate}%
  \setunit{\bibpagerefpunct}\newblock
  \usebibmacro{pageref}%
  \setunit{\addperiod\addspace}\newblock
  \iftoggle{bbx:related}
  {\usebibmacro{related:init}%
    \usebibmacro{related}}
  {}%
  \usebibmacro{finentry}}

\DeclareBibliographyDriver{manual}{%
  \usebibmacro{bibindex}%
  \usebibmacro{begentry}%
  \usebibmacro{author}%
  \setunit{\labelnamepunct}\newblock
  \usebibmacro{maintitle+title}%
  \newunit
  \printlist{language}%
  \newunit\newblock
  \usebibmacro{byauthor}%
  \newunit\newblock
  \usebibmacro{byeditor+others}%
  \newunit\newblock
  \printfield{volumes}%
  \newunit\newblock
  \iffieldundef{maintitle}
  {\printfield{volume}%
    \printfield{part}}
  {}%
  \newunit
  \printfield{note}%
  \newunit\newblock
  \printfield{type}%
  \newunit
  \printfield{version}%
  \newunit
  \printfield{note}%
  \newunit\newblock
  \usebibmacro{type+series+number+edition+organization+publisher+location+date}%
  \newunit\newblock
  \usebibmacro{chapter+pages}%
  \newunit
  \printfield{pagetotal}%
  \newunit\newblock
  \iftoggle{bbx:isbn}
  {\printfield{isbn}}
  {}%
  \newunit\newblock
  \usebibmacro{doi+eprint+url}%
  \newunit\newblock
  \usebibmacro{addendum+pubstate}%
  \setunit{\bibpagerefpunct}\newblock
  \usebibmacro{pageref}%
  \setunit{\addperiod\addspace}\newblock
  \iftoggle{bbx:related}
  {\usebibmacro{related:init}%
    \usebibmacro{related}}
  {}%
  \usebibmacro{finentry}}

% Previously unsupported types

\DeclareBibliographyDriver{review}{%
  \usebibmacro{bibindex}%
  \usebibmacro{begentry}%
  \usebibmacro{author/translator+others}%
  \setunit{\labelnamepunct}\newblock
  \usebibmacro{title}%
  \newunit
  \printlist{language}%
  \newunit\newblock
  \usebibmacro{byauthor}%
  \newunit\newblock
  \usebibmacro{bytranslator+others}%
  \newunit\newblock
  \printfield{version}%
  \newunit\newblock
  \usebibmacro{reviewof}%
  \newunit\newblock
  \iftoggle{bbx:related}{%
    \usebibmacro{related:init}%
    \usebibmacro{related}%
  }{}%
  \newunit\newblock
  \usebibmacro{in:}%
  \usebibmacro{journal+issuetitle}%
  \newunit
  \usebibmacro{byeditor+others}%
  \iffieldundef{note}{\newunit}{\setunit{\addsemicolon\addspace}}
  \usebibmacro{note+pages}%
  \newunit\newblock
  \iftoggle{bbx:isbn}
  {\printfield{issn}}
  {}%
  \newunit\newblock
  \usebibmacro{doi+eprint+url}%
  \newunit\newblock
  \usebibmacro{addendum+pubstate}%
  \setunit{\bibpagerefpunct}\newblock
  \usebibmacro{pageref}%
  \usebibmacro{finentry}}

\DeclareBibliographyDriver{audio}{%
  \usebibmacro{bibindex}%
  \usebibmacro{begentry}%
  \usebibmacro{author}%
  \setunit{\labelnamepunct}\newblock
  \iffieldequalstr{entrysubtype}{sequence}%
  {\usebibmacro{title}}%
  {\usebibmacro{maintitle+title}}%
  \newunit
  \printlist{language}%
  \newunit\newblock
  \usebibmacro{byauthor}%
  \newunit\newblock
  \usebibmacro{byeditor+others}%
  \newunit\newblock
  \iffieldequalstr{entrysubtype}{sequence}%
  {\printfield{note}%
    \newunit\newblock
    \usebibmacro{maintitle+volume}%
    \newunit}
  {\printfield{volumes}%
    \newunit\newblock
    \iffieldundef{maintitle}
    {\printfield{volume}%
      \printfield{part}}
    {}%
    \newunit
    \printfield{note}%
    \newunit\newblock}%
  \usebibmacro{publisher+type+series+number+date}%
  \newunit\newblock
  \usebibmacro{chapter+pages}%
  \newunit
  \printfield{pagetotal}%
  \newunit\newblock
  \iftoggle{bbx:isbn}
  {\printfield{isbn}}
  {}%
  \newunit\newblock
  \usebibmacro{doi+eprint+url}%
  \newunit\newblock
  \usebibmacro{addendum+pubstate}%
  \setunit{\bibpagerefpunct}\newblock
  \usebibmacro{pageref}%
  \setunit{\addperiod\addspace}\newblock
  \iftoggle{bbx:related}
  {\usebibmacro{related:init}%
    \usebibmacro{related}}
  {}%
  \usebibmacro{finentry}}

\DeclareBibliographyAlias{music}{audio}

%% \DeclareBibliographyDriver{movie}{%
%% \DeclareBibliographyDriver{video}{%

\endinput
\end{writeverbatim}
\closeoutputstream{oxnotes}

%%%%%%%%%%%%%%%%%%%%%%%%%%%%%%%%%%%%%%%%%%%%%%%%%%%%%%%%%%%%%%%%%%%%%%%%
%%% OXYEAR BBX
%%%%%%%%%%%%%%%%%%%%%%%%%%%%%%%%%%%%%%%%%%%%%%%%%%%%%%%%%%%%%%%%%%%%%%%%

\openoutputfile{oxyear.bbx}{oxyear}
\addtostream{oxyear}{\string\ProvidesFile{oxyear.bbx}}
\addtostream{oxyear}{\space\space\space\space[\Version\space Author-year biblatex style inspired by the Oxford Guide to Style]}
\begin{writeverbatim}{oxyear}
%
% First we load the common oxref features.
%
\RequireBibliographyStyle{oxref}

\endinput
\end{writeverbatim}
\closeoutputstream{oxyear}

%%%%%%%%%%%%%%%%%%%%%%%%%%%%%%%%%%%%%%%%%%%%%%%%%%%%%%%%%%%%%%%%%%%%%%%%
%%% BRITISH-OXREF LBX
%%%%%%%%%%%%%%%%%%%%%%%%%%%%%%%%%%%%%%%%%%%%%%%%%%%%%%%%%%%%%%%%%%%%%%%%

\openoutputfile{british-oxref.lbx}{src}
\begin{writeverbatim}{src}
\ProvidesFile{british-oxref.lbx}
\end{writeverbatim}
\addtostream{src}{\space\space\space\space[\Version\space British English conventions required by the biblatex-oxref styles]}
\begin{writeverbatim}{src}

\InheritBibliographyExtras{british}
\DeclareBibliographyExtras{%
  \protected\def\mkbibdatelong#1#2#3{%
    \iffieldundef{#3}
      {}
      {\stripzeros{\thefield{#3}}%
       \iffieldundef{#2}{}{\nobreakspace}}%
    \iffieldundef{#2}
      {}
      {\mkbibmonth{\thefield{#2}}%
       \iffieldundef{#1}{}{\space}}%
    \stripzeros{\thefield{#1}}}%
  \def\finalandcomma{\addcomma}%
}
\DeclareBibliographyStrings{%
  inherit          = {british},
  % Peculiar to Oxref
  at               = {{at}{at}},
  under            = {{s.v.}{s.v.}},
  facsimile        = {{facsimile edition}{facs\adddotspace edn\adddot}},
  revised          = {{revised edition}{rev\adddotspace edn\adddot}},
  reviewof         = {{review of}{review of}},
  byperformer      = {{}{}},
  bydirector       = {{directed by}{dir\adddot}},
  byreader         = {{read by}{read by}},
  % Standard (NB. Oxford style is more sparing with 'by')
  introduction     = {{introduction}{introduction}},
  withintroduction = {{with an introduction by}{with an introduction by}},
  byeditor         = {{edited by}{ed\adddot}},
  bycompiler       = {{compiled by}{comp\adddot}},
  byredactor       = {{redacted by}{red\adddot}},
  byfounder        = {{founded by}{found\adddot}},
  bycontinuator    = {{continued by}{cont\adddot}},
  bycollaborator   = {{in collaboration with}{in collab\adddotspace with}}, % FIXME: unsure
  bytranslator     = {{translated \lbx@lfromlang\ by}{trans\adddot\ \lbx@sfromlang}},
  bycommentator    = {{commented by}{comm\adddot}},
  byannotator      = {{annotated by}{annot\adddot}},
  withcommentator  = {{with a commentary by}{with a commentary by}},
  withannotator    = {{with annotations by}{with annotations by}},
  withintroduction = {{with an introduction by}{with an introduction by}},
  withforeword     = {{with a foreword by}{with a foreword by}},
  withafterword    = {{with an afterword by}{with an afterword by}},
  byeditortr       = {{edited and translated \lbx@lfromlang\ by}%
                      {ed\adddotspace and trans\adddot\ \lbx@sfromlang}},
  byeditorco       = {{edited and commented by}%
                      {ed\adddotspace and comm\adddot}},
  byeditoran       = {{edited and annotated by}%
                      {ed\adddotspace and annot\adddot}},
  byeditorin       = {{edited, with an introduction, by}%
                      {ed.\adddotspace with introduction}},
  byeditorfo       = {{edited, with a foreword, by}%
                      {ed.\adddotspace with foreword}},
  byeditoraf       = {{edited, with an afterword, by}%
                      {ed.\adddotspace with afterword}},
  byeditortrco     = {{edited, translated \lbx@lfromlang\finalandcomma\ and commented by}%
                      {ed.,\addabbrvspace trans\adddot\ \lbx@sfromlang\finalandcomma\ and comm\adddot}},
  byeditortran     = {{edited, translated \lbx@lfromlang\finalandcomma\ and annotated by}%
                      {ed.,\addabbrvspace trans\adddot\ \lbx@sfromlang\finalandcomma\ and annot\adddot}},
  byeditortrin     = {{edited and translated \lbx@lfromlang, with an introduction, by}%
                      {ed\adddotspace and trans\adddot\ \lbx@sfromlang\ with introduction}},
  byeditortrfo     = {{edited and translated \lbx@lfromlang, with a foreword, by}%
                      {ed\adddotspace and trans\adddot\ \lbx@sfromlang\ with foreword}},
  byeditortraf     = {{edited and translated \lbx@lfromlang, with an afterword, by}%
                      {ed\adddotspace and trans\adddot\ \lbx@sfromlang\ with afterword}},
  byeditorcoin     = {{edited and commented, with an introduction, by}%
                      {ed\adddotspace and comm\adddotspace with introduction}},
  byeditorcofo     = {{edited and commented, with a foreword, by}%
                      {ed\adddotspace and comm\adddotspace with foreword}},
  byeditorcoaf     = {{edited and commented, with an afterword, by}%
                      {ed\adddotspace and comm\adddotspace with afterword}},
  byeditoranin     = {{edited and annotated, with an introduction, by}%
                      {ed\adddotspace and annot\adddotspace with introduction}},
  byeditoranfo     = {{edited and annotated, with a foreword, by}%
                      {ed\adddotspace and annot\adddotspace with foreword}},
  byeditoranaf     = {{edited and annotated, with an afterword, by}%
                      {ed\adddotspace and annot\adddotspace with afterword}},
  byeditortrcoin   = {{edited, translated \lbx@lfromlang\finalandcomma\ and commented, with an introduction, by}%
                      {ed.,\addabbrvspace trans\adddot\ \lbx@sfromlang\finalandcomma\ and comm\adddotspace with introduction}},
  byeditortrcofo   = {{edited, translated \lbx@lfromlang\finalandcomma\ and commented, with a foreword, by}%
                      {ed.,\addabbrvspace trans\adddot\ \lbx@sfromlang\finalandcomma\ and comm\adddotspace with foreword}},
  byeditortrcoaf   = {{edited, translated \lbx@lfromlang\finalandcomma\ and commented, with an afterword, by}%
                      {ed.,\addabbrvspace trans\adddot\ \lbx@sfromlang\finalandcomma\ and comm\adddotspace with afterword}},
  byeditortranin   = {{edited, translated \lbx@lfromlang\finalandcomma\ and annotated, with an introduction, by}%
                      {ed.,\addabbrvspace trans\adddot\ \lbx@sfromlang\finalandcomma\ and annot\adddotspace with introduction}},
  byeditortranfo   = {{edited, translated \lbx@lfromlang\finalandcomma\ and annotated, with a foreword, by}%
                      {ed.,\addabbrvspace trans\adddot\ \lbx@sfromlang\finalandcomma\ and annot\adddotspace with foreword}},
  byeditortranaf   = {{edited, translated \lbx@lfromlang\finalandcomma\ and annotated, with an afterword, by}%
                      {ed.,\addabbrvspace trans\adddot\ \lbx@sfromlang\finalandcomma\ and annot\adddotspace with afterword}},
  bytranslatorco   = {{translated \lbx@lfromlang\ and commented by}%
                      {trans\adddot\ \lbx@sfromlang\ and comm\adddot\ by}},
  bytranslatoran   = {{translated \lbx@lfromlang\ and annotated by}%
                      {trans\adddot\ \lbx@sfromlang\ and annot\adddot\ by}},
  bytranslatorin   = {{translated \lbx@lfromlang, with an introduction, by}%
                      {trans\adddot\ \lbx@sfromlang\ with introduction}},
  bytranslatorfo   = {{translated \lbx@lfromlang, with a foreword, by}%
                      {trans\adddot\ \lbx@sfromlang\ with foreword}},
  bytranslatoraf   = {{translated \lbx@lfromlang, with an afterword, by}%
                      {trans\adddot\ \lbx@sfromlang\ with afterword}},
  bytranslatorcoin = {{translated \lbx@lfromlang\ and commented, with an introduction, by}%
                      {trans\adddot\ \lbx@sfromlang\ and comm\adddotspace with introduction}},
  bytranslatorcofo = {{translated \lbx@lfromlang\ and commented, with a foreword, by}%
                      {trans\adddot\ \lbx@sfromlang\ and comm\adddotspace with foreword}},
  bytranslatorcoaf = {{translated \lbx@lfromlang\ and commented, with an afterword, by}%
                      {trans\adddot\ \lbx@sfromlang\ and comm\adddotspace with afterword}},
  bytranslatoranin = {{translated \lbx@lfromlang\ and annotated, with an introduction, by}%
                      {trans\adddot\ \lbx@sfromlang\ and annot\adddotspace with introduction}},
  bytranslatoranfo = {{translated \lbx@lfromlang\ and annotated, with a foreword, by}%
                      {trans\adddot\ \lbx@sfromlang\ and annot\adddotspace with foreword}},
  bytranslatoranaf = {{translated \lbx@lfromlang\ and annotated, with an afterword, by}%
                      {trans\adddot\ \lbx@sfromlang\ and annot\adddotspace with afterword}},
  idem             = {{idem}{id\adddot}},
  idemsm           = {{idem}{id\adddot}},
  idemsf           = {{eadem}{ead\adddot}},
  idemsn           = {{idem}{id\adddot}},
  idempm           = {{eidem}{eid\adddot}},
  idempf           = {{eaedem}{eaed\adddot}},
  idempn           = {{eadem}{ead\adddot}},
  idempp           = {{eidem}{eid\adddot}},
  newseries        = {{new series}{\mkbibacro{NS}}},
  oldseries        = {{old series}{\mkbibacro{OS}}},
  edition          = {{edition}{edn\adddot}},
  reprint          = {{reprint\nopunct}{repr\adddot\nopunct}},
  reviewof         = {{review of}{review of}},
  reprintfrom      = {{from}{from}},
  urlseen          = {{accessed}{accessed}},
}

\endinput
\end{writeverbatim}
\closeoutputstream{src}

%%%%%%%%%%%%%%%%%%%%%%%%%%%%%%%%%%%%%%%%%%%%%%%%%%%%%%%%%%%%%%%%%%%%%%%%
%%% DOCUMENTATION TOOLS
%%%%%%%%%%%%%%%%%%%%%%%%%%%%%%%%%%%%%%%%%%%%%%%%%%%%%%%%%%%%%%%%%%%%%%%%

\openoutputfile{oxref.bib}{src}
\openoutputfile{oxnotes-doc.tex}{oxnotes}
\openoutputfile{oxyear-doc.tex}{oxyear}

\usepackage{xstring}
% \bibexample adds a sample reference to the documentation.
% #1 = status (ok, hacked, bad)
% #2 = section from which drawn
% #3 = target reference text for oxnotes
% #4 = target reference text for oxyear
% #5 = bib record
% #6 = page reference
\makeatletter
\NewDocumentCommand{\bibexample}{d()d<>vv+vo}{%
  \addtostream{src}{#5}
  \addtostream{src}{}
  \StrBetween*[1,1]{#5}{\@charlb}{,}[\bibkey]
  \ifstrempty{#3}{}{%
    \docbibexample{oxnotes}{#3}{\bibkey}{#5}{%
      \IfValueT{#1}{(#1)}%
      \IfValueT{#2}{<#2>}%
      \IfValueT{#6}{[#6]}%
    }%
  }
  \ifstrempty{#4}{}{%
    \docbibexample{oxyear}{#4}{\bibkey}{#5}{%
      \IfValueT{#1}{(#1)}%
      \IfValueT{#2}{<#2>}%
      \IfValueT{#6}{[#6]}%
    }%
  }
}
\makeatother
% \docbibexample handles the writing of the documentation code.
% #1 = name of stream to which to send code
% #2 = target reference text
% #3 = bib key
% #4 = bib record
% #5 = extra options
\NewDocumentCommand{\docbibexample}{mmmmm}{%
  \addtostream{#1}{}
  \addtostream{#1}{\string\begin{bibexbox}#5{#3}}
  \addtostream{#1}{\space\space #2}
  \addtostream{#1}{\space\space\string\tcblower}
  \addtostream{#1}{\space\space\string\begin{verbatim}}
  \addtostream{#1}{#4\string\end{verbatim}}
  \addtostream{#1}{\string\end{bibexbox}}
}

\NewDocumentCommand{\todoc}{O{oxnotes,oxyear}+v}{%
  \renewcommand{\do}[1]{\addtostream{##1}{#2}}
  \docsvlist{#1}
}

%%%%%%%%%%%%%%%%%%%%%%%%%%%%%%%%%%%%%%%%%%%%%%%%%%%%%%%%%%%%%%%%%%%%%%%%
%%% STYLE DOCUMENTATION -- FRONT MATTER
%%%%%%%%%%%%%%%%%%%%%%%%%%%%%%%%%%%%%%%%%%%%%%%%%%%%%%%%%%%%%%%%%%%%%%%%

\todoc|\documentclass[extrafontsizes,11pt,a4paper,oneside]{memoir}|

\todoc[oxnotes]|\PassOptionsToPackage{bibstyle=oxnotes,citestyle=verbose}{biblatex}|
\todoc[oxyear]|\PassOptionsToPackage{bibstyle=oxyear,citestyle=authoryear}{biblatex}|

\todoc|\input{oxref-preamble}|

\todoc[oxnotes]|
\title{Oxnotes style for Biblatex}|
\todoc[oxyear]|
\title{Oxyear style for Biblatex}|

\todoc|\author{Alex Ball}

\begin{document}
\thispagestyle{empty}
\begin{adjustwidth}{.2\textwidth}{0pt}
  \sffamily\setlength{\parindent}{0pt}%
  \LARGE\textsc{oxref bundle}
  
  \vspace{\stretch{1}}
  \LARGE\thetitle
  
  \bigskip
  \Large\theauthor
  
  \bigskip
  \Large\ifx\undefined\thedate\today\else\thedate\fi
\end{adjustwidth}

\vspace{\stretch{3}}
\noindent
\hspace*{.1\textwidth}\raisebox{0pt}[0pt][0pt]{\rule{\normalrulethickness}{\textheight}}

\newpage
\tableofcontents*
|

%%%%%%%%%%%%%%%%%%%%%%%%%%%%%%%%%%%%%%%%%%%%%%%%%%%%%%%%%%%%%%%%%%%%%%%%
%%% STYLE DOCUMENTATION -- SPECIFICATION
%%%%%%%%%%%%%%%%%%%%%%%%%%%%%%%%%%%%%%%%%%%%%%%%%%%%%%%%%%%%%%%%%%%%%%%%

\todoc|
\chapter{Introduction}

\section{Loading the style}

\section{How to use this document}

OGS = \emph{Oxford Guide to Style} (2002) ed.~R.~M.\ Ritter.

NHR = \emph{New Hart's Rules: The Oxford Style Guide} (2nd ed., 2014) ed.~R.~M.\ Ritter.

\section{To do}

The Oxref family of Biblatex styles does not yet support the following entry types
(or at least, they have not yet been styled appropriately):

\begin{multicols}{2}
  \begin{itemize}\firmlist
    \item booklet
    \item manual
    \item misc
    \item patent
    \item periodical
    \item suppperiodical
  \end{itemize}
\end{multicols}

\chapter{Books, collections and reference works}\label{sec:book}
\chapterprecis{book, mvbook, collection, mvcollection, reference, mvreference}

\section{Books}

Biblatex defines \code{book} as a monograph, written either by a single author, or by several authors who have joint responsibility for the entire work. An editor in this context has a modest role selecting or annotating the content, and therefore in Oxford style is given after the title even if the author is not given.

\subsection{Basic form}
|

\todoc[oxnotes]|
\spec{Author, \emph{Title} (Location: Publisher, Year).}
|
\todoc[oxyear]|
\spec{Author (Date), \emph{Title} (Location: Publisher).}
|

\bibexample<OGS \S15.2.1>
|George Eliot, \emph{Middlemarch} (New York: W. W. Norton, 1977).|%
||%
|@book{eliot1977m,
  author = {George Eliot},
  title = {Middlemarch},
  location = {New York},
  publisher = {W. W. Norton},
  date = {1977}
}|

\bibexample<OGS \S15.2.2>
|Antonia Tissoni Benvenuti, \emph{L'\emph{Orfeo} del Poliziano} (Padua: Editrice Antenore, 1986).|%
||%
|@book{benvenuti1986op,
  author = {Antonia Tissoni Benvenuti},
  title = {L'\emph{Orfeo} del Poliziano},
  location = {Padua},
  publisher = {Editrice Antenore},
  date = {1986}
}|

\bibexample<OGS \S15.2.2>
|Roderick O'Conor, \emph{A Sentimental Journal through \enquote{Finnegan's Wake}, with a Map of the Liffey} (Dublin: HCE Press, 1977).|%
||%
|@book{oconor1977sjf,
  author = {Roderick O'Conor},
  title = {A Sentimental Journal through \enquote{Finnegan's Wake}, with a Map of the {Liffey}},
  location = {Dublin},
  publisher = {HCE Press},
  date = {1977}
}|

\bibexample<NHR \S18.5>
||%
|Lyons, J. (1981a), \emph{Language and Linguistics: An Introduction} (Cambridge: Cambridge University Press).|%
|@book{lyons1981lli,
  author = {John Lyons},
  date = {1981},
  title = {Language and Linguistics},
  subtitle = {An Introduction},
  location = {Cambridge},
  publisher = {Cambridge University Press}
}|

\bibexample<NHR \S18.5>
||%
|Lyons, J. (1981b), \emph{Language, Meaning and Context} (London: Fontana Paperbacks).|%
|@book{lyons1981lmc,
  author = {John Lyons},
  date = {1981},
  title = {Language, Meaning and Context},
  location = {London},
  publisher = {Fontana Paperbacks}
}|


\todoc|
\subsection{No publisher}
|

\todoc[oxnotes]|
\spec{Author, \emph{Title} (Location, Year).}
|

\bibexample<OGS \S15.2.1>
|{[James Balfour]}, \emph{Philosophical Essays} (Edinburgh, 1768).|%
||%
|@book{balfour1768pe,
  author = {James Balfour},
  authortype = {inferred},
  title = {Philosophical Essays},
  location = {Edinburgh},
  date = {1768}
}|

\todoc|
\subsection{Edited book}
|

\todoc[oxnotes]|
\spec{Author, \emph{Title}, \lit{ed.} Editor(s) (Location: Publisher, Year).}
|

\bibexample<OGS \S15.2.1>
|Pseudo-Boethius, \emph{De disciplina scolarium: Édition critique, introduction et notes}, ed. Olga Weijers (Leiden, 1976).|%
||%
|@book{boethius1976dds,
  author = {Pseudo-Boethius},
  title = {De disciplina scolarium},
  subtitle = {Édition critique, introduction et notes},
  editor = {Olga Weijers},
  location = {Leiden},
  date = {1976}
}|

\todoc[oxnotes]|
\spec{\emph{Title}, \lit{ed.} Editor(s) (Location: Publisher, Year).}
|

\bibexample<OGS \S15.2.1>
|\emph{Distichia Catonis}, ed. Marcus Boas and Henricus Johannes Botschuyver (Amsterdam, 1952).|%
||%
|@book{boas.botschuyver1952dc,
  title = {Distichia Catonis},
  editor = {Marcus Boas and Henricus Johannes Botschuyver},
  location = {Amsterdam},
  date = {1952}
}|

\todoc|
\section{Collections}

Biblatex defines \code{collection} as a book made up of multiple self-contained contributions from distinct authors. There is no overall author: use the regular \code{book} entry type for collections of a single author’s work. The editor in this case has a more active role and therefore comes before the title.

\spec{Editor (\lit{ed.}), \emph{Title} (Location: Publisher, Year).}
|

\bibexample<OGS \S15.2.1>
|Rosemary Stewart et al. (eds.), \emph{Managing in Britain} (London, 1994).|%
||%
|@collection{stewart.etal1994mb,
  editor = {Rosemary Stewart and Jean-Louis Barsoux and Alfred Kieser and Hans-Dieter Ganter and Peter Walgenbach},
  title = {Managing in {Britain}},
  location = {London},
  date = {1994}
}|

\todoc|
\section{Reference works}

Biblatex uses \code{reference} for encyclopaedias and dictionaries, which are typically made up of many small contributions by distinct authors and compiled by an editorial board. From a database perspective, reference works are like collections in that they have no single author (a reference work written by a single author should use the \code{book} entry type instead). Unlike standard Biblatex, though, Oxford style considers the titles of reference works to be more important and memorable than those of the editor, and so lists the title first.

\spec{\emph{Title}, \lit{ed.} Editor(s) (Location: Publisher, Year).}
|

\bibexample<OGS \S15.2.1>
|\emph{Theophrastus of Eresus: Sources for his Life, Writings, Thought, and Influence}, ed. William W. Fortenbaugh et al., 2 vols. (Philosophia Antiqua, 54; Leiden, 1991).|%
||%
|@mvreference{fortenbaugh.etal1991tes,
  title = {{Theophrastus} of {Eresus}},
  subtitle = {Sources for his Life, Writings, Thought, and Influence},
  editor = {William W. Fortenbaugh and Pamela M Huby and Robert W. Sharples and Dimitri Gutas and others},
  volumes = {2},
  series = {Philosophia Antiqua},
  number = {54},
  location = {Leiden},
  date = {1991}
}|

\todoc|
\section{Multi-volume works}

Biblatex provides additional entry types for multi-volume works: \code{mvbook}, \code{mvcollection} and \code{mvreference} respectively. These can be used to cite all the volumes at once, or just one volume from the set.

\subsection{All volumes}

Where all volumes were published consistently, the following form is used:

\spec{Author, \emph{Title}, n \lit{vols.} (Location: Publisher, Year), VolNo. Pages.}
|

\bibexample<OGS \S15.2.6>
|Edmond Vander Straeten, \emph{La Musique aux Pays-Bas avant le XIX\textsuperscript{e} siècle}, 8 vols. (Brussels, 1867--88), ii. 367--8.|
||%
|@mvbook{straeten1867lmp,
    author = {Edmond Vander Straeten},
    title = {La Musique aux Pays-Bas avant le XIX\textsuperscript{e} siècle},
    volumes = {8},
    location = {Brussels},
    date = {1867/1888}
}|[ii.~367--8]

\todoc|
NB. A more portable way of expressing the above citation would be\\
\lstinline!\fullcite[\nopp ii.~\mkcomprange{367--368}]{straeten1867lmp}!.

Where the publisher changed between volumes, the following form is used:

\spec{Author, \emph{Title}, n \lit{vols.}, VolNo (Location: Publisher, Year), VolNo (Location: Publisher, Year).}
|

\bibexample(hacked)<OGS \S15.2.6>
|Heinrich Ritter, \emph{The History of Ancient Philosophy}, trans. Alexander J. W. Morrison, 4~vols., i--iii (Oxford: Talboys, 1838--9), iv (London: Bohn, 1846).|%
||%
|@mvbook{ritter1838hap,
  author = {Heinrich Ritter},
  title = {The History of Ancient Philosophy},
  translator = {Alexander J. W. Morrison},
  volumes = {4},
  volume = {1--iii},
  location = {Oxford},
  publisher = {Talboys},
  date = {1838/1839},
  pages = {iv},
  pubstate = {London: Bohn, 1846}
}|

\todoc|
NB. In the \emph{Oxford Style Manual}, the translator in the above reference is in parentheses,
but as this pattern does not occur in any other it may be an error or stylistic variation.

\subsection{One volume from several}

Where the volumes are merely numbered, the following form is used
\aside{Use \code{title} but not \code{maintitle}}:

\spec{Author, \emph{Title}, VolNo (Location: Publisher, Year), Pages.}
|

\bibexample<OGS \S15.2.6>
|Edmond Vander Straeten, \emph{La Musique aux Pays-Bas avant le XIX\textsuperscript{e} siècle}, ii (Brussels, 1867--88), 367--8.|%
||%
|@mvbook{straeten1867lmp.ii,
  author = {Edmond Vander Straeten},
  title = {La Musique aux Pays-Bas avant le XIX\textsuperscript{e} siècle},
  volume = {2},
  location = {Brussels},
  date = {1867/1888}
}|[367-368]

\todoc|
Where the volumes each have their own (sub)title,
and they were all published at once (more or less),
the following form is used \aside{Use both \code{title} and \code{maintitle}}:

\spec{Author, \emph{Title}, VolNo: \emph{VolTitle} (Location: Publisher, Year), Pages.}
|

\bibexample<OGS \S15.2.2>
|Thomas A. Brady, Jr., Heiko A. Oberman, and James D. Tracy (eds.), \emph{Handbook of European History, 1400–1600: Late Middle Ages, Renaissance and Reformation}, i: \emph{Structures and Assertions} (Leiden: E. J. Brill, 1994).|%
||%
|@mvcollection{brady.etal1994hehsa,
  editor = {Brady, Jr., Thomas A. and Heiko A. Oberman and James D. Tracy},
  maintitle = {Handbook of European History, 1400–1600},
  mainsubtitle = {{Late} {Middle} {Ages,} {Renaissance} and {Reformation}},
  volume = {1},
  title = {Structures and Assertions},
  location = {Leiden},
  publisher = {E. J. Brill},
  date = {1994}
}|

\bibexample<OGS \S15.2.6>
|A. W. Ward and A. E. Waller (eds.), \emph{The Cambridge History of English Literature}, xii: \emph{The Nineteenth Century} (Cambridge: CUP, 1932), 42--56.|%
||%
|@mvcollection{ward.waller1932che,
  editor = {A. W. Ward and A. E. Waller},
  maintitle = {The Cambridge History of English Literature},
  volume = {12},
  title = {The Nineteenth Century},
  location = {Cambridge},
  publisher = {CUP},
  date = {1932}
}|[42-56]

\todoc|
Where the volumes each have their own (sub)title,
and there are many volumes spanning years and possibly publishers,
the following form is used \aside{Use entry subtype \code{sequence}.}:

\spec{Author, \emph{VolTitle}, (\lit{vol.} VolNo \lit{of} Title) (Location: Publisher, Year), Pages.}
|

\bibexample<OGS \S15.2.6>
|David Hackett Fischer, \emph{Albion’s Seed: Four British Folkways in America}, (vol.\ i of \emph{America: A Cultural History}) (New York: Oxford University Press, 1989).|%
||%
|@book{fischer1989asf,
  entrysubtype = {sequence},
  author = {David Hackett Fischer},
  title = {Albion’s Seed},
  subtitle = {Four British Folkways in America},
  volume = {1},
  maintitle = {America},
  mainsubtitle = {A Cultural History},
  location = {New York},
  publisher = {Oxford University Press},
  date = {1989}
}|

\todoc|
\section{Common variations}

\subsection{Additions and translations}

Where works have significant introductions, forewords, afterwords, illustrations, etc.\ this may be noted as follows:

\spec{Author, \emph{Title}, \lit{with an} Addition \lit{by} Contributor (Location: Publisher, Year).}
|

\bibexample<OGS \S15.2.1>
|Mark Twain, \emph{A Connecticut Yankee at King Arthur's Court}, with an introduction by Justin Kaplan (Harmondsworth: Penguin, 1971).|%
||%
|@book{twain1971cyk,
  author = {Mark Twain},
  title = {A {Connecticut} Yankee at {King} {Arthur's} Court},
  introduction = {Justin Kaplan},
  location = {Harmondsworth},
  publisher = {Penguin},
  date = {1971}
}|

\todoc|
Translators are credited as follows:

\spec{Author, \emph{Title}, \lit{trans.} Translator(s) (Location: Publisher, Year).}
|

\bibexample<OGS \S15.2.15>
|Bernhard Bischoff, \emph{Latin Palaeography: Antiquity and the Middle Ages}, trans. Dáibhi Ó Cróinín and David Ganz (Cambridge, 1990).|%
||%
|@book{bischoff1990lpa,
  author = {Bernhard Bischoff},
  title = {Latin Palaeography},
  subtitle = {Antiquity and the Middle Ages},
  translator = {Dáibhi Ó Cróinín and David Ganz},
  location = {Cambridge},
  date = {1990}
}|

\todoc|
\spec{Author, \emph{Title}, \lit{trans.\ with} Addition Translator(s) (Location: Publisher, Year).}
|

\bibexample<OGS \S15.2.15>
|Joanat Martorell, \emph{Tirant lo Blanc}, trans. with foreword David H. Rosenthal (London, 1984).|%
||%
|@book{martorell1984tlb,
  author = {Joanat Martorell},
  title = {Tirant lo Blanc},
  translator = {David H. Rosenthal},
  foreword = {David H. Rosenthal},
  location = {London},
  date = {1984}
}|

\todoc|
\subsection{Revision}

Where a work has been lightly revised, the following form is used:

\spec{Author, \emph{Title}, \lit{rev.} Reviser(s) (Location: Publisher, Year).}

Where a work has been extensively revised, the following form is used:

\spec{Author, \lit{rev.} Reviser, \emph{Title}, (Location: Publisher, Year).}

\subsection{Later edition}

Where the editor is common to both, the following form is used \aside{Use entry type \code{book}}:

\spec{Author, \emph{Title} (nth \lit{edn.}, Location: Publisher, Year).}
|

\bibexample<OGS \S15.2.3>
|J. H. Baker, \emph{An Introduction to English Legal History} (3rd edn., 1990), 419--21.|%
||%
|@book{baker1990iel,
  author = {J. H. Baker},
  title = {An Introduction to {English} Legal History},
  edition = {3},
  date = {1990}
}|[419-421]

\todoc|
Where the editor has changed, the following form is used \aside{Use entry type \code{reference}}:

\spec{\emph{Title}, nth \lit{edn.}, \lit{ed.} Editor (Location: Publisher, Year).}
|

\bibexample<OGS \S15.2.3>
|\emph{The Oxford Dictionary of Quotations}, 5th edn., ed. Elizabeth Knowles (Oxford, 1999).|%
||%
|@reference{knowles1999odq,
  title = {The {Oxford} Dictionary of Quotations},
  edition = {5},
  editor = {Elizabeth Knowles},
  publisher = {Oxford},
  date = {1999}
}|

\todoc|
\subsection{First and later edition}

Where the publisher is common to both, the following form is used \aside{Use the \code{orig} fields}:

\spec{Author, \emph{Title} (Year; nth \lit{edn.}, Location: Publisher, Year).}
|

\bibexample<OGS \S15.2.3>
|J. D. Denniston, \emph{The Greek Particles} (1934; 2nd edn., Oxford, 1954).|%
||%
|@book{denniston1934gp,
  author = {J. D. Denniston},
  title = {The Greek Particles},
  origdate = {1934},
  edition = {2},
  location = {Oxford},
  date = {1954}
}|

\bibexample<OGS \S15.2.1>
|{[John Gibbon]}, \emph{Day-Fatality: Or Some Observations on Days Lucky and Unlucky}, (London, 1678; rev.~edn., 1686).|%
||%
|@book{gibbon1686dfs,
  author = {John Gibbon},
  authortype = {inferred},
  title = {Day-Fatality},
  subtitle = {Or Some Observations on Days Lucky and Unlucky},
  edition = {revised},
  origlocation = {London},
  origdate = {1678},
  date = {1686}
}|
  
\todoc|
In the \emph{Oxford Style Manual}, there is no comma after ‘rev. edn.’ in the above reference; this may be an error.

Used where the publisher has changed, the following form is used \aside{Use \code{pubstate} for the original publication date}:

\spec{Author, \emph{Title} (nth \lit{edn.}, Location: Publisher, Year) (\lit{first pub.} Year).}
|

\bibexample<OGS \S15.2.3>
|Bruno Bettelheim, \emph{The Uses of Enchantment: The Meaning and Importance of Fairy Tales} (Harmondsworth: Penguin Books, 1988) (first pub. 1976).|%
||%
|@book{bettelheim1976uem,
  author = {Bruno Bettelheim},
  title = {The Uses of Enchantment},
  subtitle = {The Meaning and Importance of Fairy Tales},
  location = {Harmondsworth},
  publisher = {Penguin Books},
  date = {1988},
  pubstate = {first pub\adddotspace 1976}
}|
  
\todoc|
\subsection{Reprint}

\aside{Use localization key \code{reprint} as the edition.}\par 
\spec{Author, \emph{Title} (Location: Publisher, Year; \lit{repr.} Location: Publisher, Year).}
|

\bibexample<OGS \S15.2.4>
|C. Adam and D. Tannery (eds.), \emph{Œuvres de Descartes} (Paris: Cerf, 1897--1913; repr. Paris: J. Vrin, CNRS, 1964--76).|%
||%
|@collection{adam.tannery1897odd,
  editor = {C. Adam and D. Tannery},
  title = {Œuvres de Descartes},
  origlocation = {Paris},
  origpublisher = {Cerf},
  origdate = {1897/1913},
  edition = {reprint},
  location = {Paris},
  publisher = {J. Vrin, CNRS},
  date = {1964/1976}
}|

\todoc|
\aside{Use localization key \code{facsimile} as the edition.}\par 
\spec{Author, \emph{Title} (Location: Publisher, Year; \lit{facs. edn.}, Location: Publisher, Year).}
|

\bibexample<OGS \S15.2.4>
|E. Allen, \emph{A Knack to Know a Knave} (London, 1594; facs. edn., Oxford: Malone Society Reprints, 1963).|%
||%
|@book{allen1594kkk,
  author = {E. Allen},
  title = {A Knack to Know a Knave},
  origlocation = {London},
  origdate = {1594},
  edition = {facsimile},
  location = {Oxford},
  publisher = {Malone Society Reprints},
  date = {1963}
}|

\todoc|
\subsection{Multiple publishers}

\spec{Author, \emph{Title} (Location: Publisher, Year; Location: Publisher, Year).}
|

\bibexample(hacked)<OGS \S15.2.5>
|L. A. Holford-Strevens, \emph{Aulus Gellius} (London: Duckworth, 1988; Chapel Hill: University of North Carolina Press, 1989).|%
||%
|@book{holfordstrevens1988ag,
  author = {L. A. Holford-Strevens},
  title = {Aulus Gellius},
  origlocation = {London},
  origpublisher = {Duckworth},
  origdate = {1988},
  location = {Chapel Hill},
  publisher = {University of North Carolina Press},
  date = {1989}
}|

\todoc|
\subsection{Work from a series}

\spec{Author, \emph{Title} (Series; Location: Publisher, Year).}

\spec{Author, \emph{Title} (Series, Number; Location: Publisher, Year).}
|

\bibexample<OGS \S15.2.7>
|Johannes de Garlandia, \emph{De mensurabili musica}, ed. Erich Reimer, 2 vols. (Beihefte zum Archiv für Musikwissenschaft, 10--11; Wiesbaden, 1972).|%
||%
|@mvbook{garlandia1972dmm,
  author = {Johannes de Garlandia},
  title = {De mensurabili musica},
  editor = {Erich Reimer},
  volumes = {2},
  series = {Beihefte zum Archiv für Musikwissenschaft},
  number = {10--11},
  publisher = {Wiesbaden},
  date = {1972}
}|

\todoc|
\spec{Author, \emph{Title} (nth \lit{ser.}, Year).}

\chapter{Reports}\label{sec:report}
\chapterprecis{report}

Oxford style does not have any special rules for referencing reports. Its provision for books, however, is sufficient for most purposes.

\spec{Author, \emph{Title} (Location: Institution, Year).}
|

\bibexample<OGS \S15.2.13>
|Government of Botswana, \emph{A Human Drought Relief Programme for Botswana} (Gabarone: Ministry of Local Government and Lands, 1980).|%
||%
|@report{botswana1980hdr,
  author = {{Government of Botswana}},
  title = {A Human Drought Relief Programme for Botswana},
  location = {Gabarone},
  institution = {Ministry of Local Government and Lands},
  date = {1980}
}|
  
\todoc|
If the report does not specify an author, the institution moves to the author position:

\spec{Author, \emph{Title} (Location, Year).}
|

\bibexample<OGS \S15.2.13>
|Penal Affairs Consortium, \emph{An Unsuitable Place for Treatment: Diverting Mentally Disordered Offenders from Custody} (London, 1988).|%
||%
|@report{pac1988upt,
  author = {{Penal Affairs Consortium}},
  title = {An Unsuitable Place for Treatment},
  subtitle = {Diverting Mentally Disordered Offenders from Custody},
  location = {London},
  date = {1988}
}|
  

\todoc|
\textsc{Extension} – Reports sometimes have a type, which can be treated like a series especially if a number is also provided:

\spec{Author, \emph{Title} (Type, Number; Location: Institution, Year).}

\chapter{Proceedings}\label{sec:proceedings}
\chapterprecis{proceedings, mvproceedings}

\section{Single volume proceedings}

The \code{proceedings} entry type is intended for a set of conference papers that have been collected together into a single volume.

\spec{\emph{Title}, \lit{ed.} Editor (Location: Publisher, Year).}
|

\bibexample<OGS \S15.2.12>
|\emph{Proceedings of the XIV International Congress of Papyrologists: Oxford, 24--31 July 1974} (London, 1975).|%
||%
|@proceedings{icp1975pic,
  title = {Proceedings of the {XIV} {International} {Congress} of {Papyrologists}},
  subtitle = {{Oxford}, 24--31 {July} 1974},
  location = {London},
  date = {1975}
}|

\todoc|
\textsc{Extension} – Sometimes the title of the proceedings does not relate to the conference; in which case the following form might be used:

\spec{\emph{Title}, EventTitle: Venue, ConfDate, \lit{ed.} Editor (Series, Number; Location: Publisher, Year).}
|

\bibexample
|(not in book)|%
||%
|@proceedings{ecdl2009,
  editor = {Maristella Agosti and José Borbinha and Sarantos Kapidakis and Christos Papatheodorou and Giannis Tsakonas},
  title = {Research and Advanced Technology for Digital Libraries},
  eventtitle = {13th European Conference, ECDL 2009},
  venue = {Corfu, Greece},
  eventdate = {2009-09-27/2009-10-02},
  series = {Lecture Notes in Computer Science},
  number = {5714},
  location = {Berlin},
  publisher = {Springer},
  date = {2009}
}|

\todoc|
\section{Multi-volume proceedings}

\textsc{Extension} – Biblatex provides \code{mvproceedings} for multi-volume proceedings.
The examples here are a combination of Oxford style for proceedings and multi-volume books.

\subsection{All volumes}

\spec{\emph{Title}, EventTitle: Venue, ConfDate, \lit{ed.} Editor, n \lit{vols.} (Series, Number; Location: Publisher, Year).}
|

\bibexample
|(not in book)|%
||%
|@proceedings{iced2009,
  title = {Proceedings of the 17th International Conference on Engineering Design (ICED ’09)},
  volumes = {10},
  editor = {Margareta Norell Bergendahl and Martin Grimheden and Larry Leifer},
  venue = {Stanford, CA},
  eventdate = {2009-08-24/2009-08-27},
  location = {Glasgow},
  publisher = {Design Society},
  date = {2009},
  isbn = {978-1-904670-12-4}
}|

\todoc|
\subsection{One volume}

\spec{\emph{Title}, VolNo: \emph{VolTitle}, EventTitle: Venue, ConfDate, \lit{ed.} Editor (Series, Number; Location: Publisher, Year).}
|

\bibexample
|(not in book)|%
||%
|@proceedings{iced2009dik,
  maintitle = {Proceedings of the 17th International Conference on Engineering Design (ICED ’09)},
  title = {Design Information and Knowledge},
  volume = {8},
  editor = {Margareta Norell Bergendahl and Martin Grimheden and Larry Leifer},
  venue = {Stanford, CA},
  eventdate = {2009-08-24/2009-08-27},
  location = {Glasgow},
  publisher = {Design Society},
  date = {2009},
  isbn = {978-1-904670-12-4}
}|

\todoc|
\chapter{Works in books, collections and reference works}\label{sec:inx}
\chapterprecis{inbook, incollection, inreference, bookinbook}

\section{Works in books}

The \code{inbook} entry type is intended for books that consist of several self-contained works by the same author, for citing one of these works. 

\begin{description}
  \item[One from a collection of single author's works] \aside{\code{author} $=$ \code{bookauthor} $\neq$ \code{editor}}
  \par Author, \enquote{Title}, \lit{in id.}, \emph{BookTitle}, \lit{ed.} Editor(s) (Location: Publisher, Year), Pages.
  \\\egcite{owen1986pi}
  \par Author, \enquote{Title}, \lit{in} \emph{BookTitle}, \lit{ed.} Editor(s) (Location: Publisher, Year), Pages.
\end{description}

\section{Works in collections}

The \code{incollection} entry type is intended for citing one of a collection of self-contained works by different authors.

\begin{description}
  \item[Work in mixed collection] \aside{\code{author} $\neq$ \code{editor}, no \code{bookauthor}}
  \par Author, \enquote{Title}, \lit{in} Editors (\lit{eds.}), \emph{BookTitle} (Location: Publisher, Year), Pages.
  \\\egcite{shearman1993vsf}
  
  \item[Work by the editor in mixed collection] Replace the editor's name with \enquote{id.}\slash \enquote{ead.}\slash \enquote{eid.}\slash \enquote{eaed.} \aside{\code{author} $=$ \code{editor}, no \code{bookauthor}}
  \par Author, \enquote{Title}, \lit{in id.} (\lit{ed.}), \emph{BookTitle} (Location: Publisher, Year), Pages.
  \\\egcite{todd1974dhp}
  
\end{description}

\section{Anthologies of independently published works}

Biblatex provides the \code{bookinbook} entry type for citing a part of an anthology that has previously been published as a book in its own right. Oxford style provides two different mechanisms for this, depending on circumstance.

\begin{description}
  \item[Both versions equally accessible] \aside{If using Biber, put the key of the original work in \code{related}, and in \code{relatedtype} use the keyword \code{reprintfrom}. Otherwise, use \code{addendum} to give \enquote{From} and the reference to the original, but omit \code{author} and \code{title} from the entry for the original work.}\par
  Later Reference. From Earlier Reference.
  \\\egcite{owen1986pi}
  \item[Later version more accessible] \aside{Use \code{pubstate} to give the date of original publication.}\par
  \par Author, \emph{Title} (Location: Publisher, Year) (\lit{originally pub.} Year).
  \\\egcite{frege1892osr}
\end{description}

\chapter{Papers in proceedings}\label{sec:inproceedings}
\chapterprecis{inproceedings}

The \code{inproceedings} entry type is intended for a conference paper published as part of a proceedings volume. For papers that have not been collected into a proceedings volume, or presentation slide sets, use the \code{unpublished} entry type instead.

\begin{description}
  \item[Published conference paper] \aside{Extension}\par
  Author, \enquote{Title}, \lit{in} \emph{BookTitle}, EventTitle: Venue, ConfDate, \lit{ed.} Editor (Series, Number; Location: Publisher, Year).
  \\\egcite{tonkin.strelnikov2009iem}\par
  Author, \enquote{Title}, \lit{in} \emph{BookTitle}, EventTitle: Venue, ConfDate, \lit{ed.} Editor, n \lit{vols.} (Series, Number; Location: Publisher, Year), VolNo. Pages.
  \\\egcite{ding.etal2009sfc}
  
  \item[Published orations, addresses, lectures, and speeches] Type (of oration), below, is something like \enquote{Inaugural Lecture} or \enquote{Romanes Lecture}.\par Author, \emph{Title}, Type, Venue, Date (Location: Publisher, Year).
  \\\egcite{gombrich1957as}
\end{description}

\chapter{Articles in periodicals}\label{sec:article}
\chapterprecis{article}

\begin{description}
  \item[Article in issue with no volume or number]~
  \par Author, \enquote{Title}, \emph{JournalTitle}, Day Month Year, Pages.
  \\\egcite{boyce1957pgi}
  \\\egcite{lee1995ehf}
  \\\egcite{bw1984wen}
  
  \item[Article in issue with volume/number]~
  \par Author, \enquote{Title}, \emph{JournalTitle}, Vol/Number (Year), Pages.
  \\\egcite{vaucouleurs1975nmn}
  \\\egcite{inalcik1992csm}
  \\\egcite{goldblatt1973dmm}
  
  \item[Periodicals with series]~
  \par Author, \enquote{Title}, \emph{JournalTitle}, nth \lit{ser}. Vol/Number (Year), Pages.
  \par Author, \enquote{Title}, \emph{JournalTitle}, \textsc{ns} Vol/Number (Year), Pages.
  \\\egcite{barnes1971has}
  
  \item[Work published as an issue] Taking the monograph edition as primary \aside{For this, abuse the appropriate monograph entry type: \code{book} or \code{collection}.}
  \par Author, \emph{Title} = \emph{JournalTitle}, Vol/Number (Location: Publisher, Year), pages.
  \\\egcite{bec1976isc}
  
  \item[Work published as an issue] Taking the serial edition as primary \aside{Use \code{issuetitle} instead of \code{title}.}
  \par Author, \emph{Title} = \emph{JournalTitle}, Vol/Number (Year).
  \\\egcite{trisoglio1973gnq}
  
  \item[Article in issue that is a supplement to another issue] \aside{Use \code{note} field to write ‘Supplement to’ and a reference to a \code{periodical}.}
  \par Author, \enquote{Title}, \emph{JournalTitle}, Vol/Number (Year); \lit{Supplement to} \emph{MainJournalTitle} Vol/Number (Year), Pages.
  \\\egcite{zhentao.etal1989ars}
  
  \item[Review] \aside{If using Biber, use \code{related} to give the key of the reviewed item, and in \code{relatedtype} give the keyword \code{reviewof}. Otherwise, set the option \code{reviewed} with the value of the key of reviewed item; the reviewed item is given the category \code{reviewed} so it can be excluded from the bibliography if desired.}
  \par Author, \enquote{Title}, \lit{review of} ReviewedWork, \lit{in} \emph{JournalTitle}, Vol/Number (Year), Pages.
  \\\egcite{dean1995rpb}
  \\\egcite{jocelyn1989pav}
  %\item[Alternative styles]~
  %\par Author, \enquote{Title}, \emph{JournalTitle}, Vol: Number (Year), Pages.
  %\par Author (Year), \enquote{Title}, \emph{JournalTitle}, Vol/Number: Pages.
  %\par Author (Year), \enquote{Title}, \emph{JournalTitle}, Vol (Number): Pages.
\end{description}

\section{Notes}

\begin{itemize}
  \item Oxford style permits volumes and part numbers to be written like 23: 2 instead of 23/2. \aside{To do, as a package option.}
  
  \item If the last\slash only word of the JournalTitle is abbreviated (indicated by point or capital letter), it does not need a comma after it.
  \\\egcite{lindars1965eir}
\end{itemize}

\chapter{Theses and dissertations}\label{sec:thesis}
\chapterprecis{thesis}

\begin{description}
  \item[Preferred]~\par Author, \enquote{Title}, Type (Institution, Year).
  \\\egcite{blackburn1970tlp}
  
  \item[Alternative] \aside{To do, as package option.}\par Author, \enquote{Title}, (Type, Institution, Year).
\end{description}

%\subsection{Notes}
%\begin{itemize}
%\item \enquote{Dissertation} should be abbreviated to \enquote{diss.} if the degree/level is given before it, or \enquote{Diss.} otherwise (i.e. German or Dutch dissertations).
%\end{itemize}

\chapter{Unpublished works}\label{sec:unpublished}
\chapterprecis{unpublished}

\begin{description}
  \item[Accepted journal article, pre-publication]~\par
  Author, \enquote{Title} (\lit{to be published in} \emph{JournalTitle}, Year).
  
  \item[Unpublished conference paper]~\par
  Author, \enquote{Title}, \lit{paper given at the} ConfTitle, Venue, EventDate.
  \\\egcite{holfordstrevens1999hlm}
  
  \item[Unpublished orations, addresses, lectures, and speeches]~\par Author, \enquote{Title}, Description.
\end{description}

%\subsection{Notes}
%\begin{itemize}
%\item Description (of oration) is capitalized if derived from a place or event (Sermon on the Mount; Gettysburg Address) but not if derived from a time (Easter sermon; speech in the House of Lords, 18 March 1751).
%\end{itemize}

\chapter{Poems}\label{sec:poem}

\begin{description}
  \item[Long poems] Used for poems divided into cantos
  \par Author, \emph{Title}, Book. Canto. Stanza. Line.
  \par Author, \emph{Title}, \lit{bk.} Book, \lit{canto} Canto, \lit{stanza} Stanza, \lit{line} Line.
  
  \item[Short poems]~\par
  \par Author, \enquote{Title}, in\dots
\end{description}

\section{Notes}

\begin{itemize}
  \item When using numerals, Book uses full capital roman, Canto uses small capital roman, Stanza uses lowercase roman, Line uses arabic: II. \textsc{iv}. xxxi. 7.
  
  \item When using words (i.e. when citing whole stanzas or greater), use small capital roman numerals: bk. ii, canto iv, stanza xxxi.
\end{itemize}

\chapter{Plays}\label{sec:play}

\begin{description}
  \item[Basic form]~
  \par Author, \emph{Title}, Act. Scene. Line.
  \par Author, \emph{Title}, \lit{Act} Act, \lit{Scene} Scene, \lit{line} Line.
\end{description}

\section{Notes}

\begin{itemize}
  \item When using numerals only in a parenthetical citation, Act uses small capital roman numerals. Otherwise, Act uses full capital roman numerals.
  
  \item Scene always uses lowercase roman numerals and Line always uses arabic.
  
  \item The words \enquote{Act} and \enquote{Scene} are capitalized, whereas \enquote{line} is not: Act IV, Scene iii, line 10.
\end{itemize}

\chapter{Manuscripts}\label{sec:ms}

\begin{description}
  \item[Form]~
  \item Author, Title, City, Archive, Collection Series, Volume.
\end{description}

\section{Notes}

\begin{itemize}
  \item Volume means any lower level division (file, bundle, box). Use whatever system the archive uses.
  \item Abbreviate folio to fo. (or fos.). Superscript \textsuperscript{r} and \textsuperscript{v} for recto and verso: fo. 7\textsuperscript{r--v}, fos. 117\textsuperscript{v}--121\textsuperscript{r}.
\end{itemize}

\chapter{Audio and visual broadcasts and recordings}\label{sec:audiovideo}

\begin{description}
  \item[Audio recordings] Oxref provides the additional editor types \code{performer}, \code{director} and \code{reader}.
  \par Composer, \emph{Title}, Artist (Series, Number, Year).
  \\\egcite{couperin1970pdc}
  \par Composer, \emph{Title}, Ensemble, \lit{dir.} Director; Credit (Publisher Type Number, Year).
  \\\egcite{pageNDmn}
  \par Composer, \emph{Title}, Artist (Publisher Series, Number, Year) (Medium).
  \\\egcite{hopkinsNDcar}
  \par Author, \emph{Title}, \lit{read by} Reader (Medium, Number, Year).
  \\\egcite{lewis1981lww}
  \par Speaker, \emph{Title} (Publisher, Number, Year).
  \item[Video recordings]~
  \par \emph{Title} (Publisher, Number, Year).
  \par \enquote{EpisodeTitle}, \emph{SeriesTitle} \lit{series} (Publisher, Number, Year).
  \par Director (\lit{dir.}), \emph{Title} (Publisher, Number, Year).
  \par \emph{Title}, \lit{dir.} Director (Publisher, Number, Year).
  \par \emph{Title} (Publisher, Number, Year), Credit.
  \item[Broadcasts]~
  \par \emph{Title}, Channel, Date.
  \par \enquote{EpisodeTitle}, \emph{SeriesTitle}, \lit{Episode} EpisodeNumber, Channel, Date.
\end{description}

\section{Notes}

\begin{itemize}
  \item Credit means a string like \enquote{edition and project supervision by Editor}.
\end{itemize}

\chapter{Electronic data}\label{sec:electronic}

\begin{description}
  \item[Accessible by URL]~
  \par Author, \enquote{SectionTitle}, \emph{Title} [Medium], (Date) $\langle$URL$\rangle$ Division, \lit{accessed} URLDate.
  \\\egcite{quint2001ohm}
  \item[Accessible by DOI]~
  \par Author, \enquote{SectionTitle}, \emph{Title} [Medium], (Date) $\langle$\lit{DOI} DOI$\rangle$ Division.
\end{description}

\section{Notes}

\begin{itemize}
  \item Indicate missing date of publication with n.d.
  \item Clarify meaning of date if needed: posted 10 Oct. 2001.
  \\\egcite{ritter2001aog}
  \item Access date can be omitted if publication date is given.
  \item Division can include instruction to search: s.v. \enquote{trignonometry}.
\end{itemize}

\chapter{General formatting}

\section{Citation and bibliography styles}

The preferred method of citation in the Oxford style is by footnotes or endnotes; this is the method employed by the Oxnotes citation style and assumed by the Oxnotes bibliography style. The sections below specify the form taken by citation notes. References in a bibliography are formatted in the same way except for one specific change: authors' names are inverted and printed in full and small capitals.

\begin{itemize}
  \item
  \textsc{Author, First}, \lit{and} \textsc{Author, Second}, \emph{Title} (Location: Publisher, Year).
\end{itemize}

\section{Author's Name}

\begin{itemize}
  \item Author names are not inverted (except in the bibliography).
  
  \item Full names are not reduced to initials, but if they are given in the work as initials they need not be expanded.
  
  \item No more than three authors are given:
  \begin{enumerate}
    \item First Author
    \\\egcite{eliot1977m}
    
    \item First Author and Second Author
    \\\egcite{rouse.rouse1991btu}
    
    \item First Author, Second Author, and Third Author
    \\\egcite{brady.etal1994hehsa}
    
    \item First Author et al.
    \\\egcite{stewart.etal1994mb} % This is actually five authors
  \end{enumerate}
  
  \item If adding a famous pseudonym to a work published under a real name: Real Name (Pen Name). \aside{Use \textsf{biblatex} field \code{nameaddon} for the alternative name.}\\\egcite{dodgson1896sl}
  
  \item Pseudo-Name should convert to \textsc{Name (Ps.-)} or \textsc{Name (Pseudo-)} in bibliography. \aside{To do, low priority}
  
  \item If square brackets are used for inferred authorship \aside{by giving as the value of the \textsf{biblatex} field \code{authortype} the definition of the macro \texttt{\textbackslash isInferred}, by default \enquote{inferred}}, the name should be alphabetized as if the brackets weren't there. Repetitions still get replaced by a rule.\\\egcite{balfour1768pe}
  
  \item Anonymous works: Author missed out from footnotes, can be replaced by \enquote{Anonymous} in bibliography. \aside{To do}
  \\\egcite{anon1822san}
  \\\egcite{anon1956lu}
\end{itemize}

\section{Titles and subtitles}

\begin{itemize}
  \item Titles are not converted to lowercase.
  
  \item Subtitles are separated from titles by a colon.
  
  \item Subtitles are given an initial capital.
  \par\egcite{bettelheim1976uem}
\end{itemize}

\section{Publishers}

\begin{itemize}
  \item If the place of publication is absent, replace with n.p.
\end{itemize}

\aside{Publisher names may be uniformly omitted from all citations and bibliography entries. There may be a package option to switch off publisher names in future.}

\section{Dates}

\begin{itemize}
  \item Months should be abbreviated to three or four letters: Apr.\ May June July Sept. % Already so by default.
\end{itemize}

\section{Foliation}

\begin{itemize}
  \item Page number format: 32.
  \item Volume and page number format: iv.\ 32.
  \item Volume and preliminary page number format: vol.\ iv, p.\ xxxii.
  \item Page ranges should be compact: 32--3.\\\egcite{dean1995rpb}
  \item If a particular page is required the first time a source is cited, it should be given after the full page range: 259--338 at 332.\\\egcite[332]{rouse.rouse1991btu}
\end{itemize}

\section{Repeated citations}

Oxford style does not prescribe a particular way of abbreviating the second and subsequent citations of a work, but does suggest the following shorthand expressions for use with closely proximate repeated citations:

\begin{itemize}
  \item Ibid. = in the same place = same author and title as immediately preceding citation.
  \item Author, op. cit. = in the cited work = same title as the last cited work by the specified author. Best suited to occasions where a work is being singled out from a recently cited list of works. 
  \item Author, art. cit. = in the cited article = same title as the last cited article by the specified author. Best suited to occasions where a work is being singled out from a recently cited list of works.
  \item Author, loc. cit. = at the cited place = same title and page range as the last cited work by the specified author. Rarely needed.
  \item Id. = same = same male author as the immediately preceding citation.
  \item Ead. = same = same female author as the immediately preceding citation.
  \item Eid. = same = same male authors, or male and female authors, as the immediately preceding citation.
  \item Eaed. = same = same female authors as the immediately preceding citation.
\end{itemize}
|

%%%%%%%%%%%%%%%%%%%%%%%%%%%%%%%%%%%%%%%%%%%%%%%%%%%%%%%%%%%%%%%%%%%%%%%%
%%% STYLE DOCUMENTATION -- REFERENCE SAMPLES
%%%%%%%%%%%%%%%%%%%%%%%%%%%%%%%%%%%%%%%%%%%%%%%%%%%%%%%%%%%%%%%%%%%%%%%%

\todoc|
\chapter{Sample references}

Below are a series of sample references taken from the \emph{Oxford Style Manual} (in black), each followed by a corresponding reference (in colour) generated using the \textsf{oxref} style. \textcolor{Green}{Green} references can be generated using standard \textsf{biblatex} semantics in the \texttt{.bib} file; \textcolor{red}{red} or missing references aren't currently supported properly; \textcolor{orange}{orange} references require some cheating in the \texttt{.bib} file, and so would have to be altered should a different style be chosen. The database key for each reference is given in the margin to make it easier to look up in the sample database. The section numbers refer to the location of the references in the \emph{Oxford Style Manual}.
|

\bibexample<OGS \S15.1>
|Mary A. Rouse and Richard H. Rouse, \enquote{The Book Trade at the University of Paris, c.1250--c.1350}, in eid., \emph{Authentic Witnesses: Approaches to Medieval Texts and Manuscripts} (Publications in Medieval Studies, 17; Notre Dame, IN, 1991), 259--338 at 332.|%
||%
|@inbook{rouse.rouse1991btu,
  author = {Mary A. Rouse and Richard H. Rouse},
  gender = {pp},
  title = {The Book Trade at the University of Paris, c.1250--c.1350},
  bookauthor = {Mary A. Rouse and Richard H. Rouse},
  booktitle = {Authentic Witnesses},
  booksubtitle = {Approaches to Medieval Texts and Manuscripts},
  series = {Publications in Medieval Studies},
  number = {17},
  location = {Notre Dame, IN},
  date = {1991},
  pages = {259--338}
}|[332]


\bibexample<OGS \S15.2.1>
|Gerald de Vaucouleurs et al., \enquote{The New Martian Nomenclature of the International Astronomical Union}, \emph{Icarus}, 26 (1975), 85--98.|%
||%
|@article{vaucouleurs1975nmn,
  author = {Gerald de Vaucouleurs and J. Blunck and M. Davies and A. Dollfus and I. Koval and G. Kuiper and H. Masursky and S. Miyamoto and V Moroz and Carl Sagan},
  title = {The New {Martian} Nomenclature of the {International} {Astronomical} {Union}},
  journaltitle = {Icarus},
  volume = {26},
  date = {1975},
  pages = {85-98}
}|


\bibexample<OGS \S15.2.1>
|C. L. Dodgson (Lewis Carroll), \emph{Symbolic Logic} (Oxford: Clarendon Press, 1896).|%
||%
|@book{dodgson1896sl,
  author = {C. L. Dodgson},
  nameaddon = {Lewis Carroll},
  title = {Symbolic Logic},
  location = {Oxford},
  publisher = {Clarendon Press},
  date = {1896}
}|

% Do we need special handling for whether a work is marked 'Anon'?
\bibexample<OGS \S15.2.1>
|\emph{Stories after Nature} (London: Allman, 1822).\quad [\enquote{Anonymous} in Bibliography]|%
||%
|@book{anon1822san,
  title = {Stories after Nature},
  location = {London},
  publisher = {Allman},
  date = {1822}
}|

\bibexample<OGS \S15.2.1>
|\emph{Liber usualis} (Tournai: Desclée, 1956).|%
||%
|@book{anon1956lu,
  title = {Liber usualis},
  location = {Tournai},
  publisher = {Desclée},
  date = {1956}
}|





\bibexample<OGS \S15.3>
|John Shearman, \enquote{The Vatican Stanze: Functions and Decoration}, in George Holmes (ed.), \emph{Art and Politics in Renaissance Italy: British Academy Lectures} (Oxford: Clarendon Press, 1993), 185--240.|%
||%
|@incollection{shearman1993vsf,
  author = {John Shearman},
  title = {The Vatican Stanze},
  subtitle = {Functions and Decoration},
  editor = {George Holmes},
  booktitle = {Art and Politics in Renaissance Italy},
  booksubtitle = {British Academy Lectures},
  location = {Oxford},
  publisher = {Clarendon Press},
  date = {1993},
  pages = {185-240}
}|

\bibexample<OGS \S15.3>
|W. B. Todd, \enquote{David Hume: A Preliminary Bibliography}, in id. (ed.), \emph{Hume and the Enlightenment: Essays Presented to Ernest Campbell Mossner} (Edinburgh: Edinburgh University Press, 1974).|%
||%
|@incollection{todd1974dhp,
  author = {W. B. Todd},
  gender = {sm},
  title = {David Hume},
  subtitle = {A Preliminary Bibliography},
  editor = {W. B. Todd},
  booktitle = {Hume and the Enlightenment},
  booksubtitle = {Essays Presented to Ernest Campbell Mossner},
  location = {Edinburgh},
  publisher = {Edinburgh University Press},
  date = {1974}
}|

\bibexample<OGS \S15.3>
|G. Frege, \enquote{On Sense and Reference}, in id., \emph{Philosophical Writings}, trans. and ed. P. T. Geach and M. Black (Oxford: Blackwell, 1952) (originally pub. 1892).|%
||%
|@bookinbook{frege1892osr,
  author = {G. Frege},
  gender = {sm},
  title = {On Sense and Reference},
  bookauthor = {G. Frege},
  booktitle = {Philosophical Writings},
  translator = {P. T. Geach and M. Black},
  editor = {P. T. Geach and M. Black},
  location = {Oxford},
  publisher = {Blackwell},
  date = {1952},
  pubstate = {originally pub\adddotspace 1892}
}|

\bibexample<OGS \S15.3>
|G. E. L. Owen, \enquote{Philosophical Invective}, in id., \emph{Logic, Science and Dialectic}, ed. M. Nussbaum (Ithaca, NY: Cornell University Press, 1986), 347--64. From \emph{Oxford Studies in Ancient Philosophy}, 1 (1983), 1--25.|%
||%
|@bookinbook{owen1986pi,
  author = {G. E. L. Owen},
  gender = {sm},
  title = {Philosophical Invective},
  booktitle = {Logic, Science and Dialectic},
  bookauthor = {G. E. L. Owen},
  editor = {M. Nussbaum},
  location = {Ithaca, NY},
  publisher = {Cornell University Press},
  date = {1986},
  pages = {347-364},
  related = {osap1983},
  relatedtype = {reprintfrom}
}
@periodical{osap1983,
  journaltitle = {Oxford Studies in Ancient Philosophy},
  volume = {1},
  date = {1983},
  pages = {1-25}
}|

\todoc|
If not using Biber, instead of the related fields you would use the following:\\
\tcboxverb{addendum = {From \fullcite[1-25]{osap1983}}}
|

% Article in yearbook

\bibexample<OGS \S15.3>
|M. Boyce, \enquote{The Parthian \emph{Gsn} and Iranian Minstrel Tradition}, \emph{Journal of the Royal Asiatic Society}, 1957, 10--45.|%
||%
|@article{boyce1957pgi,
  author = {M. Boyce},
  title = {The Parthian \emph{Gsn} and Iranian Minstrel Tradition},
  journaltitle = {Journal of the Royal Asiatic Society},
  date = {1957},
  pages = {10-45}
}|

\bibexample<OGS \S15.4.1>
|Halil Inalcik, \enquote{Comments on \enquote{Sultanism}: Max Weber's Typification of the Ottoman Polity}, \emph{Princeton Papers in Near Eastern Studies}, 1 (1992), 49--72.|%
||%
|@article{inalcik1992csm,
  author = {Halil Inalcik},
  title = {Comments on \enquote{Sultanism}},
  subtitle = {Max Weber's Typification of the Ottoman Polity},
  journaltitle = {Princeton Papers in Near Eastern Studies},
  volume = {1},
  date = {1992},
  pages = {49-72}
}|

\bibexample<OGS \S15.4.1>
|Robert Goldblatt, \enquote{Diodorean Modality in Minkowski Space-Time}, \emph{Studia Logica}, 39/3 (1973), 219--36.|%
||%
|@article{goldblatt1973dmm,
  author = {Robert Goldblatt},
  title = {Diodorean Modality in Minkowski Space-Time},
  journaltitle = {Studia Logica},
  volume = {39},
  number = {3},
  date = {1973},
  pages = {219-236}
}|

\bibexample<OGS \S15.4.1>
|Alan Lee, \enquote{England Haunted by Familiar Failings}, \emph{The Times}, 23 June 1995.|%
||%
|@article{lee1995ehf,
  author = {Alan Lee},
  title = {England Haunted by Familiar Failings},
  journaltitle = {The Times},
  date = {1995-06-23}
}|

\bibexample<OGS \S15.4.1>
|\enquote{Who's Excellent Now?}, \emph{Business Week}, 5 Nov. 1984, 76--86.|%
||%
|@article{bw1984wen,
  title = {Who's Excellent Now?},
  journaltitle = {Business Week},
  date = {1984-11-05},
  pages = {76-86}
}|

\todoc|
\aside{The following effect was achieved by added \code{\textbackslash nopunct} to the end of the journal title.}
|

\bibexample(hacked)<OGS \S15.4.1>
|B. Lindars, \enquote{Ezechiel and Individual Responsibility}, \emph{VT} 15 (1965), 452--67.|%
||%
|@article{lindars1965eir,
  author = {B. Lindars},
  title = {Ezechiel and Individual Responsibility},
  journaltitle = {VT\nopunct},
  volume = {15},
  date = {1965},
  pages = {452-467},
}|

\todoc|
\aside{Here the \code{=} and journal title are given as the \code{titleaddon}, and the issue information as the \code{note}.}
|

% This one is \emph{weird}
\bibexample(hacked)<OGS \S15.4.1>
|C. Bec (ed.), \emph{Italie 1500--1550: Une situation de crise?} = \emph{Annales de l'Université Jean Moulin}, 1975/2 (Langues étrangères, 2; Lyon, 1976), 99--109.|%
||%
|@collection{bec1976isc,
  editor = {C. Bec},
  title = {Italie 1500--1550},
  subtitle = {Une situation de crise?\nopunct},
  titleaddon = {= \emph{Annales de l'Université Jean Moulin}},
  note = {1975/2},
  series = {Langues étrangères},
  number = {2},
  location = {Lyon},
  date = {1976},
  pages = {99--109}
}|

\bibexample<OGS \S15.4.1>
|F. Trisoglio, \emph{Gregorio di Nazianzo in un quarentennio di recherche (1925--1965)} = \emph{Rivista Iasalliana}, 40 (1973).|%
||%
|@article{trisoglio1973gnq,
  author = {F. Trisoglio},
  issuetitle = {Gregorio di Nazianzo in un quarentennio di recherche (1925--1965)},
  journaltitle = {Rivista Iasalliana},
  volume = {40},
  date = {1973}
}|

\bibexample<OGS \S15.4.1>
|X. Zhentao, K. K. C. Yau, and F. R. Stephenson, \enquote{Astronomical Records on the Shang Dynasty Oracle Bones}, \emph{Archaeoastronomy}, 14 (1989); Supplement to \emph{Journal for the History of Astronomy}, 20, pp. S61--S72.|%
||%
|@article{zhentao.etal1989ars,
  author = {X. Zhentao and K. K. C. Yau and F. R. Stephenson},
  title = {Astronomical Records on the Shang Dynasty Oracle Bones},
  journaltitle = {Archaeoastronomy},
  volume = {14},
  date = {1989},
  note = {Supplement to \fullcite{jha1989}},
  pages = {\ppno~S61-S72}
}
@periodical{jha1989,
  title = {Journal for the History of Astronomy},
  volume = {20}
}|

\bibexample<OGS \S15.4.3>
|J. Barnes, \enquote{Homonymy in Aristotle and Speusippus}, \emph{Classical Quarterly}, \textsc{ns} 21 (1971), 65--80.|%
||%
|@article{barnes1971has,
  author = {J. Barnes},
  title = {Homonymy in Aristotle and Speusippus},
  journaltitle = {Classical Quarterly},
  series = {newseries},
  volume = {21},
  date = {1971},
  pages = {65--80}
}|

% With Biber: using related field mechanism
\bibexample<OGS \S15.4.4>
|J. Dean, review of Philippe Basiron, \emph{My Early Life} (Bourges, 1994), in \emph{Res facta}, 17 (1995), 56--9.|%
||%
|@review{dean1995rpb,
  author = {J. Dean},
  related = {basiron1994mel},
  relatedtype = {reviewof},
  journaltitle = {Res facta},
  volume = {17},
  date = {1995},
  pages = {56-59}
}
@book{basiron1994mel,
  author = {Philippe Basiron},
  title = {My Early Life},
  location = {Bourges},
  year = {1994}
}|

% Without Biber: need special field for reviewed document
\bibexample<OGS \S15.4.4>
|H. D. Jocelyn, \enquote{Probus and Virgil}, review of Maria Luisa Delvigo, \emph{Testo virgiliano e tradizione indiretta} (Pisa, 1987), in \emph{CR}, \textsc{ns} 39 (1989), 27--8.|%
||%
|@review{jocelyn1989pav,
  author = {H. D. Jocelyn},
  title = {Probus and Virgil},
  options = {reviewed=delvigo1987tvt},
  journaltitle = {CR},
  series = {newseries},
  volume = {39},
  date = {1989},
  pages = {27-28}
}
@book{delvigo1987tvt,
  author = {Maria Luisa Delvigo},
  title = {Testo virgiliano e tradizione indiretta},
  location = {Pisa},
  date = {1987}
}|

\bibexample<OGS \S15.5.1>
|Bonnie J. Blackburn, \enquote{The Lupus Problem}, Ph.D. diss. (University of Chicago, 1970).|%
||%
|@thesis{blackburn1970tlp,
  author = {Bonnie J. Blackburn},
  title = {The Lupus Problem},
  type = {Ph.D. diss.},
  institution = {University of Chicago},
  date = {1970}
}|

\bibexample<OGS \S15.6>
|Leofranc Holford-Strevens, \enquote{Humanism and the Language of Music Theory Treatises}, paper given at the 65th Annual Meeting of the American Musicological Society, Kansas City, MO, 4--7 Nov. 1999.|%
||%
|@unpublished{holfordstrevens1999hlm,
  author = {Leofranc Holford-Strevens},
  title = {Humanism and the Language of Music Theory Treatises},
  howpublished = {paper given at the 65th Annual Meeting of the American Musicological Society},
  location = {Kansas City, MO},
  date = {1999-11-04/1999-11-07}
}|

\bibexample<OGS \S15.6>
|E. H. Gombrich, \emph{Art and Scholarship}, Inaugural Lecture, University College London, 14 Feb. 1957 (London: H. K. Lewis, 1957).|%
||%
|@book{gombrich1957as,
  author = {E. H. Gombrich},
  title = {Art and Scholarship},
  note = {Inaugural Lecture, University College London, 14 Feb. 1957},
  location = {London},
  publisher = {H. K. Lewis},
  date = {1957}
}|

\todoc|
\section*{15.8} % Poems

Spenser, \emph{Faerie Queene}, II. \textsc{vi}. xxxv. 7

Spenser, \emph{Faerie Queene}, bk. ii, canto vi, stanza iii

W. H. Auden, \enquote{Es regnet auf mir in den Schottische Lände} [\emph{sic}], in \enquote{The German Auden: Six Early Poems}, trans. David Constantine, in Katherine Bucknell and Nicholas Jenkins (eds.), \emph{W. H. Auden, \enquote{The Map of All my Youth}: Early Works, Friends, and Influences} (Auden Studies, 1; Oxford, 1990), 1--15 at 6.

William of Blois, \enquote{The Quarrel of the Flea and the Fly} (\emph{Pulicis et musce iurgia}), trans. in Jan M. Ziolkowski, \emph{Talking Animals: Medieval Latin Beast Poetry, 750-1150} (Middle Ages Series, ed. Edward Peters; Philadelphia: University of Pennsylvania Press, 1993), 274--8.
|

\bibexample<OGS \S15.14.1>
|Francois Couperin, \emph{Pièces de clavecin: Huit préludes de L'Art de toucher le clavecin. Livre I. Troisième et quatrième ordres}, Huguette Dreyfus (Valois, MB 797, 1970).|%
||%
|@music{couperin1970pdc,
  author = {Francois Couperin},
  title = {Pièces de clavecin},
  subtitle = {Huit préludes de L'Art de toucher le clavecin. Livre I. Troisième et quatrième ordres},
  editor = {Huguette Dreyfus},
  editortype = {performer},
  series = {Valois},
  number = {MB 797},
  date = {1970}
}
|

\bibexample<OGS \S15.14.1>
|Paul Hillier, \emph{Proensa} (ECM Records compact disc ECM 1368, 1989).|%
||%
|@music{hillier1989p,
  author = {Paul Hillier},
  title = {Proensa},
  publisher = {ECM Records},
  type = {compact disc},
  number = {ECM 1368},
  date = {1989}
}|

\bibexample<OGS \S15.14.1>
|Lightnin' Hopkins, \emph{The Complete Aladdin Recordings} (EMI Blues Series, CDP-7-96843-2) (two-volume compact disc set).|%
||%
|@music{hopkinsNDcar,
  author = {Lightnin’ Hopkins},
  title = {The Complete Aladdin Recordings},
  publisher = {EMI},
  series = {Blues Series},
  number = {CDP-7-96843-2},
  pubstate = {two-volume compact disc set}
}|

\bibexample<OGS \S15.14.1>
|\emph{The Mirror of Narcissus: Songs by Guillaume de Machaut}, Gothic Voices, dir. Christopher Page (Hyperion compact disc CDA 66087).|%
||%
|@music{pageNDmn,
  title = {The Mirror of Narcissus},
  subtitle = {Songs by Guillaume do Machaut},
  editor = {Gothic Voices},
  editortype = {performer},
  editora = {Christopher Page},
  editoratype = {director},
  publisher = {Hyperion},
  type = {compact disc},
  number = {CDA 66087}
}|

\bibexample<OGS \S15.14.1>
|C. S. Lewis, \emph{The Lion, the Witch, and the Wardrobe}, read by Sir Michael Hordern (2 audio cassettes, TO1611, 1981).|%
||%
|@audio{lewis1981lww,
  author = {C. S. Lewis},
  title = {The Lion, the Witch, and the Wardrobe},
  editor = {Sir Michael Hordern},
  editortype = {reader},
  howpublished = {2 audio cassettes},
  number = {TO1611},
  date = {1981},
}|

\todoc|
\section*{15.14.2} % Films, video recordings, and slides

\emph{The Ashes: Victory in Australia} (BBCV 4040, 1987).

Charles Chaplin (dir.), \emph{Modern Times} (United Artists, 1936).

\emph{Leningrad Cowboys Go America} (Villealfa Filmproductions, 1989), dir.\ and screenplay by Aki Kaurismäki, story by Sakke Järvenpää, Aki Kaurismäki, Mato Valtonen.

\enquote{Percy and Harold and Other Stories}, \emph{Thomas the Tank Engine and Friends} series (Britt Allcroft 5-014861-100224, 1986).

\emph{This is Spinal Tap: A Rockumentary by Marti Di Bergi}, dir.\ Rob Reiner (Embassy Pictures, 1983).

\section*{15.14.3} % Television and radio programmes

\emph{Sunday Grandstand}, BBC2, 17 Sept. 1995.
|

\bibexample<OGS \S15.15.2>
|Barbara Quint, \enquote{One Hour to Midnight: \emph{Tasini} Oral Arguments at the Supreme Court}, \emph{Information Today} [online journal], 18/5 (May 2001) $\langle$\url{http://www.infotoday.com/newsbreaks/nb010330-1.htm}$\rangle$, accessed 1 July 2001.|%
||%
|@article{quint2001ohm,
  author = {Barbara Quint},
  title = {One Hour to Midnight: \emph{Tasini} Oral Arguments at the Supreme Court},
  journaltitle = {Information Today},
  entrysubtype = {online journal},
  volume = {18},
  number = {5},
  date = {2001-05},
  url = {http://www.infotoday.com/newsbreaks/nb010330-1.htm},
  urldate = {2001-07-01}
}|

\bibexample<OGS \S15.15.3>
|R. M. Ritter, \enquote{About the Oxford Guide to Style}, \emph{Oxford Editorial} [website], (updated 10~Oct.\ 2001) $\langle$\url{http://www.ritter.org.uk/Oxford_Editorial/AboutOGS.html}$\rangle$|%
||%
|@online{ritter2001aog,
  author = {R. M. Ritter},
  title = {About the Oxford Guide to Style},
  maintitle = {Oxford Editorial},
  entrysubtype = {website},
  date = {2001-10-10},
  options = {datetype=updated},
  url = {http://www.ritter.org.uk/Oxford_Editorial/AboutOGS.html}
}|

\bibexample<OGS \S15.15.3>
|William Strunk, Jr., \emph{The Elements of Style} (1st edn., Geneva, NY: W. P. Humphrey, 1918), published online July 1999 $\langle$\url{http://www.bartleby.com/141}$\rangle$|%
||%
|@book{strunk1999tes,
  author = {Strunk, Jr., William},
  title = {The Elements of Style},
  edition = {1},
  location = {Geneva, NY},
  publisher = {W. P. Humphrey},
  date = {1918},
  url = {http://www.bartleby.com/141},
  urldate = {1999-07},
  options = {urldatetype=published{\space}online}
}|

\bibexample<OGS \S15.19.2>
||%
|Smith, T. (1988), \emph{Famine and Gender} (New Haven, Conn.).|%
|@book{smith1988fag,
  author = {Smith, T.},
  title = {Famine and Gender},
  year = {1988},
  location = {New Haven, Conn.}
}|

%%%%%%%%%%%%%%%%%%%%%%%%%%%%%%%%%%%%%%%%%%%%%%%%%%%%%%%%%%%%%%%%%%%%%%%%
%%% SAMPLE BIB FILE
%%%%%%%%%%%%%%%%%%%%%%%%%%%%%%%%%%%%%%%%%%%%%%%%%%%%%%%%%%%%%%%%%%%%%%%%

\begin{writeverbatim}{src}

% Extensions

@inproceedings{tonkin.strelnikov2009iem,
  author = {Emma Tonkin and Alexey Strelnikov},
  title = {Information Environment Metadata Schema Registry},
  editor = {Maristella Agosti and José Borbinha and Sarantos Kapidakis and Christos Papatheodorou and Giannis Tsakonas},
  booktitle = {Research and Advanced Technology for Digital Libraries},
  eventtitle = {13th European Conference, ECDL 2009},
  venue = {Corfu, Greece},
  eventdate = {2009-09-27/2009-10-02},
  series = {Lecture Notes in Computer Science},
  number = {5714},
  location = {Berlin},
  publisher = {Springer},
  date = {2009},
  pages = {487-488},
  isbn = {978-3-642-04345-1},
  issn = {0302-9743}
}

@inproceedings{ding.etal2009sfc,
  author = {Lian Ding and Alex Ball and Manjula Patel and Jason Matthews and Glen Mullineux},
  title = {Strategies for the Collaborative Use of CAD Product Models},
  maintitle = {Proceedings of the 17th International Conference on Engineering Design (ICED ’09)},
  booktitle = {Design Information and Knowledge},
  volume = {8},
  editor = {Margareta Norell Bergendahl and Martin Grimheden and Larry Leifer},
  venue = {Stanford, CA},
  eventdate = {2009-08-24/2009-08-27},
  location = {Glasgow},
  publisher = {Design Society},
  date = {2009},
  pages = {123-134},
  isbn = {978-1-904670-12-4},
  url = {http://opus.bath.ac.uk/14285}
}
\end{writeverbatim}

%%%%%%%%%%%%%%%%%%%%%%%%%%%%%%%%%%%%%%%%%%%%%%%%%%%%%%%%%%%%%%%%%%%%%%%%
%%% STYLE DOCUMENTATION -- ENDMATTER
%%%%%%%%%%%%%%%%%%%%%%%%%%%%%%%%%%%%%%%%%%%%%%%%%%%%%%%%%%%%%%%%%%%%%%%%

\todoc|
\printbibliography[notcategory=reviewed]
\end{document}
|

\closeoutputstream{oxnotes}
\closeoutputstream{oxyear}
\closeoutputstream{src}

%%%%%%%%%%%%%%%%%%%%%%%%%%%%%%%%%%%%%%%%%%%%%%%%%%%%%%%%%%%%%%%%%%%%%%%%
%%% COMMON PACKAGE DOCUMENTATION
%%%%%%%%%%%%%%%%%%%%%%%%%%%%%%%%%%%%%%%%%%%%%%%%%%%%%%%%%%%%%%%%%%%%%%%%

% For typesetting the implementation
\usepackage{metalogo}
\usepackage{makedoc}[2012/04/03]
\usepackage{niceverb}[2012/08/27]
\renewcommand{\SetOffInlineCmdBox}{\textcolor{red!75!black}}
\renewcommand*{\TreatAsCode}[1]{%
  \ifPackageCode
  \else
  \WriteResult{\string\begin{tcblisting}{breakable,listing only,listing options={firstnumber=last}}}%
  \PackageCodeTrue
  \fi
  \WriteResult{#1}%
}
\renewcommand*{\TreatAsComment}[1]{%
  \ifPackageCode
  \WriteResult{\string\end{tcblisting}}%
  \WriteResult{}
  \PackageCodeFalse
  \EveryComment
  \fi
  \WriteResult{#1}%
}
\makeatletter
\PushCatMakeLetter\_
\renewcommand*{\make_doc_arg}[1]{%
  \makedoc_maybe_autocorr
  \ifnum\value{lstnumber}>\z@
  \WriteResult{\string\begin{tcblisting}{breakable,listing only,listing options={firstnumber=1}}}%
  \PackageCodeTrue
  \else
  \PackageCodeFalse
  \fi
  \ProcessFileWith{#1}{%
  \CountInputLines
  \makedoc_line_body
  \process_line_message
  }%
  \ifPackageCode
  \WriteResult{\string\end{tcblisting}}
  \PackageCodeFalse
  \fi
}
\PopLetterCat\_
\makeatother

\title{Oxref: Biblatex styles inspired by the Oxford Guide to Style}
\author{Alex Ball}

\providecommand*{\listinglabel}[1]{%
  \llap{\scriptsize\itshape\the#1}\hskip\listingoffset\relax
}%
\providecommand*{\TODO}{\textcolor{blue}{TODO}}
 

\begin{document}
\thispagestyle{empty}\noNiceVerb
\begin{adjustwidth}{.2\textwidth}{0pt}
  \sffamily\setlength{\parindent}{0pt}%
  \LARGE\textsc{oxref bundle}
  
  \vspace{\stretch{1}}
  \LARGE OXREF
  
  \Large Biblatex styles inspired by the \emph{Oxford Guide to Style}
  
  \bigskip
  \Large\theauthor
  
  \bigskip
  \Large\ifx\undefined\thedate\today\else\thedate\fi
\end{adjustwidth}

\vspace{\stretch{3}}
\noindent
\hspace*{.1\textwidth}\raisebox{0pt}[0pt][0pt]{\rule{\normalrulethickness}{\textheight}}

\chapter{Introduction}

This document provides the documented sources for the \textsf{oxref}
biblography styles:
\begin{itemize}
  \item
  \textsf{oxnotes}: a style similar to the standard \textsf{verbose}, intended for use with footnotes;
  \item
  \textsf{oxyear}: a style similar to the standard \textsf{authoryear}, intended for use with parenthetical in-text citations.
\end{itemize}
For information on how to use the styles, please refer
to the separate documentation.

\lstset
  { basicstyle=\footnotesize\ttfamily
  , breaklines=true
  , escapechar=
  , numbers=left
  , numberstyle=\footnotesize\itshape\color{gray}
  , numbersep=20pt
  }%
\useNiceVerb
\bgroup\let\tt\ttfamily

\chapter{oxnotes.bbx}

\renewcommand*{\mdJobName}{oxref}
\MakeInputJobDoc[bbx]{0}{\ProcessInputWith{PScomment}}

\chapter{oxnotes.bbx}

\renewcommand*{\mdJobName}{oxnotes}
\MakeInputJobDoc[bbx]{0}{\ProcessInputWith{PScomment}}

\chapter{oxyear.bbx}

\renewcommand*{\mdJobName}{oxyear}
\MakeInputJobDoc[bbx]{0}{\ProcessInputWith{PScomment}}

\chapter{british-oxref.lbx}

\renewcommand*{\mdJobName}{british-oxref}
\MakeInputJobDoc[lbx]{0}{\ProcessInputWith{PScomment}}

\egroup
\noNiceVerb
\end{document}